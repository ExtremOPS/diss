\usepackage{geometry}
\geometry{a4paper,top=30mm,left=22mm,right=22mm,bottom=30mm,headsep=10mm,footskip=12mm}
\renewcommand{\baselinestretch}{1.05}\normalsize %Zeilenabstand
%vor Boston: 1.1 Zeilenabstand, Grösse 13 pt top=30mm,left=18mm,right=18mm,bottom=30mm,headsep=10mm,footskip=12mm

%distance between floats on the top or the bottom and the text, standard: 20.0pt plus 2.0pt minus 4.0pt
\setlength{\textfloatsep}{7pt plus 1.0pt minus 2.0pt}
% distance between two floats, standard: 12.0pt plus 2.0pt minus 2.0pt:
\setlength{\floatsep}{7pt plus 1.0pt minus 2.0pt}
%distance between floats inserted inside the page text (using h) and the text, standard: 12.0pt plus 2.0pt minus 2.0pt:
\setlength{\intextsep}{7pt plus 1.0pt minus 2.0pt}

% indent every paragraph
\usepackage{indentfirst}

% math packages
\usepackage{amsmath,amssymb}
% :=
\usepackage{mathtools}
% nice fractions of quotients
\usepackage{xfrac}

\usepackage[colorlinks = true,
            linkcolor = blue,
            urlcolor  = blue,
            citecolor = blue,
            anchorcolor = blue]{hyperref}
% must be after hyperref and amsmath
\usepackage[nameinlink]{cleveref}
  \crefname{figure}{FIG.}{FIGs.}
  \crefname{table}{TAB.}{TABs.}
  \crefname{equation}{EQ.}{EQs.}
  \crefname{section}{Section}{Section}

\usepackage{floatflt}
\usepackage{moresize}
\usepackage[]{ethimes}               % New styles and commands
\setcounter{tocdepth}{3} \setcounter{secnumdepth}{3}
% heading
\usepackage{fancyhdr}
\usepackage{graphicx}
  \graphicspath{
      {SECTION/20_Theory/10_Figures/}
      {SECTION/30_Timeconstant/10_Figures/LaTeX/}
      {SECTION/30_Timeconstant/10_Figures/PGF/}
      {SECTION/30_Timeconstant/10_Figures/}
    }
\usepackage[dvips]{epsfig}
\usepackage{epstopdf}
% Placement of floating objects
\usepackage{float}
% Placement of floating objects
\usepackage{here}
% lipsum text
\usepackage{lipsum}
\usepackage[ngerman,spanish,english]{babel}
\usepackage[utf8]{inputenc}\DeclareUnicodeCharacter{2212}{-}
\usepackage[T1]{fontenc}

% siunit typing
\usepackage{siunitx}
  \DeclareSIUnit\samples{S} % frames per second
  \DeclareSIUnit\MS{\mega\samples\per\second} % frames per second
  % \DeclareSIUnit\Vrms{\volt_{rms}} % volt RMS
  \DeclareSIUnit\Vrms{V_{rms}} % volt RMS
  \DeclareSIUnit\fps{fps} % frames per second
  \DeclareSIUnit\mW{\milli\watt}
% colors
\usepackage[table,xcdraw]{xcolor}

% table configurations
\usepackage{tabularx}
\usepackage{array} % table addition
\usepackage{rotating} % Rotating table
\usepackage{booktabs} % \bottomrule
  \newcolumntype{x}[1]{>{\centering\arraybackslash\hspace{0pt}}p{#1}}

% citing
\usepackage[
  backend=biber,
  hyperref,
  giveninits=true,
  maxcitenames=2,
  doi=true,
  url=false,
    ]{biblatex}
\addbibresource{All.bib}
\addbibresource{supplemental.bib}

\usepackage{csquotes}

\newcommand\blfootnote[1]{%
  \begingroup
  \renewcommand\thefootnote{}\footnote{#1}%
  \addtocounter{footnote}{-1}%
  \endgroup
}

% physical writing
\usepackage{physics}

% subfigures
\usepackage[
  textformat=simple,
  labelfont=bf,
  format=hang]{caption}  %Caption options
\usepackage{afterpage}
\usepackage{subcaption}

% bold math
\usepackage{bm}

% nomenclature
\usepackage{nomencl}
\makenomenclature

%% This code creates the groups for the nomenclature
% -----------------------------------------
\usepackage{etoolbox}
\renewcommand\nomgroup[1]{%
  \item[\bfseries
  \ifstrequal{#1}{R}{Roman Alphabet}{%
  \ifstrequal{#1}{G}{Greek Alphabet}{%
  \ifstrequal{#1}{O}{Operators}{%
  \ifstrequal{#1}{A}{Abbreviations}{%
  \ifstrequal{#1}{S}{Sub- and Superscripts}{}}}}}%
]}
% -----------------------------------------

%% This will add the units
%----------------------------------------------
\newcommand{\nomunit}[1]{%
  \renewcommand{\nomentryend}{\hspace*{\fill}#1}
}
%----------------------------------------------
