\section{Conclusion\label{sec:VT-conclusion}}

The combination of an optical trap and a transparent VT device opened the 
possibility to measure the VT independently of the acoustic radiation force. The 
power spectrum analysis provided the quantitative information about the angular 
frequency $\Omega$. Unwanted effects related to close proximities of walls near 
the rotating particle and influences of oscillating micro gas bubbles were 
avoided.  The optically levitated particle ensured a largely unaffected rotation 
due to the acoustic VT.\@ 

Moving the stage of the optical set-up changed the rotation direction of a 
trapped and rotated particle between two neighboring pressure nodes because of 
the acoustic VT \cite{Lamprecht2015} (see the supplemental material for a 
particle rotation in different directions depending on the spatial location 
inside the wave field).  These kinds of experiments were so far unattainable in 
a continuous manner and for unstable acoustic particle positions (positive 
acoustic contrast factor) of zero VT.\@

The validation experiments showed that the detected additional peaks in the 
measured power spectrum are directly related to the rotational rate of the 
particle rotation. The detected signals had peaks at multiples of this peak 
frequency.  However, for transparent silica particles with an almost perfectly 
spherical shape the amplitudes of the multiples were too small to overcome the 
signal-to-noise-ratio. The high-speed video analysis is limited by the camera's 
frame rate. In contrast, the detection bandwidth of the optical trap easily 
spans tens of \si{\kilo\hertz}. As already mentioned, optical trapping has some 
unique advantages to investigate the acoustic VT: 1) Allows to probe almost  any 
spatial position within the acoustofluidic device. 2) Measures rotational 
frequencies up to multiple \si{\kilo\hertz}. 3) Works with conventional, non 
coated, spherical particles. 4) Frequencies are directly visible in the data (no 
need for post processing of video data).

In order to calculate the theoretical rotational rate $\finalOmega$ of the 
particle, the local acoustic pressure amplitude needs to be known in advance.  
Because of that, a local acoustic pressure amplitude analysis was carried out 
before the rotation detection experiments. Depending on the calculation 
approach, different results are obtained for the rotational rate. The analytical 
calculation for the final rotational rate $\finalOmega$ (see \cref{eq:VT-Eq1}) 
\cite{Lamprecht2015, Busse1981, Rudnick1977, Wang1989} with a dynamic fluid viscosity of 
$\mu_{f} = \SI{0.06}{\pascal\second}$ led to rates between 
\SIrange{613}{811}{\hertz} (\SIrange{36.8e3}{48.7e3}{\rpm}). The numerical 
calculations of \citeauthor{Hahn2016} \cite{Hahn2016} yield a final rotational 
rate for a \SI{2.06}{\micro\meter} silica particle with $\normBdLayer=1.30$ of 
\SIrange{208}{277}{\hertz} (\SIrange{12.5e3}{16.6e3}{\rpm}) at room temperature 
(\SI{25}{\celsius}).  The first value for the rotational rate corresponds each 
time to the theoretical wavelength of $\lambda \approx \SI{1.4}{\mm}$ (see 
\cref{fig:VT-Fig4}) and acoustic pressure amplitude $p_{a}\left(\lambda\right) = 
\SI{245}{\kilo\pascal} $; the latter to the measured $\lambda \approx 
\SI{1.9}{\mm}$ (see \cref{fig:VT-Fig10}) and $p_{a}\left(\lambda\right) = 
\SI{282}{\kilo\pascal} $. The disagreement between our experiments and 
\cref{eq:VT-Eq1} is regarding the equilibrium state for the final rotational rate 
$\finalOmega$. The spatial dependency of \cref{eq:VT-Eq1} ($\cos\left(k\,X\right), 
\cos\left( k\,Y \right)$) agrees with our experiments.

In contrast to that, the power spectrum-method estimates the steady-state 
rotational rate $\finalOmega$ to \SI{229}{\hertz} (\SI{13.8e3}{\rpm}). This 
value is very close to the numerically obtained values (about 10$\%$ higher for 
$\lambda \approx \SI{1.4}{\mm}$ or 17$\%$ lower for $\lambda \approx 
\SI{1.9}{\mm}$).  These deviations can be in part explained by a decrease of 
the fluid viscosity due to laser-induced heating (up to \SI{2}{\kelvin}) in 
close proximity of the laser focus \cite{Peterman2003}. Since the measured force 
of our trap scales with $\sqrt{\mu}$ and the viscosity variation around 
\SI{25}{\celsius} is relatively small, the temperature induced measurement 
errors are about 2\%.  This slightly changes the acoustic pressure amplitudes 
with respect to the calibrated pressure of \SI{245}{\kilo\pascal} ($\lambda 
\approx \SI{1.4}{\mm}$) or \SI{282}{\kilo\pascal} ($\lambda \approx 
\SI{1.9}{\mm}$) because the investigated pressure node was located slightly off 
the calibration lines.  Also, the in oil immersed lens of the optical trap 
changes the theoretical (pure) 1-dimensional acoustic field to a 3-dimensional.

The experiment clearly showed that the analytical calculations of 
\citeauthor{Lamprecht2015,Busse1981, Rudnick1977, Wang1989} \cite{Lamprecht2015, Busse1981, 
Rudnick1977, Wang1989}  overestimate the rotational velocities at ratios $\normBdLayer 
\approx 1$. Furthermore, the complexity and spatially varying torques on 
\si{\micro\meter} particles were measured, whereas the simulations are limited 
to the ideal case of plane-standing waves and incompressible particles in an 
infinitely large fluid domain. 

A further application of the acoustic torque analysis with an optical trap could 
be the experimental determination of the influence of the particle density on 
the acoustic VT.\@ Particles with the same density can show different rotational 
directions at a fixed point in the acoustic field, if the ratio $\normBdLayer$ 
changes \cite{Hahn2016}.

Optical torques on double refracting quartz particles is a possible tool to 
directly measure acoustic torques \cite{La2004}. The laser power of such 
modulated optical traps could be used to calibrate acoustic torques on trapped 
particles at equilibriums where the optical torque counteracts the acoustic 
torque. 

% \vspace*{7mm}

% A.L and C.G. contributed equally to this work.
