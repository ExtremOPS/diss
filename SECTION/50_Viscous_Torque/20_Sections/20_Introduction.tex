\section{Introduction \label{sec:VT-introduction}}

Ultrasonic particle manipulation offers a wide range of possibilities to handle 
particles suspended in a fluid. They can be aligned 
\cite{Phan2014,Leibacher2015b,Leibacher2015a}, sorted by size 
\cite{Collins2014, Laurell2007}, or separated \cite{Petersson2004, 
Grenvall2009} inside micro-fluidic devices.  Furthermore, particles can be 
moved or rotated \cite{Schwarz2015, Lamprecht2015, Bernard2017b} by changing 
the properties of the exciting standing wave. The ultrasound manipulation 
techniques for \si{\micro\meter}-sized particles gained an increase of interest 
in chip-based micro-robotics \cite{Dual2012, Ahmed2016}, biological analysis 
\cite{Grenvall2009, Sitters2015}, and chemical applications \cite{Evander2012}.  
Mainly the translational behavior was investigated so far.  However, a deeper 
understanding of acoustic particle rotation adds new degrees of freedom to such 
systems.

An acoustic rotation of spherical \si{\micro\meter} particles is generally 
realized by streaming near the surface of the particle. It is generated by a 
local acoustic pressure, that provides a continuously changing phase of the 
harmonic pressure amplitude from $0~\text{to}~2\pi$ in circumferential direction 
of the particle. Two acoustic orthogonal standing waves \cite{Wang1989}, or acoustic 
vortex beams \cite{Marston2016} can generate such conditions close to 
the surface of particles. 

A time-averaged streaming field is formed inside the VBL 
$\delta$ around the particle and initializes the particle rotation by surface 
tractions.  In the literature, it is known as acoustic viscous torque (VT) 
\cite{Busse1981, Rudnick1977, Wang1989}. Also, for non-spherical particles the acoustic 
radiation torque might contribute to rotations as shown by \citeauthor{Maidanik1958} 
\cite{Maidanik1958} in \citeyear{Maidanik1958}. In \citeyear{Schwarz2015}, 
\citeauthor{Schwarz2015} \cite{Schwarz2015,Dual2012} verified experimentally that 
micro fibers and arbitrarily shaped particle clumps can be rotated by acoustic 
radiation torques in micro-fluidic cavities by two phase modulated orthogonal 
standing waves. They showed in detail the phase dependency of particle clump 
patterns, by two orthogonal standing wave modes in micro-fluidic cavities.

The cause for the acoustic VT of spherical particles is streaming, a 
time-averaged effect that is related to the VBL. In 
\citeyear{Nyborg1958}, \citeauthor{Nyborg1958} \cite{Nyborg1958} derived the basic equations 
and assumptions for the analytical prediction of acoustic streaming effects. In 
\citeyear{Rudnick1977}, non-zero time-averaged torques on solid bodies due to 
acoustic streaming were first observed by \citeauthor{Rudnick1977} \cite{Rudnick1977}. 
They used an experimental set-up to rotate an axisymmetric cylinder and they 
measured its rim speed at different phase shifts of the excitation.  
\citeauthor{Busse1981} \cite{Busse1981} presented corresponding analytical calculations 
for the acoustic VT on rigid macro-cylinders, macro-circular plates and 
macro-spheres and they measured the acoustic torque on a fixed cylinder with a 
torsion fiber. In the 1980s, \citeauthor{Barmatz1983} \cite{Barmatz1983,Barmatz1989} and 
\citeauthor{Elleman1983} \cite{Elleman1983} reported in their patents an experimental setup 
that is able to rotate objects in all three dimensions by acoustic plane waves.  
In \citeyear{Wang1989}, \citeauthor{Wang1989} \cite{Wang1989} used the theory of 
\citeauthor{Nyborg1958} to calculate an analytical solution of the acoustic 
streaming field near a rigid micro-sphere due to two orthogonal standing waves.  
However, this model is only valid for $\delta \ll R$.

In addition, all theoretical investigations considered fixed objects and 
boundaries only.  \citeauthor{Lamprecht2013} \cite{Lamprecht2013} extended the theory of 
\citeauthor{Nyborg1958} to rotating objects with moving boundary conditions and 
keeping the restriction of $\delta \ll R$. Moreover, they compared their 
experiments to the results of \citeauthor{Wang1989} for rotating 
\si{\micro\meter}-spheres due to the acoustic VT\@.  \citeauthor{Lamprecht2013} were 
able to observe rotational rates up to \SI{20}{\hertz} (\SI{1200}{\rpm}) of 
polymethyl methacrylate (PMMA) particles with diameters between 
\SI{71}{\micro\meter} and \SI{446}{\micro\meter} by a high-speed camera and 
observed a dependency of the VT in phase, position and excitation amplitude.  
For particles smaller than \SI{71}{\micro\meter} a reliable rotation detection 
of the half-gold covered particles was not possible.

\citeauthor{Aubert2016} \cite{Aubert2016} applied two orthogonally 
orientated acoustic \emph{Scholte waves} \cite{Cegla2005} coupled into a 
micro-fluidic cavity to excite rotations of spherical \si{\micro\meter} 
particles. As mentioned before, acoustic vortex beams can also provide the 
conditions to apply an acoustic torque on spherical particles 
\cite{Marston2016}. This effect was experimentally demonstrated in 
several publications \cite{Hefner1999, Volke2008, 
Demore2012}. In the near vicinity of pressure nodes of two orthogonal 
standing waves, the conditions of the formed local acoustic field are comparable 
to those of an acoustic vortex beam \cite{Zhang2014}.

Recently, \citeauthor{Hahn2016} \cite{Hahn2016} concluded a numerical 
investigation on rotating \si{\micro\meter} particles by the acoustic VT\@. They 
found that the analytical models are a reliable tool to predict the rotational 
rate and direction for situations where the normalized VBL 
$\normBdLayer = \sfrac{\delta}{R} < \sfrac{1}{15}$, with $R$ being the particle 
radius and $\delta$ the VBL. They also investigated 
situations where $\normBdLayer > \sfrac{1}{15}$ and found that in this regime 
the density ratio $\tilde{\rho} = \sfrac{\rho_{s}}{\rho_{f}}$ between the 
particle density $\rho_{s}$ and the surrounding fluid density $\rho_{f}$ is not 
neglectable \cite{Hahn2016}. This ratio is not considered in the analytical 
prediction of the acoustic VT for a rigid sphere. In addition, 
\citeauthor{Hahn2016} \cite{Hahn2016} found that the rotational direction 
reverses at specific density ratios $\tilde{\rho}$ when the normalized viscous 
boundary layer $\normBdLayer \gg 1$. A reliable detection of the rotational rate 
$\Omega$ is limited by the frame rate of the high-speed camera.  
\citeauthor{Lamprecht2013} \cite{Lamprecht2013} measured for a \SI{71}{\micro\meter} 
silica particle a final rotational rate of \SI{20}{\hertz} (\SI{1200}{\rpm}).  
Higher rates could not be detected reliably, in particular for smaller 
particles.

This work utilizes a calibrated optical trap to overcome these limitations of 
high-speed imaging to detect rotational rates up to \SI{229}{\hertz} 
(\SI{13.8e3}{\rpm}) for \SI{2.06}{\micro\meter} silica particles. The optical 
trap offers the possibility to analyze the power spectrum of a particle motion 
while the particle rotates due to the acoustic VT inside the optical trapping 
potential.  This procedure is known from rotational flagella frequency 
investigations of bacteria \cite{Kirchner2014}, or motility efficiency of 
unicellular parasites \cite{Stellamanns2014}, where the detector gets the 
signal from the bacteria or parasite directly. However, the use of optical 
trapping in acoustofluidics put additional constraints on the design of the 
optical trap.  Initial experiments that were performed with an instrument that 
was designed to detect flagella rotations did not provide consistent results. A 
closed loop 3D positioning system is essential to position the particles with 
respect to the acoustic field. Also, the long term-stability of the instrument 
was optimized by operation in a thermally controlled environment, a careful 
selection of the laser and even the choice of the microscope immersion media.  
Details of these technical adaptations can be found in the methods section 
(\ref{sec:VT-opticalTrap}).

Furthermore, the optical trap has the advantage that its optical potential keeps 
the \si{\micro\meter} particles trapped against external acoustic radiation 
forces. This property offers the opportunity to experimentally investigate the 
position dependent acoustic VT, which was so far not found in the literature.  
In addition, the combination of the optical trap and the device design of 
\citeauthor{Schwarz2015} \cite{Schwarz2015} opens a novel possibility to levitate 
rotating particles and avoid any influences of near-boundary effects, or 
streaming due to vibrating micro air bubbles \cite{Lamprecht2013}.

It is demonstrated in the following that the peaks in the power spectrum of 
optically trapped and rotating particles will provide information about their 
rotational rate. If the rotation is multi-modal, multiple peaks are detected in 
the power spectrum. Additionally, trapped particles are moved inside the 
background pressure field to experimentally verify the position dependent change 
of rotational direction of the particles, as predicted in theory. In addition, 
the rotation measurement for cases where $\normBdLayer \approx 1$ supports the 
numerical results by \citeauthor{Hahn2016}.

In \cref{sec:VT-theory}, the underlying theory for the acoustic VT is summarized.  
\Cref{sec:VT-experimentalSetUp} explains the working principle of the optical trap 
and the used acoustic device. In \cref{sec:VT-experimentalProcedure}, the 
experimental protocol is defined, as well as the validation of the rotation 
measurement of the optical trap. In addition, the results for $\normBdLayer 
\approx 1$ are presented and discussed. \Cref{sec:VT-conclusion} discusses the 
findings of this work.
