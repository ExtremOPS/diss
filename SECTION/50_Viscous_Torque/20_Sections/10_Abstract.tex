Two orthogonal standing acoustic waves, generated by piezoelectric excitation, 
can form a two-dimensional pressure field in microfluidic devices. A phase 
difference of the excitation waves can be employed to rotate spherical 
\si{\micro\meter}-sized silica particles by a torque mediated through the 
viscous boundary $\delta$ around the particle.

The measurement of the rotational rate is, so far, limited to high-speed cameras 
and their frame rate, and gets increasingly difficult when the sphere gets 
smaller.  We report here a new method for measuring the rotational rate of 
\si{\micro\meter} sized spherical particles. We utilize an optical trap with 
high-speed position detection to overcome the frame rate limitation of wide 
field image recording. The power spectrum of an optically trapped, rotating 
particle reveals additional peaks corresponding to the rotational frequencies -- 
compared to a non-rotating particle. We validate our method at low rotational 
rates against high-speed video observation. 

To demonstrate the potential of this method we addressed a recent controversy 
about the rotation of particles with a relatively large viscous boundary layer 
$\delta$. We measured steady-state rotational rates up to \SI{229}{\hertz} 
(\SI{13.8e3}{\rpm}) for a particle with a radius $R \approx \delta$.  Recent 
numerical research suggests that in this regime the existing theoretical 
approach (valid for $R\gg\delta$) overpredicts the steady-state rotational rate 
by a factor of 10.  With our new method we also confirm the numerical results 
experimentally.
