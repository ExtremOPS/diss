\section{Introduction\label{sec:TC-introduction}}

In recent years, acoustofluidics has provided many powerful tools. Due to being 
contact-less, label-free, and biocompatible 
\cite{Antfolk2015,Abdulla2020,Zielke2020,Binkley2020,Cai2020}, acoustofluidic 
manipulation can be used in medical applications for cancer research
\cite{Antfolk2015,Abdulla2020,Zielke2020,Binkley2020}, Alzheimer research 
\cite{Cai2020}, targeted drug delivery \cite{Bose2015}, and for pumping medical 
fluids \cite{Wu2019}. In addition, there are biological 
\cite{Gerlt2020,Xie2019} and engineering applications (e.g., micro-pumping 
\cite{Wu2019,Huang2014,Lin2019,Ozcelik2021}).

Most of these applications utilize the acoustic radiation force (ARF) to 
manipulate objects on the micro-scale. The ARF is a second-order time-averaged 
effect that arises from the interaction of an acoustic field scattered at an 
object surface and a background acoustic field 
\cite{Doinikov1994Rigid,Hasegawa1969,Yosioka1955,Gorkov1962,Bruus2012}.
These objects can be solid particles, air bubbles, fluid droplets, biological 
samples, as long as their material properties (density $\rho$ and speed of 
sound $c$) are different from the surrounding medium. However, there coexists 
a fluid motion called acoustic streaming (AS) 
\cite{Nyborg1965,Kolb1956,Nyborg1953}. This motion can arise either from
viscous losses in the fluid (Eckhart type streaming \cite{Eckart1948}) or it 
can arise in the viscous boundary layer at a fluid to wall interface 
(Schlichting and Rayleigh streaming \cite{Riley1998,Schlichting1932}).


The theoretical derivations usually describe the steady-state of the AS field. 
A theoretical numerical study \cite{Muller2015} investigated the temporal build 
up of the ARF and AS field. In contrast to the ARF, the viscous drag force 
arising from AS is independent of the object material properties because it is 
a motion of the fluid. The AS direction coincides with the direction of the 
relative motion between fluid and particle.

For a spherical object of radius $R$, the drag force in laminar flow scales 
linearly with the object radius $\FAS \propto R$. In contrast to the $\FAS$, 
the ARF scales with the volume $\FARF \propto \R^{3}$ \cite{Bruus2012-10}.  
Based on the fluid and the object material properties, the $\FARF$ will 
dominate over the $\FAS$ if the radius $\R$ is greater than the critical radius 
$\R_{\text{crit}}$, where $\FAS = \FARF$ holds. The direction of $\FAS$ can be 
different from the $\FARF$. Therefore, the $\FAS$ is usually undesired.

The ARF and the AS occur not only in the bulk of the fluid, but also on sharp 
edges of a device \cite{Doinikov2020a,Doinikov2020b,Leibacher2015,Nama2016}. 
So-called micro-streaming around the surface of a spherical particle can even 
cause a sign inversion of the ARF if the viscous boundary layer $\delta$ is 
sufficiently large \cite{Baasch2019}. However, there are applications that take 
advantage of the AS \cite{Antfolk2014,Mao2017,Hao2020}: a complete overview of 
AS applications can be found in \cite{Wiklund2012a}.

In literature, it is well understood how long it takes until the acoustic 
field, and hence the ARF, needs to build up \cite{Muller2015} and how long the 
particle focusing takes \cite{Bruus2012-10}. However, it is still not fully 
clear how long it takes for the AS to build up, and what the definition for the 
analytical AS time constant is. In the acoustofluidics community, it is 
generally accepted that the build up for the AS field takes longer than the 
build up of the ARF. By using a pulsed actuation of the acoustic field and 
therefore exploiting this time offset, \citeauthor{Hoyos2013} prevented the 
build up of AS \cite{Hoyos2013,Castro2016}. They varied the number of periods 
for which the acoustic actuation is switched on and off, respectively. They 
experimentally showed that for a ratio of about 1 to 1 between 500 on- and 500 
off-periods the streaming velocity is less than 50\% of its steady-state 
magnitude while the ARF is not affected by that much.

\citeauthor{Muller2015} studied the build up of the acoustic energy density and 
streaming velocity with a numerical model \cite{Muller2015}. Their model 
consisted of a fluid cavity without any surrounding structure such as the 
cavity walls. They found numerically that indeed the ARF builds up 
significantly faster than the AS. However, the simulations with a pulsed 
actuation of different ratios of on- to off-periods did not prevent the build 
up of AS because its decay -- as the build up -- is slow compared to the ARF. 
The streaming builds up significantly slower during the on-periods, however, it 
does not decay to its initial value during the off-periods. Over time the 
influence of AS increases because the ARF alternates between some magnitude in 
the on-periods and zero in the off-periods. This implies, that the simulation 
of \citeauthor{Muller2015} could not explain the experimental results by 
\citeauthor{Hoyos2013}.

In this work, we experimentally measure the time until a \Dtwo~spherical 
silicon-dioxide (\SiO) particle moves with constant velocity when accelerated 
by the ARF and AS. Instead of using a camera, we utilize a data acquisition 
board (DAQ) with a sampling frequency of $f_{\text{s}} = \SI{1.25}{\MHz}$ to 
measure the relative particle trajectory as soon as the ultra-sound (US) is 
switched on. This high sampling frequency $f_{\text{s}}$ yields a high 
temporal resolution of $ \Dt = \SI{0.8}{\us}$. Considering the acoustic 
excitation frequency $\fex = \SI{4.015}{\MHz}$, we sample at least every fourth 
excitation period.

The optical tweezer (OT) for this study has already been successfully applied 
in the fields of acoustofluidics for stationary force measurements within a 
microfluidic chip \cite{Lamprecht2016,Lakaemper2015} as well as acoustic 
viscous torque investigations \cite{Lamprecht2021}. Here, we characterize in a 
first step the stationary force field in the bulk of the device to ensure, that 
we measure in a second step the time resolved build up of AS and the ARF 
separately and not their superposition. The separation is done by choosing a 
particle position within the acoustic field, where the $\FAS$ and $\FARF$ are 
orthogonal to each other. In order to measure in the second step solely the 
effects of the acoustic field on the particle and not the characteristics of 
the OT, we alter the usual trapping setup. The modification is that the 
particle is released from the OT before the acoustic excitation starts and 
retrapped after it.  Hence, during the measurement just gravity and the forces 
of the acoustic field act upon the particle. With our modified trapping setup, 
we are able to measure precisely the ARF and AS induced movement of a single 
particle in the bulk of the fluid.

Our manuscript is structured as follows: in \cref{sec:TC-theory} we derive and 
list all time constants in our system and we compute the traveled distances of 
a free floating particle in an acoustic field. Those influences need to be 
considered for our measurement protocol. In addition, we perform numerical AS 
simulations of our device to further understand the AS field; in 
\cref{sec:TC-experimental-setup} we explain our experimental setup and its 
modifications; in \cref{sec:TC-experimental-procedure} we show the results of the 
stationary force measurement, before explaining our time evolution measurement 
protocol and the data post-processing; and in \cref{sec:TC-results} we show and 
discuss the results of this study.



