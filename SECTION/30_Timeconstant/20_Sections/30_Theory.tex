\section{Preliminary Theoretical Considerations\label{sec:TC-theory}}
\subsection{Time Constants}

\begin{figure}[ht]
  \centering
  \def\svgwidth{\figWidth}
  \svginput{\relPath/10_Figures/LaTeX/Device.pdf_tex}
  \caption{Sketch of device. The light-gray area is the silicon channel walls, 
    the light-blue is the fluid cavity, and the dark-gray block the 
    piezoelectric element. The total length of the device is \SI{76}{\mm}. All 
dimensions are as listed in \cref{tab:TC-device-dimensions}.}\label{fig:device}
\end{figure}

\begin{table}
  \centering
  \begin{tabular}{l *{8}{x{10mm}}}
    \toprule
    \toprule
    Symbol & $W$ & $H$ & $W_{D}$ & $H_{T}$ & $H_{B}$ & $l$ & $w$ & $h$ \\
    Value [\si{\milli\meter}] & 3 & 0.1 & 26 & 0.13 & 0.9 & 20 & 4 & 0.5\\
    \bottomrule
    \bottomrule
  \end{tabular}
  \caption{Overview of device dimensions}\label{tab:TC-device-dimensions}
\end{table}


In our experiments there are multiple time constants that need to be considered. 
In the center of interest are the evolution of the ARF and the AS field. The 
acoustic energy $E_{\text{ac}}$, and hence the ARF, has the characteristic 
time constant \cite{Muller2015}
\begin{equation}
    \tarf = \frac{Q}{\omega_{0}} = \frac{Q}{2\pi\,\fex}
  \label{eq:TC-tau-arf}
\end{equation}
with $Q$ being the quality factor of the considered acoustic pressure mode and 
$\fex$ the excitation frequency. For the AS field, a theoretical expression for 
the time constant does not exist. Nevertheless, \citeauthor{Muller2015} report 
a \emph{momentum diffusion time}
\begin{equation}
  \tas = \frac{1}{2\nu}  L^{2} = \frac{\rhof}{2\muef} L^{2}
  \label{eq:TC-tau-nu}
\end{equation}
as the time constant for the AS field. Here, $L$ is half the radius of a 
streaming roll, $\nu=\sfrac{\muef}{\rhof}$ the kinematic viscosity, $\rhof$ the 
density, and $\muef$ the dynamic viscosity of the fluid. This formula is except 
for a factor of $\sfrac{1}{2}$ the same to $\tiner$ (equation 1.88) in 
\cite{Bruus2015}, which is the time a Poisseuille flow needs to fully stop in a 
circular tube of radius $L$ after the immediate removal of its driving 
pressure. To the best of the authors' knowledge, there is so far no better 
approximation for the time constant of the AS field.

When a particle is stably trapped, our OT has the properties of a linear 
mechanical spring \cite{Lamprecht2016}. This spring-like behavior of the OT has 
also a time constant until an acting force moves the trapped particle in its 
equilibrium position. The stiffness of the OT $k_{i}$ is linearly related to a 
characterization parameter of the OT called the cut-off frequency $f_{\text{c}} 
= \sfrac{k_{i}}{2\pi\,\gamma}$ with $\gamma$ being Stokes' drag coefficient 
\cite{Lamprecht2016,Lamprecht2017}. This frequency is the \SI{-3}{\dB} point 
in the Brownian motion power spectrum (more detail in 
\cite{Lakaemper2015,Lamprecht2016}). We can therefore compute the time 
constant of the OT as
\begin{equation}
  \tOT = \frac{1}{2\pi\,f_{c}}.
  \label{eq:TC-tau-OT}
\end{equation}
Lastly, our DAQ system has the time constant $\tqpd$ which describes how fast we 
can measure a sudden change in laser intensity of the OT. This parameter is 
found by changing the laser intensity at a precise point in time and then 
extracting the temporal difference until the DAQ measures it. 

With the parameters of our experiment (see \cref{tab:TC-parameters}) the mentioned 
time constants are as listed in \cref{tab:TC-time-constants}. Hence, with the 
usual trapping mode of the OT, we cannot measure the ARF and AS because $\tOT 
\approx \tas$ and $\tOT\gg\tarf$. In the limit of zero laser power there is no 
trapping potential and hence $\tarf$ and $\tas$ can be measured.

\begin{table}
  \centering
  \begin{tabular}{lccr}
    \toprule
    \toprule
    {\bfseries Parameter} & {\bfseries Symbol} & {\bfseries Value} & {\bfseries 
    Unit}\\
    \midrule
    \textbf{Fluid} & & \\
    Density & $\rhof$ & 1000 & \si{\kg\per\cubic\meter} \\
    Speed of sound & $\cfl$ & 1500 & \si{\m\per\s} \\
    Compressibility & $\kappa_{\text{f}}$ & 4.4E-10 & \si{\per\pascal} \\
    Dynamic viscosity & $\muef$ & 890 & \si{\micro\pascal\second} \\
    Kinematic viscosity & $\nu_{\text{f}}=\sfrac{\muef}{\rhof}$ & 0.890 & 
    \si{\square\mm\per\second} \\
    \midrule
    \textbf{Particle} & & \\
    Density & $\rhop$ & 1850 & \si{\kg\per\cubic\meter} \\
    Radius & $\Rtwo$ & 1.03 & \si{\um}\\
    Radius & $\Rfour$ & 2.195 & \si{\um}\\
    Compressibility & $\kappa_{\text{p}}$ & 1.6E-11 & \si{\per\pascal} \\
    \midrule
    Device quality factor & $Q$ & 36 & - \\
    Corner frequency of OT & $f_{\text{c}}$ & $\approx 100$ & \si{\hertz} \\
    \midrule
    Excitation frequency & $\fex$ & 4.015 & \si{\MHz} \\
    \bottomrule
    \bottomrule
    
  \end{tabular}
  \caption{Symbols and physical properties of the fluid, the particle, and the 
    experimental setup. The quality factor $Q$ is extracted from an admittance 
    measurement of the device filled with water and fixed in the microscope as 
for all measurements. The magnitude of $f_{\text{c}}$ is the usual value in 
stationary force measurements for the OT.}\label{tab:TC-parameters}
\end{table}

\begin{table}
  \centering
  \begin{tabular}{l S[table-format=7.5]S[table-format=7.1]}
    \toprule
    \toprule
    {\bfseries Symbol} & {\bfseries $\tau_{i}$ [\si{\ms}]} & {\bfseries 
    $\sfrac{\tau_{i}}{t_{0}}$ [-]}  \\
    \midrule
    $\tOT$ & 1.59 & 6383.9 \\
    $\tqpd$ & 0.050 & 200.8 \\
    $\tarf$ & 0.0014 & 5.6 \\
    $\tas\vert_{L=\sfrac{H}{2}}$ & 1.44 & 5781.6\\
    $\tas\vert_{L=\sfrac{H}{4}}$ & 0.35 & 1405.3\\
    $\tdrag$ & 0.00049 & 2.0 \\
    \midrule
    $\Dt_{\text{DAQ}}$ & 0.0008 & 3.2 \\
    \bottomrule
    \bottomrule
    
  \end{tabular}
  \caption{Overview of time constants $\tau_{\text{i}}$ for the system. The 
    values are obtained by using the values from \cref{tab:TC-parameters} and 
    \cref{eq:TC-tau-nu,eq:tau-arf,eq:tau-OT,eq:tau-drag}. $\tqpd$ is measured, 
    $\Dt_{\text{DAQ}} = \sfrac{1}{f_{\text{s}}}$, and 
$t_{0}=\sfrac{1}{\fex}$.}\label{tab:TC-time-constants}
\end{table}

\subsection{Free Particle Motion}

If there is no trapping laser power, the spherical particle with mass $m$ will 
move in the fluid due to some acting force $F$; this force can be gravity, the 
ARF, the drag force from AS, or a combination of them. The one dimensional 
dynamic equation for the particle displacement $q$ far away from any walls is 
the same for the three spatial directions $\ex, \ey$ and $\ez$.
\begin{equation}
  \ddot{q} = - \frac{F}{m} - \frac{\gamma}{m}\,\dot{q} =
  - \tilde{F} - \frac{1}{\tdrag}\,\dot{q}
  \label{eq:TC-free-fall}
\end{equation}
with $F$ being a force acting along the direction of $q$ and
\begin{equation}
  \tdrag = \frac{m}{\gamma} = \frac{V\,\rhop}{6\pi\,\R\,\muef}
  = \frac{2}{9}\,\R^{2}\,\frac{\rhop}{\muef}.
  \label{eq:TC-tau-drag}
\end{equation}
Here, $\R$ is the particle radius, $V$ the particle volume, and $\rhop$ the 
particle density. In microfluidics the viscous effects dominate over the 
inertial effects \cite{Bruus2015}. Therefore, we neglect $\ddot{q}$ for further 
calculations. Solving the modified first order ordinary differential equation
\begin{equation}
    \dot{q}= - \tdrag\,\tilde{F} = -\frac{F}{m}\,\tdrag
  \label{eq:TC-mod-free-fall}
\end{equation}
with the initial condition $q\vert_{t=0} = 0$, gives the linear relation $q(t) 
= - \tdrag\,\tilde{F}\,t$ with the integration constant being zero.

As already mentioned, we measure while there is no trapping potential of the 
OT. Therefore, on the particle act only gravity and the forces from the 
acoustic field. In our experiment we have for $F$ along $\ey$ a spatially 
varying force with a maximal value of \SI{0.5}{\pico\newton} and along $\ez$ 
the buoyancy corrected gravitational force $\tilde{m}g=V\,\left( \rhop-\rhof 
\right)\,g$ with a magnitude of \SI{38.2}{\femto\newton} and an acoustic force 
with a maximal value of \SI{0.25}{\pico\newton} (for acoustic force magnitudes 
see \cref{fig:2um_map}). As we will explain later (see 
\cref{sec:TC-experimental-procedure}), we have \SI{25}{\ms} without any laser 
power where the particle will solely move due to gravity and then \SI{30}{\ms} 
of US excitation where also the acoustic forces are acting.

Hence, a spherical \SiO~$\Rtwo=\SI{1.03}{\um}$~particle will have moved 
$0\Rtwo$ along $\ey$ and $0.05\Rtwo$ along $\ez$ after $\SI{25}{\ms}$ with just 
gravity acting. And, after $\SI{55}{\ms}$, when there are additionally 
constant acoustic forces, the particle will have traveled distances of 
$0.84\Rtwo$ and $0.54\Rtwo$ along $\ey$ and $\ez$, respectively. For the 
latter, $0.54\Rtwo$ is the sum of $0.12\Rtwo$ due to gravity and $0.42\Rtwo$ 
due the force from the acoustic field.

\subsection{Numerical Streaming Simulations}

To understand the influences and implications of the AS on our measurements, we 
simulate with {\ttfamily COMSOL Multiphysics 5.6} (COMSOL Inc., Stockholm, 
Sweden) 2 two-dimensional structures that relate to the experimental device; 
one with just the fluid cavity as the baseline model (cavity-only model); and 
the other with added structure around the cavity to reflect our real device 
(whole-device model). See \cite{supplemental} for both models as one {\ttfamily 
.mph} file. For both we follow the work of \citeauthor{Muller2015} 
\cite{Muller2015} in terms of the fluid mesh size.

\afterpage{
  \vspace*{\fill}
\begin{figure}[H]
  \centering
  \begin{subfigure}{\figWidthDouble}
    \centering
      \caption{Streaming simulation for cavity-only model at $f_{\text{max}} = 
      \SI{3.987}{\MHz}$.}\label{subfig:JC-streamingrolls}
    \def\svgwidth{\figWidthDouble}
    \textsf{\tiny % that is for the added unit @ the axis
    % \input{10_Figures/LaTeX/JC-streaming-rolls.pdf_tex}
    \svginput{\relPath/10_Figures/LaTeX/JC-streaming-rolls_paraview.pdf_tex}}
    \end{subfigure}\\%k
  \begin{subfigure}{\figWidthDouble}
    \centering
    \caption{Streaming simulation for whole-device model at $f_{\text{max}} = 
    \SI{3.745}{\MHz}$. The gray area is the structure around the 
  cavity.} \label{subfig:FD-streamingrolls}
    \def\svgwidth{\figWidthDouble}
    \textsf{\tiny % that is for the added unit @ the axis
    \svginput{\relPath/10_Figures/LaTeX/FD-streaming-rolls_paraview.pdf_tex}}
    % \input{10_Figures/LaTeX/FD-streaming-rolls.pdf_tex}
  \end{subfigure}
  \caption{Results for streaming simulations of a cavity-only model (a) and a 
    model with surrounding structure (b). The colormap shows the total acoustic 
    pressure and the white arrows the streaming flow. For both simulations 
    $f_{\text{max}}$ is the frequency of maximal acoustic energy density 
    $E_{\text{ac}}$ for a pressure mode with 16 nodal lines. The pressure mode 
  for both simulations is the same besides the phase shift of 
$\pi$.}\label{fig:comsol-streaming}
\end{figure}
\vspace*{\fill}
  \clearpage
}

We model a two-dimensional $yz$ slice of the whole device as seen in 
\cref{fig:device} without the piezoelectric transducer (PZT) and its glue 
layer. Therefore this whole-device model consists of two glass, two silicon, 
and one water domain. We utilize the Solid Mechanics \comsol{solid} interface 
for the silicon and glass. For the cavity we employ the Creeping Flow 
\comsol{spf} interface with the spatial variation of the Reynolds Stress as 
source and the Stokes drift as the boundary condition. Also in the cavity, we 
use the Thermoviscous Acoustics \comsol{ta} interface. Lastly, we couple the 
Solid Mechanics with the Thermoviscous Acoustics via Thermoviscous 
Acoustic-structure Boundary \comsol{tsb}. The cavity-only model solely needs 
the Creeping Flow and the Thermoviscous Acoustics interface without 
multiphysics coupling.

Besides the added structure around the cavity, the main difference between the 
two models is the location of the excitation. The whole-device model has as 
boundary condition a prescribed displacement along $\ez$ of 
$z_{\text{BD}}=\SI{0.1}{\nm}$, where the PZT is glued onto the device. The 
cavity-only model, however, has a prescribed constant velocity of its left 
cavity wall along $\ey$ of $\dot{y}_{\text{BD}} = \SI{25}{\mm\per\second}$. 
This magnitude corresponds to the mean wall velocity of the whole-device model, 
where the excitation is at the PZT. With those two boundary conditions the 
acoustic pressure is \SI{310}{\kPa} for the whole-model and \SI{550}{\kPa} for 
the cavity-only model at their respective 16 nodal pressure line frequency with 
maximal acoustic energy density $E_{\text{ac}}$. The discrepancy in pressure 
amplitude comes from the applied boundary conditions of respective models.

The respective frequencies of maximal $E_{\text{ac}}$ 
(\SI{3.9876}{\MHz} and \SI{3.7450}{\MHz}) inside the cavity while having a 16 
nodal line mode were determined with a frequency sweep. This is the same mode 
we have in our experiment as well. For the streaming simulation, we employ a 
stationary study of the Creeping Flow interface that uses the results from the 
frequency domain study in its source term and as the boundary conditions.

\Cref{fig:comsol-streaming} shows the results for the pressure and streaming 
fields of both models. The magnifications correspond to the area where we 
perform our measurements in the experiment. One can see that the simulated 
pressure fields are qualitatively the same, however, the streaming fields 
differ to a great extent. The cavity-only simulation depicts spatially 
repetitive streaming rolls over the whole fluid domain. In the bulk of the 
fluid is Rayleigh streaming. However, near-boundary Schlichting streaming is 
not visible because the viscous boundary layer is relatively small. In contrast 
to that, the simulation for the whole-device model has a non-spatially 
repetitive streaming field. There are regions where its similar to the 
cavity-only model streaming field. But, the streaming pattern is non-repetitive 
and exhibits strong local differences. As a consequence, care must be taken to 
chose a measuring point where the $\FAS$ and $\FARF$ are orthogonal to each 
other to ensure no superposition of forces.

Although the clamping of the microscope setup and the oil immersion layer for 
the lens are excluded in the model with structure, one can see that the 
streaming field is a local, non-periodic effect, whereas the pressure field is 
spatially periodic. We expect that these tendencies remain the same when the 
clamping, the immersion layer, and the PZT are added.
