The combination of a bulk acoustic wave device and an optical trap allows 
studying the build up time of the respective acoustic forces. In particular, we 
are interested in the time it takes to build up the acoustic radiation force 
and acoustic streaming. For that, we measure the trajectory of a spherical 
particle in an acoustic field over time. The shape of the trajectory is 
determined by the acoustic radiation force and by acoustic streaming; both 
acting on different time scales. For that, we utilize the high temporal 
resolution ($\Dt = \SI{0.8}{\us}$) of an optical trapping setup. With our 
experimental parameters the acoustic radiation force on the particle and the 
acoustic streaming field theoretically have characteristic build up times of 
\SI{1.4}{\us} and \SI{1.44}{\ms}, respectively. By choosing a resonance mode 
and a measurement position where the acoustic radiation force and acoustic 
streaming induced viscous drag force act in orthogonal directions, we can 
measure the evolution of these effects separately. Our results show, that the 
particle is accelerated nearly instantaneously by the acoustic radiation force 
to a constant velocity, whereas the acceleration phase to a constant velocity 
by the acoustic streaming field takes significantly longer. We find that the 
acceleration to a constant velocity induced by streaming takes in average about 
17'500 excitation periods ($\approx \SI{4.4}{\ms}$) longer to develop than the 
one induced by the acoustic radiation force. This duration is about 4 times 
larger than the so-called momentum diffusion time which is used to estimate the 
streaming build up. In addition, this rather large difference in time can 
explain why a pulsed acoustic excitation can indeed prevent acoustic streaming 
as it has been shown in some previous experiments.
