\section{Conclusion\label{sec:TC-conclusion}}

In this work we presented the measurement of the temporal evolution of the AS 
field and the ARF in a BAW device utilizing an OT. We slightly modified our 
validated optical trapping setup \cite{Lamprecht2016,Lamprecht2021} to 
accommodate the requirements of this experiment. With a temporal resolution of 
$\Dt=\SI{0.8}{\us}$ we could measure at least every fourth time period of 
excitation. We validated our measurement protocol against the stationary force 
field.

We monitored the trajectory of a \Dtwo~\SiO~particle as soon as the US 
excitation of the device started. We selected measurement positions in a 
standing pressure wave mode where ARF dominates in one direction and AS 
orthogonal to it. In addition, we chose the spatial location within the mode 
to maximize the amplitude of both effects. Our measurements show, that the ARF 
is established almost immediately after the US is switched on; whereas the AS 
takes in average 17'500 excitation periods (\SI{4.4}{\ms}) longer to evolve. 
This time is about four times larger than the theoretical approximation with 
the momentum diffusion time.

These results show that the build up of AS takes significantly longer than the 
build up of the ARF. This temporal difference can explain why a pulsed acoustic 
excitation can prevent streaming as it has been experimentally shown by 
\citeauthor{Hoyos2013} \cite{Hoyos2013,Castro2016}. In addition, the results of 
the streaming simulations of a cavity-only model and a whole-device model show 
that simplified models are enough for simulations of the pressure fields, 
however they cannot reflect \emph{real} streaming patterns. This insight might 
also explain why \citeauthor{Muller2015} could not reproduce the suppression of 
AS with a pulsed excitation in their cavity-only model.
