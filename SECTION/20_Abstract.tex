\chapter*{Abstract} \markboth{Abstract}{Abstract}
\addcontentsline{toc}{chapter}{Abstract}
%Abstract structure: background – activity/purpose – methods – results – 
%conclusion (BAMRC)

A typical channel within a micro-scale acoustofluidic (MSAF) device has a small 
cross-section where the height is usually less than \SI{200}{\um} and the width 
less than \SI{5}{\mm}; the length can be up to several \si{\cm}, however, often 
is not of big interest. Direct measurement of, e.g., the pressure produced by 
the acoustic excitation is impossible due to the smallness of the region of 
interest. Additionally, the acoustic driving frequencies are generally above 
\SI{100}{\kilo\hertz} such that the time resolution of measurements must be 
even at least twice as high if one is also interested in the transient 
behaviour and build up of, e.g., the acoustic pressure field and not only the 
steady-state.

At the moment, the most common and straightforward way to approximate the 
acoustic pressure within the channel is to optically measure the velocity of 
several objects of known size and material properties and then calculate back 
which pressure would have led to this velocity. The validity and correctness of 
the pressure approximation depends on several uncertainties. Besides the object 
dimensions and the material parameters of the object and the fluid, the biggest 
uncertainty is the validity of the underlying theory that is used for the 
calculation. There exist many MSAF models for the calculation of the acoustic 
forces which differ mainly in the assumptions regarding the physical model for 
the fluid and the immersed object. Each theory has its parameter space where it 
is superior to the others because it includes, e.g., the effect of 
visco-elasticity of the fluid.

Here, an optical trapping (OT) apparatus is utilized to investigate two 
phenomena where controversies exist in the MSAF community: 1) the transient 
build up of the acoustic radiation force (ARF) and the drag force from acoustic 
streaming (AS) for a continuous and pulsed acoustic excitation; 2) the 
quantification of the steady-state rotational velocity of a spherical particle 
driven by the acoustic viscous torque where the viscous boundary layer (VBL) 
thickness is comparable to the particle radius itself.

So far, OTs have mainly been used as force sensors on single particles within 
MSAF devices. For the measurements of both phenomena we take advantage of the 
fine spatial and temporal resolution that the OT offers, as well as the OT 
property that single particle measurements are possible.

In order to measure the build up of the ARF and AS, an acoustic excitation 
frequency and measurement location within the standing pressure wave was used 
where the two forces were orthogonal to each other. The orthogonality as well 
as the division into ARF and drag force from AS was measured and validated by 
force measurements with the OT throughout the fluid channel with differently 
sized particles.

The results of a continuous excitation showed that the ARF starts to build up 
almost instantaneously after the acoustic excitation was switched on, whereas 
the AS takes significantly longer. Interestingly, the fast ARF build up was 
expected from theoretical considerations, but the slow AS build up was 
underestimated by a factor of about 4. The pulsed excitation experiments 
revealed that depending on the specific pulse parameters the build up of AS can 
be suppressed substantially while the ARF is not affected as much as AS. 
Therefore, smaller particles can still be mainly manipulated by the ARF because 
the relative importance of AS decreases for a pulsed excitation. Our 
measurements strengthen experimental findings for a pulsed excitation that 
could not yet be explained theoretically.


For the steady-state rotational speed measurement, a high viscosity mixture of 
water with glycerol (7 to 3) was created such that the formed VBL around the 
particle was about the same as the particle radius. The phase difference 
between two acoustic excitation sources spatially orthogonal to each other led 
to a time-averaged acoustic streaming field in the VBL of the particle along 
its circumference. This streaming field creates a driving viscous torque that 
causes a rotation with the rotational velocity at which the driving viscous 
torque equals the counteracting viscous drag torque.

The theoretical formula overestimates the steady-state rotational speed for the 
experimental parameters by more than one order of magnitude. This was expected 
because, up to now, there are no theories that are valid for the regime where 
the radius is the same size as the VBL. However, a numerical study investigated 
exactly this regime and proposed a calculation for the final rotational 
velocity including the effects of the VBL. The rotational velocities measured 
with the OT confirmed two points: 1) the expected invalidity of the simplified 
theory in the regime of high viscosity and, hence, the necessity of its 
inclusion in the calculations; 2) the correctness of the numerical results.

