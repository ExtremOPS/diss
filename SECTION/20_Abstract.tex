\chapter*{Abstract} \markboth{Abstract}{Abstract}
\addcontentsline{toc}{chapter}{Abstract}
%Abstract structure: background – activity/purpose – methods – results – 
%conclusion (BAMRC)

A typical cavity within a micro-scale Acoustofluidic (MSAF) device has a small 
cross-section. The height of the fluid cavity is usually less than 
\SI{200}{\um} and the width less than \SI{10}{\mm}; the length can be up to 
several \si{\cm}, however, often not of big interest. Direct measurement of, 
e.g., the pressure produced by the acoustic excitation is impossible due to the 
smallness of the region of interest. Additionally, the acoustic driving 
frequencies are often above \SI{100}{\kilo\hertz} such that the time resolution 
of measurements must be even higher if one is also interested in the transient 
behaviour and build up of, e.g., the acoustic pressure field and not only the 
steady-state.

At the moment, the most common and easiest way to approximate the acoustic 
pressure within the cavity is to measure optically the velocity of several 
objects and then calculate back which pressure would have led to this velocity. 
The validity and correctness of the pressure approximation depends on several 
uncertainties. Besides the object dimensions and the object and fluid material 
parameters, the biggest uncertainty is the validity of the underlying theory 
that is used for the back calculation. There exist many theories for MSAF 
applications which differ mainly in the assumptions regarding the physical 
model for the fluid and the immersed object. Each theory has its parameter 
regime where it is superior to the others because it includes, e.g., the effect 
of visco-elasticity of the fluid.

Here, an optical trapping (OT) apparatus is utilized to investigate two 
phenomena where controversies exist for both in the MSAF community: 1) the 
transient build up of the acoustic radiation force (ARF) and the drag force 
from acoustic streaming (AS) for a continuous and pulsed acoustic excitation; 
2) the quantification of the steady-state rotational velocity of a spherical 
particle driven by the acoustic viscous torque where the viscous boundary layer 
(VBL) is as big as the particle radius itself.

So far, OTs have mainly been used as force sensors on single particles within 
MSAF devices. For the measurements of both phenomena we take advantage of the 
fine spatial and temporal resolution that the OT offers, as well as the 
property of the OT that measurements on single particle are possible.

In order to measure the build up of the ARF and AS, an acoustic excitation 
frequency and measurement location within the standing pressure wave field was 
used where the two forces were orthogonal to each other. The orthogonality as 
well as the division into ARF and drag force from AS was measured and validated 
by force measurements with the OT throughout the fluid cavity with different 
sized particles.

The results of a continuous excitation showed that the ARF starts to build up 
almost instantaneously after the acoustic were switched on, whereas the AS 
takes significantly longer. Interestingly, the fast ARF build up was expected 
from theoretical considerations, but the slow AS build up was underestimated by 
a factor of about 4. The pulsed excitation experiments revealed that depending 
on the specific pulse parameters the build up of AS can be suppressed 
substantially while the ARF is not affected by the same amount as AS. These 
results strengthen experimental findings that could not yet be explained by a 
MSAF theory.

For the steady-state rotational speed measurement, a high viscosity mixture of 
water with glycerol (7 to 3) was created such that the formed VBL around the 
particle was about the same as the particle radius. The phase difference 
between two acoustic excitation sources spatially orthogonal to each other 
created a continuous phase change of the pressure amplitude around the particle 
circumference that initiates a rotation up to the rotational velocity where the 
driving viscous torque equals the counteracting viscous drag torque.

The theoretical formula overestimates the steady-state rotational speed for the 
experimental parameters by more than one order of magnitude. This was expected 
because, up to now, there does not exist a theory that is valid for the regime 
where the radius is the same size as the VBL. However, a numerical study 
investigated exactly this regime and proposed a calculation including the 
effects of the VBL. The rotational velocities measured with the OT proved both 
satements: 1) the invalidity of the theory in the regime of high viscosity and, 
hence, the necessity of its inclusion; 2) the correctness of the numerical 
results.

