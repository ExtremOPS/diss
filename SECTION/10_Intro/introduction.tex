\chapter{Introduction\label{ch:intro}}
\section{Acoustofluidics}

During the second half of this doctoral thesis the global pandemic of Corona 
was ongoing. The symptoms and the course of the disease depends on the variant 
of the virus. But all in all, each variant up to now was not very deadly. 
Nevertheless, the pandemic could not be stopped within 2 years although 
researches managed to engineer working vaccinations against 
it~\cite{Polack2020,Mahase2020,Voysey2021}. One of the reasons for that is, 
that Covid-19 has a rather long incubation period where people often do not 
feel sick, yet, but can already transmit the diseases. Especially in the early 
stages the key for preventing further spreading was testing of people.

Besides finding an useful and working test process another problem was the 
decentralized testing facilities. Up to now the most accurate test is a 
polymerase chain reaction (PCR) test that is conducted in biomedical lab. These 
labs are already rare in the northern hemisphere but almost non-existing in the 
developing countries. With the invention of rapid antigen tests that could be 
performed at home and by everybody without a special training the testing was 
decentralized and multiplied. Having a rapid test that is 
precise~\cite{Albert2021} and that has the results within about 
\SI{15}{\minute} helps further spreading of the disease because it is much 
faster than the \SI{24}{\hour} to \SI{48}{\hour} that a PCR test takes and more 
convenient because it can be taken everywhere.

Corona is one good example that one of the keys for the prevention of local and 
global spreading of any disease is fast and easy detection of ill people. 
Micro-scale acoustofluidics (MSAF) offers the possibility to miniaturize whole 
labs onto a single chip because it posses the ability to manipulate in a 
controlled way objects that are immersed in a fluid. The two main forces 
causing the manipulation is the so called acoustic radiation force (ARF) and 
the drag force from acoustic streaming (AS). The manipulation is -- amongst 
others -- lable-free and biocompatible and hence interesting for biological and 
medical applications. This manipulation can also be of such a kind that objects 
are stably trapped against background fluid flows. Hence, this effect is know 
with the phrase \emph{acoustic traps} (ATs).

The mini-labs are often referred to as \emph{Lab on a Chip} and are not bounded 
to a central testing facility because the final chip usually does not need the 
working environment of biological or medical lab and can be therefore be used 
everywhere. The target object size for MSAF is in the \si{\um} range; therefore 
MSAF is not yet suitable for viral diseases like Covid-19. However, recently 
\cname{Gu2020} were able to also handle objects in the \si{\nano\meter} range, 
\cname{Gerlt2022} successfully manipulated polystyrene and metal particles with 
a radius of \SI{500}{\nm} in two dimensions in a glass capillary, and 
\cname{Evander2015} captured platelet-derived microparticles of \si{\nm} size 
from human plasma samples.

MSAF is a rather new technology that is also depending on the advances in 
microfabrication. Up to now, the fabrication process of MSAF devices is mainly 
done in clean rooms offering the possibility to fabricate very precisely 
\si{\um}- but also \si{\nm}-structures. However, clean rooms are very expensive 
and not available everywhere. In addition, the needed know-how for the 
production is high. More recently, there emerges the trend to investigate also 
other fabrication techniques like using standard glass capillary as fluid 
cavities~\cite{Wiklund2001,Hammarstrom2010,Hammarstrom2012,Mishra2014,Gralinski2014,Gerlt2022} 
producing MSAF devices with rapid prototyping~\cite{Adams2012}, using micro 
machined fluid channels in aluminum with Polydimethylsiloxane (PDMS) 
coverslips~\cite{Gautam2018}, building a glass-PDMS-glass chip with standard 
glass coverslips~\cite{Xu2019}, directly machined fluid channels into Poly 
Methyl Methacrylate (PMMA)~\cite{Harris2012}, building microfluidic devices out 
of paper~\cite{Martinez2010}, bonding two parts out of thermoplastics together 
to create closed microfluidic structures~\cite{Mueller2013}, or, last but not 
least, using devies made out of PMMA~\cite{Gonzalez2015,Yang2017}.

At the moment MSAF is at a point in time where the citations with the keyword 
\emph{acoustocluidics} increased by one order of magnitude over the past 10 
years (2012-2021)~\cite{Novotny2021} and where the research turns from pure 
observations of MSAF phenomena like particle 
manipulation~\cite{Wiklund2012a,Laurell2007,Collins2016,Gedge2012,Ding2012b,Cetin2016b,Aubert2016,Novotny2021}
, acoustic streaming~\cite{Hoyos2013,Castro2016,Lei2016}, or the effects on the 
AS and the ARF around sharp 
edges~\cite{Leibacher2015,Chen2021,Doinikov2020,Doinikov2020b} to biological 
and medical applications.

MSAF is already used (1) as a handling toolbox for cancer 
research~\cite{Antfolk2015,Wu2021,Wang2020,Nguyen2021}, (2) for the analysis of 
single cells suspended in \si{\nano\liter} droplets~\cite{Gerlt2020a}, (3) for 
the approximation of the compressibility of \emph{Caenorhabditis 
elegans}~\cite{Baasch2018} which is a model organism frequently used for 
biomedical studies, (4) for the focusing of metal particles to possibly 
increase the resolution of three dimensional metal printing~\cite{Gerlt2022}, 
(5) as mixer of fluids utilizing 
AS~\cite{Patel2014,Ozcelik2014,Bachman2020,Zhang2021}, (6) for single cell 
analysis of HeLa cancer cells by controlled acoustic induced 
rotation~\cite{Laeubli2021}, and last but not least (7) as a tool for 
continuous exchange of medium for bacteria~\cite{Gerlt2021}.

Often the limiting factor of MSAF applications is the size of the object of 
interest. One reason is that the drag force from AS scales linearly with the 
object area subjected to the streaming flow and the ARF scales linearly with 
the object volume. For spherical objects the critical radius where those two 
force are equal in magnitude is about \SI{1}{\um} for water-like fluids. Bigger 
objects are dominated by the ARF and smaller by the drag force from AS. In most 
cases AS is undesired because the shape and the occurrence of the flow is hard 
to control. Therefore, a good understanding of the pressure and the fluid flow 
field is key for further advances in MSAF applications.

\section{Optical Trapping for Acoustics}

With the Noble Prize winning technology of optical trapping by Arthur 
Ashkin~\cite{Ashkin1978,Ashkin1987,Ashkin2002,Ashkin1986,Ashkin1992,Ashkin1997} 
there exist the possibility to combine the acoustic and optical trap (OT) 
together to gain further understanding of the micromeachnics inside a 
acoustocluidic device.

\cyear{Thalhammer2011} were the first to combine a \SI{1064}{\nm} OT with an 
AT.  They designed their OT such that it is a low numerical aperture (NA) OT to 
have a wider field of view. Usually, OTs need a high NA to create enough 
trapping force magnitude. This was possible because by the usage of a mirror 
they could mimic a dual-beam OT with a single laser beam. They used their setup 
in two ways: first, they combined the two kinds of traps to trap big particles 
against gravity (AT) and precisely manipulate them (OT); second, they used the 
OT as a force sensor for the AT to measure the acoustic forces acting the 
non-spherical objects.

In a similar way, \cname{Bassindale2014}~(\citeyear{Bassindale2014}) used a 
\SI{1064}{\nm} single beam high NA OT as force sensor on \SI{5}{\um} silica 
particles.  Furthermore, they split the measured force on the particle into a 
contribution from AS and from the ARF. They validated their measurements by 
comparing it to the forces retrieved from particle image velocimetry.

\cyear{Fury2014} utilized a \SI{1064}{\nm} single beam high NA OT together with 
an AT to manipulate microbubbles with a size of \SI{13}{\um}. Their aim was to 
utilize the wide trapping range of the AT, as well as the fact that the AT can 
trap multiple particle simultaneously, together with the high selectivity and 
fine spatial resolution of the OT. Additionally, they measured the forces 
subjected to the microbubble for varying parameters of the AT and OT (acoustic 
frequency, excitation voltage, laser power, beam diameter).

\cyear{Lakaemper2015} build a \SI{980}{\nm} single beam high NA OT to 
characterize in two dimensions the time-averaged forces on \si{\um} particles 
in a standing pressure wave field. They measured the acoustic forces at 
different positions in the standing pressure wave on a single particle while 
sweeping the frequency. They compared their results with numerical simulations 
and found good agreement. With the measured force amplitudes they were also 
able to approximate the acoustic pressure inside the acoustofluidic device.

\cyear{Lamprecht2016} extended the work of~\cite{Lakaemper2015} to measure with 
a \SI{785}{\nm} single beam high NA OT the acoustic forces in all three spatial 
dimensions. Their main objective was to visualize the pressure field throughout 
the fluid volume for different frequencies. In addition, they measured with the 
same experimental settings the acoustic forces with two different sizes of 
particle in order to split the total measured forces into linear contributions 
from AS and cubic contributions due to the ARF. They also used two devices with 
different channel widths (\SI{2}{\mm} and \SI{4}{\mm}) to show the scaling of 
the pressure wavelength with the channel dimension.

\cyear{Thalhammer2016} use a \SI{1064}{\nm} single beam high NA OT with a 
layered AT to measure primarily the axial forces on a particle. Furthermore, 
they investigate the influence of the oil immersion layer of the lens on to the 
acoustic resonances and define design guides for limiting the perturbations on 
the acoustic field of the immersion oil. Their force measurement technique is 
independent of the particle shape allowing force measurements of arbitrary 
shaped small objects. Additionally, their driving acoustic transducer is 
transparent for the laser wavelength such that it can be placed in close 
proximity of the fluid chamber.


\section{Advances in Streaming Suppression}

As discussed before, AS is often undesired because it can counteract the ARF.  
Besides the fluid mixer driven by AS, there are also other applications for 
AS~\cite{Wiklund2012a}. Nevertheless, the suppression of AS is still of high 
interest for MSAF.

\cyear{Hoyos2013} demonstrated that a pulsed ultrasound excitation can reduce 
and suppress AS while the ARF is still present. They studied in two MSAF 
devices the effects of different pulse settings to the steady state streaming 
velocity and the ARF. Both geometries yield similar results showing that is not 
geometry dependent. To visualize both effects simultaneously they use 
\SI{800}{\nm} and \SI{15}{\um} latex particles. The smaller ones are dominated 
by the drag force from AS and the larger one by the ARF. In addition, they 
could show that the secondary forces, like particle-particle interaction, are 
still present while pulsing allowing to also trap the small particles with the 
larger ones as seed particles.

\cyear{Castro2016}

\section{Research Question\label{sec:I-researchquestion}}

\section{Outline of Thesis\label{sec:I-outline}}
