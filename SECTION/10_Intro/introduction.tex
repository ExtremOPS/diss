\chapter{Introduction\label{ch:intro}}

This thesis combines an optical trap (OT) and an acoustic trap (AT) to 
investigate multiple acoustic phenomena. Since the thesis discusses the 
combination of the respective different research areas, in the following 
sections, we will first give an introduction to acoustofluidics, then summarize 
the history of the OT-AT combination, then discuss the background of two 
acoustic phenomena that we investigated with the OT ((1) transient build up of 
the acoustic radiation force and acoustic streaming; (2) direct rotational 
velocity measurement of spherical particles driven by the acoustic viscous 
torque), and then outline the remaining chapters of the thesis.

\section{Acoustofluidics}

During the second half of this doctoral thesis the global pandemic of Covid-19 
was ongoing. The symptoms and the course of the disease depends on the variant 
of the Covid-19 virus. Fortunately up to now, all Covid-19 variants were not 
highly lethal. Nevertheless, the pandemic could not be stopped within 2 years 
although researchers managed to engineer working vaccinations against 
it~\cite{Polack2020,Mahase2020,Voysey2021}. One of the reasons is, that 
Covid-19 has a rather long incubation period when people often do not 
experience any symptoms, yet, but can already transmit the disease. Especially 
in the early stages of this pandemic the key for preventing further spreading 
was testing and quarantining of people.

Besides finding an useful and working test procedure for the detection of 
infected people, another problem was the decentralized testing facilities. At 
the moment, the most accurate test is a polymerase chain reaction (PCR) test 
that is conducted in biomedical labs. These labs are already rare in the 
northern hemisphere but almost non-existing in developing countries. With the 
invention of a rapid antigen test for Covid-19 that could be performed at home 
and by everybody without a special training, the testing was decentralized and 
rapidly increased in total number of persons tested. Having a rapid test that 
is precise~\cite{Albert2021} and that gives results within \SI{15}{\minute} 
constrains further spreading of the disease because it is much faster than the 
\SI{24}{\hour} to \SI{48}{\hour} that it takes to get the result of a PCR test. 
Also, the rapid antigen test is more convenient because it can be taken 
anywhere and by anyone.

Covid-19 is one good example that one of the keys for the prevention of local 
and global spreading of any disease is fast and easy detection of infected 
people. During the beginning of the Covid-19 pandemic a major hindering factor 
was the availability of biomedical testing facilities. Micro-scale 
acoustofluidics (MSAF) offers the possibility to miniaturize whole biomedical 
labs onto a single chip. MSAF possesses the ability to manipulate micro-meter 
sized objects in a controlled way while being immersed in a fluid. The two main 
forces causing the manipulation is the so called acoustic radiation force (ARF) 
and the drag force from acoustic streaming (AS). The manipulation is -- amongst 
others -- label-free and biocompatible and hence interesting for biological and 
medical applications. \emph{Acoustic traps} (ATs) is one synonym for the effect 
of object manipulation and trapping by means of acoustic waves.

The mini-labs are often referred to as \emph{Lab on a Chip} and are not bound 
to a central testing facility because the final chip usually does not need the 
working environment of a biological or medical lab and could therefore be used 
together with related electronics as a stand a lone unit. The target object 
size for MSAF devices is in the \si{\um} range; therefore this technology is 
not yet suitable for viral diseases like Covid-19 where the virus has a 
diameter between \SI{60}{\nm} to \SI{140}{\nm}~\cite{Bar-On2020}. However, 
recently \cname{Gu2020} were able to also handle objects in the 
\si{\nano\meter} range, \cname{Gerlt2022} successfully manipulated in a 
standard glass capillary polystyrene (PS) and metal particles with a radius of 
$\R=\SI{500}{\nm}$ in two dimensions, and \cname{Evander2015} captured 
platelet-derived microparticles of \si{\nm} size from human plasma samples. 
Not only the minimal possible object size is a challenge for broad MSAF 
applications but also the production of MSAF devices.

MSAF is a rather new technology that is also depending on the advances in 
microfabrication. Up to now, the fabrication process of MSAF devices is mainly 
done in clean rooms offering the possibility to fabricate very precisely 
\si{\um}- but also \si{\nm}-structures. However, clean rooms are very expensive 
and not available everywhere. In addition, the needed know-how for the 
production is high. More recently, there emerges the trend to investigate also 
other device fabrication techniques with the benefit of easier, cheaper, and 
more accessible production: using standard glass capillaries as fluid 
channels~\cite{Wiklund2001,Hammarstrom2010,Hammarstrom2012,Mishra2014,Gralinski2014,Gerlt2022}; 
producing MSAF devices with rapid prototyping~\cite{Adams2012}; using micro 
machined fluid channels in aluminum with polydimethylsiloxane (PDMS) 
coverslips~\cite{Gautam2018}; building a glass-PDMS-glass chip with standard 
glass coverslips~\cite{Xu2019}; directly machined fluid channels into poly 
methyl methacrylate (PMMA)~\cite{Harris2012}; building microfluidic devices out 
of paper~\cite{Martinez2010}; bonding two parts of standard thermoplastics 
together to create closed microfluidic structures~\cite{Mueller2013}; or, last 
but not least, directly using devices made out of 
PMMA~\cite{Gonzalez2015,Yang2017}. By making the production more accessible and 
by increasing the size range of manipulatable objects, MSAF increases its 
research audience.

At the moment MSAF is at a point in time where the citations with the keyword 
\emph{acoustofluidics} increased by one order of magnitude over the past 10 
years (2012-2021)~\cite{Novotny2021} and where the research turns from pure 
observations of MSAF phenomena like particle 
manipulation~\cite{Wiklund2012a,Laurell2007,Collins2016,Gedge2012,Ding2012b,Cetin2016b,Aubert2016,Novotny2021}, 
acoustic streaming~\cite{Hoyos2013,Castro2016,Lei2016}, or the effects on the 
AS and the ARF around sharp 
edges~\cite{Leibacher2015,Chen2021,Doinikov2020,Doinikov2020b} to biological 
and medical applications. Amongst many, seven typical applications using the 
MSAF technology for bio(-medical) are:

MSAF is already used (1) as a handling toolbox for cancer 
research~\cite{Antfolk2015,Wu2021,Wang2020,Nguyen2021}, (2) for the analysis of 
single cells suspended in \si{\nano\liter} droplets~\cite{Gerlt2020a}, (3) for 
the measurement of the compressibility of \emph{Caenorhabditis 
elegans}~\cite{Baasch2018} which is a model organism frequently used for 
biomedical studies, (4) for the focusing of metal particles to possibly 
increase the resolution of three dimensional metal printing~\cite{Gerlt2022}, 
(5) as mixer of fluids utilizing 
AS~\cite{Patel2014,Ozcelik2014,Bachman2020,Zhang2021}, (6) for single cell 
analysis of HeLa cancer cells by controlled acoustic induced 
rotation~\cite{Laeubli2021}, and (7) as a tool for continuous exchange of 
medium for bacteria~\cite{Gerlt2021}.

Often the limiting factor of future MSAF applications is the size of the object 
of interest because the drag force from AS and the ARF are competing forces, 
where the ARF should dominate~\cite{Barnkob2012}. The drag force from AS scales 
linearly with the object radius and the ARF scales linearly with the object 
volume. For spherical objects made out of silicone dioxide (\SiO) the critical 
radius where those two forces are equal in magnitude is about \SI{1}{\um} for 
water-like fluids and a typical excitation frequency in the low 
\si{\mega\hertz} range. Bigger objects are dominated by the ARF and smaller by 
the drag force from AS. In most cases AS is undesired because the 
directionality and the occurrence of the flow is hard to control. Therefore, a 
good understanding of the pressure field and the fluid flow inside the devices 
is one key for future MSAF applications outside of research facilities.

\section{Optical Trapping for Acoustics}

With the Noble Prize winning technology of optical trapping by Arthur 
Ashkin~\cite{Ashkin1978,Ashkin1987,Ashkin2002,Ashkin1986,Ashkin1992,Ashkin1997} 
the possibility exists to combine the AT and OT to gain further understanding 
of the physics inside an acoustofluidic device.

\cyear{Thalhammer2011} were the first to combine a \SI{1064}{\nm} OT with an 
AT. They designed their OT such that it has a low numerical aperture (NA) 
enabling a wider field of view. Usually, OTs need a high NA to create enough 
trapping force magnitude. In their study, a low NA was possible because they 
could mimic a dual-beam OT with a single laser beam by the usage of a mirror 
for the laser. They used their setup in two ways: 1) they combined an AT with 
an OT to trap big particles against gravity (AT part) and precisely manipulate 
them (OT part) in the plane of the AT pressure node; 2) they used the OT as a 
force sensor for the AT to measure the acoustic forces acting on non-spherical 
objects.

In a similar way, \cname{Bassindale2014}~(\citeyear{Bassindale2014}) used a 
\SI{1064}{\nm} single beam high NA OT as a force sensor on \SI{5}{\um} silica 
particles. Furthermore, they split the measured force on the particle into a 
contribution from AS and from the ARF. They validated their measurements by 
comparing it to the forces retrieved from particle image velocimetry.

\cyear{Fury2014} utilized a \SI{1064}{\nm} single beam high NA OT together with 
an AT to manipulate microbubbles with a size of \SI{13}{\um}. Their aim was to 
utilize the wide trapping range of the AT, as well as the fact that the AT can 
trap multiple particles simultaneously, together with the high selectivity and 
fine spatial resolution of the OT. Additionally, they measured the forces 
acting on the microbubble for varying parameters of the AT and OT (acoustic 
frequency, excitation voltage, laser power, beam diameter).

\cyear{Lakaemper2015} built a \SI{980}{\nm} single beam high NA OT to 
characterize in two dimensions the time-averaged acoustic forces on \si{\um} 
particles in a standing pressure wave field. They measured the acoustic forces 
at different positions in the standing pressure wave on a single particle while 
sweeping the frequency. They compared their results with numerical simulations 
and found good agreement. With the measured force amplitudes they were also 
able to measure the acoustic pressure inside the acoustofluidic device.

\cyear{Lamprecht2016} extended the work of~\cname{Lakaemper2015} to measure 
with a \SI{785}{\nm} single beam high NA OT the acoustic forces in all three 
spatial dimensions. Their main objective was to visualize the pressure field 
throughout the fluid volume for different frequencies. In addition, they 
measured with the same experimental settings the acoustic forces with two 
different sizes of particle in order to split the total measured forces into 
linear contributions from AS and cubic contributions due to the ARF. They also 
used two devices with different channel widths (\SI{2}{\mm} and \SI{4}{\mm}) to 
show the scaling of the acoustic wavelength with the channel width.

\cyear{Thalhammer2016} used a \SI{1064}{\nm} single beam high NA OT with a 
layered AT to measure primarily the axial acoustic forces on a particle. 
Furthermore, they investigated the influence of the oil immersion layer of the 
lens on to the acoustic resonances and defined design guides for limiting the 
perturbations on the acoustic field of the immersion oil. Their force 
measurement technique is independent of the particle shape allowing force 
measurements of arbitrary shaped small objects. Additionally, their driving 
acoustic transducer is transparent for the laser wavelength such that it can be 
placed in close proximity of the fluid chamber.

Although OTs exist since the 1990s, the combination with an AT is rather young. 
Therefore, many areas for this unique combination are still not even opened. 
The next two sections will introduce the two acoustic phenomena that are 
studied with an OT in the underlying thesis.

\section{Advances in Streaming Suppression}

As stated before, AS is often undesired because it can counteract the ARF. 
Besides the fluid mixer driven by AS, there are also other applications where 
AS is utilized advantageously; a summary is given in~\cite{Wiklund2012a}. 
Nevertheless, the suppression of AS is still of high interest for MSAF 
applications, because the controlled manipulation of smaller particles is often 
hindered by the drag force from AS.

\cyear{Hoyos2013} demonstrated that a pulsed ultrasound excitation can reduce 
and suppress AS while the ARF is still present. They studied in two MSAF 
devices the effects of different pulse settings to the steady-state streaming 
velocity and the ARF. Both device designs yielded similar results showing that 
it is not geometry dependent. To visualize both effects simultaneously they 
used \SI{800}{\nm} and \SI{15}{\um} latex particles. The smaller ones are 
dominated by the drag force from AS and the larger ones by the ARF. In 
addition, they could show that the secondary forces, like particle-particle 
interaction, are still present with a pulsed acoustic excitation which allows 
trapping of small particles with larger ones as seed particles.

\cyear{Castro2016} extended their study of~\cname{Hoyos2013} to investigate the 
effects of a pulsed excitation on the steady-state AS while also changing other 
experimental parameters, like the particle size and device geometry. They 
showed that there are parameter settings for which latex particles smaller than 
\SI{2}{\um} are manipulated only by the ARF without any AS perturbations.

\cyear{Karlsen2018a} performed a theoretical, numerical, and experimental study 
on the effects of fluid inhomogeneities in terms of density and compressibility 
on the boundary driven AS. The fundamental idea is to utilize the different 
time scales for the build up of the ARF and AS. They realised the 
inhomogeneities in the device by layering water with two sheath flows of 20\% 
iodixanol at the respective channel walls. With this layered fluid 
configuration they were able to suppress AS in the middle of the device almost 
completely. The numerical model verified their experimental results.

\cyear{Bach2020} performed a numerical study with the objective to optimize the 
cross-section of the fluid channel to reduce streaming while still allowing ARF 
driven particle manipulation. With the optimized shape, they were theoretically 
able to suppress streaming in the bulk of the homogeneous fluid by two orders 
of magnitude. For a rectangular cross-section and an acoustic frequency of 
\SI{1.95}{\mega\hertz} the minimal particle radius of a PS particle in water is 
around \SI{1}{\um} before the drag force from AS dominates over the ARF. With 
their optimized shape they reduce the minimal particle radius to 
\SI{0.15}{\um}.

Most recently in 2021, \cname{Winckelmann2021} investigated theoretically and 
numerically the effects of alternating current electroosmosis (ACE) to the 
streaming pattern within a rectangular fluid channel. As it is for AS, the 
streaming from ACE is also boundary driven. By analysing the form of the ACE 
streaming they are able to suppress the streaming in the bulk of the fluid by 
two orders of magnitude. The applied voltage for the ACE streaming to 
successfully suppress the AS is about \SI{125}{\milli\volt}. In addition, the 
bulk of the fluid is of neutral charge and therefore no electric charges could 
harm possible living objects within the system.

These five publications show the rising interest in the topic of acoustic and 
other streaming suppression and the necessity to further investigate AS; 
especially its origin and its build up experimentally.

\section{Viscous Torque induced Spherical Particle Rotation}

\cyear{Lamprecht2015} investigated the equilibrium rotational velocity of 
spherical particles that are within two orthogonal standing pressure wave 
fields of same frequency extending the work of \cname{Wang1989} which neglected 
the rotational term in their calculation. In addition to being orthogonal to 
each other, the phase of the acoustic excitation was shifted such that a 
streaming field within the VBL of the particle had formed which creates a 
torque acting on the particle. They call this driving torque acoustic viscous 
torque. Under the assumption that the radius of the particle is much larger 
than the viscous boundary layer (VBL) thickness $\R\gg\delta$ they derived an 
analytical formula for the steady-state rotational speed where the driving 
acoustic viscous torque equals the oppositely directed viscous drag torque.  
They validated their findings with experiments of different sized particles in 
water. They measured the rotational final velocity on particles ranging from 
$\R=\SI{35.5}{\um}$ to $R=\SI{325}{\um}$ by investigating the single frames of 
an high-speed camera.

\cyear{Hahn2016} conducted a numerical study on spherical and non-spherical 
particles that are subjected to the same configuration of two standing pressure 
waves that are orthogonal and phase shifted with respect to each other as in 
\cname{Lamprecht2015}. With their numerical model they were able to investigate 
the final rotational speed without any restriction to the ratio of $\R$ and 
$\delta$. They validated the theoretical formula of~\cite{Lamprecht2015} for 
the regime $\R\gg\delta$ and found, unsurprisingly, large deviations between 
the numerical and analytical model for the regime $\R\leq\delta$. However, they 
could not validate their numerical findings with experiments because it was not 
possible to measure the high steady-state rotational speeds of small particles 
with the property $\R\leq\delta$ due to the required minimal frame rate of the 
camera.

\section{Scope of this Thesis}

Beside the trapping capabilities of the OT, one of the main advantages is the 
detection system in the back focal plane. The detection is performed by photo 
detectors which convert the incoming laser intensity to a voltage. Fast 
measurements of voltages with time resolutions smaller than $\Delta t < 
\SI{1}{\us}$ are straightforward and inexpensive. Here, we utilize the fast 
measuring capabilities of our OT to measure for the first time two MSAF 
phenomena (one transient MSAF phenomenon and one steady-state MSAF phenomenon) 
that cannot be measured optically otherwise.

The transient phenomenon is the build up of the acoustic radiation force and 
the drag force from acoustic streaming for a continuous and pulsed acoustic 
excitation. And the steady-state phenomenon is the rotational speed 
measurement of spherical particles driven by the acoustic viscous torque where 
the particle radius is as big as the VBL thickness $\R\approx\delta$.

\section{Outline of Thesis\label{sec:I-outline}}

This thesis is a cumulative thesis. It consists of three peer-reviewed 
published papers surrounded by an extended introduction at the beginning and by 
a summarizing discussion and outlook at the end.

We describe in this manuscript the combination of an optical trap with an 
acoustical trap. Therefore, in \cref{ch:ac-theory} we discuss and explain the 
fundamental acoustic theory. The following \cref{ch:optic-theory} discusses the 
optical theory relevant in the context of this thesis. The whole optical 
trapping setup will not be explained in great detail. Interested readers are 
pointed to the preceding thesis by \cname{Lamprecht2017} who explains the used 
components and the whole setup.

\Cref{ch:timeconstant,ch:pulsing} are two publications which investigated the 
effects of a pulsed excitation onto the build up of the ARF and AS. The first 
one introduces the necessary changes to the previous OT setup in order to 
measure the build up of the two effects and, additionally, the build up with a 
continuous excitation is measured. The following publication is a direct 
extension where this new measuring technique is applied to a pulsed excitation. 
The results showed that the build up of the ARF and AS are affected differently 
by a pulsed acoustic excitation.

In \cref{ch:viscoustorque} the OT is utilized for measuring the rotational 
speed of spherical particles in a viscous fluid driven by two phase-shifted 
orthogonal standing pressure waves. The combination of high viscosity and small 
particles leads to a VBL thickness in the order of the particle radius itself. 
We validated the effects of the viscosity on the rotational speed in this 
regime and showed that a simplified theoretical formula~\cite{Lamprecht2015} is 
overestimating the rotational speed by more than one order of magnitude.

Finally, in \cref{ch:discussion} we summarized all presented findings, put them 
in context, and motivated future possible research directions.
