\chapter[Discussion \& Outlook]{Discussion and Outlook}\label{ch:discussion}

\section{Discussion}
In this thesis we presented the application of the fast position readout of a 
trapped spherical particle within our optical trapping (OT) setup to two micro 
scale Acoustofluidics (MSAF) phenomena: 1) the transient build up behaviour of 
acoustic streaming (AS) as well as of the acoustic radiation force (ARF); 2) 
the viscous torque induced rotation of spherical particles in a viscous fluid 
where the viscous boundary layer (VBL) is as large as the particle radius 
itself $\delta \approx R$.

The investigations of both phenomena are possible because the laser beam 
causing the trapping potential \emph{carries} positional information about the 
particles relative position to the laser focal point. In order to extract this 
information from the light beam, the beam must be collimated after the trapping 
and focused onto photo detectors. For the resolution of two orthogonal 
directions of the in-plane movement, the photo detector (PD) must be a quadrant 
photo detector (QPD). For the axial movement a single PD is sufficient because 
only the total intensity onto the PD is needed.

The physical unit of the (Q)PDs output is \si{\volt}. In order to convert the 
voltage to the unit of \si{\meter}, we take advantage of the linear regime of 
(Q)PDs where a change in measured voltage is proportional to the movement of 
the particle within the trap. In this regime, the OT can also be used as force 
measurement device because it has the same properties as a linear mechanical 
spring. The respective start and end point of this regime is dependent on a 
multitude of parameters. Movements of less than \SI{100}{\nm} are still linear 
related for our OT setup. The voltage-conversion factor as well as the 
stiffness of the OT can be derived from the single-sided frequency spectrum of 
the trapped particle.  The particles in the fluid suspended are so lightweight 
and small that they undergo visible Brownian motion. Brownian motion is a 
random process with the property that over a long time period the particle will 
be back at its initial position. Also while being trapped the particle 
undergoes Brownian motion. The frequency content of a trapped particle in any 
spatial direction follows the curve of a low-pass filter. With the amplitude at 
zero frequency $A_{0}$ and the cutoff frequency $f_{\MR{c}}$ where 
$A\vert_{f=f_{\MR{c}}}=\sfrac{A_{0}}{2}$ it is possible to calculate the 
voltage-meter conversion factor as well as the stiffness for each spatial 
direction separately.

Besides the size of the trapped particle the conversion factor is dependent on 
the viscosity of the surrounding fluid. Exact knowledge of the viscosity is 
unavoidable for quantitative precise measurements. The viscosity of the fluid 
is in general a function of its temperature. While the ambient temperature is 
straightforward to measure, the measurement of the temperature within a MSAF 
device or at the focal point of the trapping laser is cumbersome. Although the 
intensity of our laser exceeds the sun's intensity by order of magnitudes, we 
showed in \cref{sec:TO-temperature} that for our setup the laser induced 
temperature change and therefore the temperature induced viscosity change is 
negligible small for all our experiments. However, this is not true for every 
OT because the temperature change does not only dependent on the laser power 
itself but even more on the laser wavelength-fluid combination, and therefore 
the absorption at this specific wavelength.

To investigate the first MSAF phenomena, we develop a new measurement method in 
\cref{ch:timeconstant} to visualize the transient build up of the ARF and AS 
and then study in \cref{ch:pulsing} the effects of a pulsed excitation on the 
respective build up times. Key for those experiments is, that the ARF and AS 
induced particle displacement is along orthogonal spatial direction. Moreover, 
the ARF displacement is along on of the axis of the in-plane QPD and the AS 
displacement is along the axial direction on another QPD. Hence, the measured 
displacement data is also separated and cross-talk free. We validated the 
orthogonality of the ARF and the drag force from AS by steady state force 
measurements with two different sized particles where all experimental 
parameters besides the particle radius kept the same. The drag force from AS 
scales with the particle radius, $\FAS\propto R$ and the ARF with the volume of 
the particle, $\FARF\propto R^{3}$. The ratio of the measured forces along same 
directions but of different particles sizes showed the same scaling laws. 

However, without modifications on the OT we are not able to measure any of 
those effects because the timeconstant of the OT $\tOT$ is much larger than the 
theoretical build up time of the ARF. In fact, for our set of experimental 
parameters and our device configuration $\tOT \approx \tas \gg \tarf$ with 
$\tas = \SI{1.59}{\ms}$ and $\tarf = \SI{1.4}{\us}$, respectively. We reduced 
$\tOT$ effectively to zero by reducing the laser power to almost zero such that 
there is no effective trapping by the OT. We still need some laser power 
because the beam is the cause for the measured intensity at any of the (Q)PDs.

Because of the laser power reduction, we also needed to install an optical 
shutter right before the (Q)PDs. The laser intensity is too high for the photo 
detectors. In normal trapping mode a set of filters reduces the intensity to 
less than 0.01\% of the incoming intensity. The optical filter is actuated to 
be almost completely closed such that it acts as intensity filter if the laser 
power is high (trapping) and opens as soon as the laser power is reduced for 
the measurement itself.

The laser power change occurs almost instantaneously but the shutter has a 
opening time of less than \SI{15}{\ms}. To ensure undisturbed data we wait 
\SI{25}{\ms} before starting the acoustic excitation. During this time the 
particle will start to sediment because it is not trapped stably anymore. 
However, this movement is only $\approx 0.05\,R$ along the axial direction and, 
hence, does not hinder the measurement; the in-plane movement is zero because 
no forces besides gravity are acting on the particle.

The (pulsed) acoustic excitation is then switched on for only \SI{30}{\ms} 
because a longer excitation would lead to such large displacements -- primarily 
along the ARF direction -- that the particle could not be re-trapped. For our 
studies we were not interested in the actual displacements but in the time it 
takes until the movement along any direction starts. For actual displacements 
magnitudes one would need to retrieve the voltage-meter conversion factor but 
for that the particle needs to be trapped. Hence, only the voltage at the 
(Q)PDs can be investigated.

Because the particle is free-floating and, therefore, uncontrolled in its 
location for the first \SI{25}{\ms} we repeat the measurement multiple times 
per location and average the measured data. To further subtract the gravity 
induced movement from the data we perform the same amount of measurements with 
no acoustics at all and then subtract the averaged \emph{no-acoustic} data from 
the \emph{acoustics} data.

The data for both types of experiments (non-pulsed and pulsed) showed that the 
ARF induced displacement starts immediately whereas the AS induced displacement 
takes significantly longer. The measured time offset between ARF and AS is an 
explanation why in experiments~\cite{Castro2016,Hoyos2013} a pulsed acoustic 
excitation suppresses the build up of AS whereas numerical 
investigations~\cite{Muller2015} with an ideal fluid cavity reveal a much 
smaller offset which does not suppress streaming. Although the data has the 
unit of \si{\volt} and not \si{\meter}, the slope the AS and ARF induced 
displacement is showing that the acceleration after the build up by any of two 
effects occurs fast and then transitions into constant velocity (linear slope). 
% It is a valid statement because the measured voltage (within the linear regime 
% of the (Q)PDs) is the displacement divided by the voltage-meter conversion 
% factor. The slope of the voltage-over-time data equals the slope of the 
% displacement which is in turn the velocity of the particle.

Additionally, the pulsed excitation measurements revealed that a pulsed 
excitation leads to a greater reductions in terms of the final displacement at 
the end of the measurement on the AS than it does on the ARF. We measure this 
observation for all pulsing frequencies. The ARF and AS are linear dependent on 
the acoustic energy density within the fluid cavity. All our data also showed 
this relation but with a steeper slope for AS than the ARF. The pulsed 
excitation results help to further explain the experimental shown suppression 
of AS. However, with our measurement protocol we can only measure for the first 
\SI{30}{\ms} of acoustic excitation, and any MSAF application is operated 
continuously where all fields have developed fully. At the moment, we cannot 
measure these effects during steady-state.


Compared to the first phenomena, no set up modifications at the OT are 
necessary for the second MSAF phenomena: the rotational speed measurement of a 
spherical particle driven by the viscous torque (\cref{ch:viscoustorque}). The 
two main requirements for the generation of a viscous torque that is 
sufficiently big to initiate a particle rotation are: firstly, the formation of 
a VBL $\delta$ around the particle; secondly, a two 
dimensional orthogonal acoustic excitation with phase difference such that the 
acoustic pressure at the particle surface has a continuous phase change in 
circumferential direction.

The second requirement is straightforward to accomplish by the use of a 
quadratic fluid cavity where the excitation is applied to the system from two 
orthogonal directions. The phase difference of the two excitation signals is 
easily realized with most dual channel function generators. The viscosity 
increase of the fluid and therefore the larger VBL around 
the particle can be created by a glycerol water mixture (3:7 for our 
experiments). A greater viscosity increase was not possible, because glycerol 
has the side effect that the glycerol also increases the refractive index of 
the mixture.  For stable trapping the refractive index of the particle must be 
larger than of the surrounding medium. For equal indices the particle is 
\emph{invisible} for the laser beam because no path change of the light occurs 
at the particle-fluid interface. A greater index of refraction of the particle 
than the fluid index leads to repulsion of the particle from the laser focal 
point.

Our measurement utilizes the property that a distinct amplitude peak appears at 
the rotational frequency of the particle in the one sided power spectrum of the 
trapped rotating particle. We validated this finding with deformed and partial 
gold coated particles where the rotation is also visible optically. By having a 
highspeed camera mounted to the OT and choosing the excitation parameters such 
that the viscous torque induced rotation is clearly visible by analysis of the 
single frames from the video. The deformed particles did not rotate with a 
single constant velocity because of its non-spherical shape. Therefore, 
multiple additional peaks were also visible in the power spectrum. The 
partially gold coated particles, however, showed a single peak that matched the 
rotational speed from the optical analysis.

With a validated method for measuring the rotational speed of a spherical 
particle we investigated the situation where the particle radius is as big as 
the VBL itself $R\approx\delta$. In the regime where $\delta 
> \sfrac{R}{15}$ a numerical study by \cname{Hahn2016} predicts the final 
terminal rotational velocity order of magnitudes lower than a theoretical 
investigation by \cname{Lamprecht2015}. Note here, that~\cite{Lamprecht2015} 
restrict their calculations to the assumption that $R\gg\delta$. Our 
measurements of rotations up to \SI{13700}{\rpm} confirm the results from 
\cname{Hahn2016} and prove the invalidity of the theoretical formula of 
\cname{Lamprecht2015} in the regime where $R\approx\delta$.

\section{Outlook}

The OT is a great tool for measuring acoustics effects on micro-scale particles 
suspended in a fluid. The four main advantages are: a) measurements on single 
particles; b) high spatial precision and repeatability of the measurement 
locations; c) high temporal resolution of particle position; d) high 
sensitivity to surrounding. The last point is also by far the biggest 
disadvantage. For all our experiments we were very cautious about us being in 
the same lab and any other disturbances, like fast and careless opening/closing 
of the door or like constructions works in the neighboring lab. In any case, 
the combination of the optical trap as characterization/measurement device for 
the acoustical trap offers many possible future research contributions. In the 
following we want to categorize our suggestions in three sections: a) deeper 
theoretical understanding of the OT; b) future experiments with our setup where 
no big modifications are needed; c) more general ideas which would need larger 
modifications to our setup.

\subsection{Optical Trap Calibration}

In this thesis we did not explain the calibration protocol for our OT. As 
before, interested readers are pointed to the thesis of \cname{Lamprecht2017} 
(Section 3.2.4) where our calibration process is explained in great detail. The 
basic procedure is to fit a low-pass filter to the frequency content of a 
trapped particle. This fitting is based on two parameters: the amplitude 
$A_{0}$ at zero frequency and the cut-off frequency $f_{\MR{c}}$. With those 
two parameters the stiffness $\kappa_{i}\propto f_{\MR{c}}$ and the 
voltage-meter conversion factor $r_{i}\propto \sfrac{1}{(f_{\MR{c}}\sqrt{A_{0}} 
)}$ can be calculated for all three spatial directions separately. The total 
force is then simply the product of the measured voltage times the two factors 
and, hence, \begin{equation}
  \force_{i,\MR{measured}}\propto \kappa_{i}\,r_{i}\propto 
  \sfrac{1}{\sqrt{A_{0}}}.
\end{equation}
Interestingly, the measured force is only proportional to one of those fitting 
parameters $A_{0}$. The fit itself, especially, the parameter $A_{0}$ depends 
highly on the beginning of the data range used for the fit. From experience we 
used for our OT calibration the frequency spectrum data ranging from 
\SI{20}{\hertz} up to \SI{6}{\kilo\hertz} for the $\ex$ and $\ey$ direction and 
\SI{10}{\hertz} to \SI{2}{\kilo\hertz} for $\ez$. The $\ez$ direction has a 
smaller and lower range because it is known that OTs in general are weaker 
along the beam axis and, hence, the cut-off frequency is also lower.

However, the error introduced by a false data range is \emph{just} a false 
scaling of the forces. Measurements that use the same fitting procedure can 
still be compared because all measurements might share the same scaling error. 
Up to our knowledge we are not aware of any publication that discusses (and 
resolves) this small but important detail for the power spectrum OT 
calibration.

\subsection{How should I call you?}

Although we investigated two MSAF phenomena with our OT setup, there are many 
more open questions and interesting projects. We select three projects which we 
think are of great interest for the MSAF community and also achievable without 
any modifications to our present OT setup.

The first is a variation of \cref{ch:timeconstant,ch:pulsing}. In a first step 
we were able to measure the build up of the two effects, but only for the first 
120'000 excitation periods ($\approx \SI{30}{\ms}$). Although we could show 
that AS is significantly slower in its build up than the ARF, we could not 
investigate the steady-state behavior of a pulsed excitation.  Therefore, we 
suggest to measure the steady-state forces exerted on a single particle for 
various pulse frequencies settings and compare the forces to a non-pulsed 
measurement. To reduce uncertainties between different settings one should 
create a protocol where all data that will be compared to each other is 
measured in one device initialization.

The second possible field of interest is also investigating AS and the ARF.  
However this time, one would be investigating the scaling laws of AS and the 
ARF. From theory we know that they scale $\propto R$ and $\propto R^{3}$ for a 
spherical particle, respectively. The total acoustic force 
$\forcevector^{\MR{ac}}$ is understood to contain a linear part from AS and a 
cubic part due to the ARF
\begin{equation}
  \forcevector^{\MR{ac}}\left( R \right) = \vb{C}^{\MR{AS}}\,R + 
  \vb{C}^{\MR{rad}}\,R^{3}.
\end{equation}
From force measurements of two different sized particles with the same 
experimental condition it is possible to extract the constants 
$\vb{C}^{\MR{AS}}$ and $\vb{C}^{\MR{rad}}$. Therefore one could image the pure 
fields that is creating the ARF and also that is due to AS. In order to 
validate, that the assumptions holds that the total acoustic force is mainly 
composed of the two above contributions a third measurement with another 
particle size is needed. Based on the constants $\vb{C}^{\MR{AS}}$ and 
$\vb{C}^{\MR{rad}}$ one is able to predict the force field in all three 
dimensions for the third particle size $R_{3}$
\begin{equation}
  \forcevector^{\MR{ac}}\left( R_{3} \right) = \vb{C}^{\MR{AS}}\,R_{3} + 
  \vb{C}^{\MR{rad}}\,R^{3}_{3}.
\end{equation}
The main challenge for this study is the to ensure the same pressure amplitude 
inside the cavity for thee three different particle sizes.

The third and last potential study we want to discuss here is a controversy 
about the ARF for heavy particles in viscous fluids. \cname{Baasch2019} showed 
numerically that the contributions of microstreaming around heavy particles in 
a viscous fluid are of such significance that they could lead to a 
force-inversion. Their findings strengthen the theory of 
\cname{Doinikov1994Rigid} which also predicts this force inversion whereas the 
theory of \cname{Settnes2012} does not. With the measurement of the rotational 
speed in a glycerol-water mixture we could already show that the OT can operate 
within viscous fluid. Besides being heavy the other main requirement is to be 
trappable for the OT setup. Additionally, those particles should have the same 
optical absorption as the surrounding fluid because otherwise the laser induced 
heating might be too high such that the changes in viscosity are not negligible 
small anymore.

\subsection{How should I call you?}

For the final part of this thesis we will motivate three possible studies that 
are not applicable to our specific OT setup. For the first one, the main 
limitation is not directly our OT but the devices we use. The main requirement 
for our devices is that they are transparent from top to bottom for our laser 
light as well as being wide enough to accommodate the converging cone of the 
laser beam.  Therefore, we cannot measure close to walls because because the 
silicon walls are non-transparent for our laser wavelength and hinder the laser 
beam and, hence, would violate the assumption of a symmetrical trapping 
potential which is the basis for the trap calibration. Acoustic wall effects 
are an interesting phenomena that is important for narrow channels and could 
lead to further applications. Devices where also the channel side walls are 
made out of glass (all-glass devices) might be an interesting opportunity for 
studying these effects. However, one would need to investigate the refractive 
index of the fluid and the specific glass carefully.

For our investigated applications this is not a problem because the two 
interfaces of the top cover glass with the immersion oil layer and the fluid 
inside the cavity are parallel. Therefore, the beam passes this interface 
without a change in direction. However, in an all-glass device which would be 
used for measuring close to the channel wall some parts of the laser beam will 
be going through the channel wall. For different refractive indices of the 
fluid and the glass this fact will lead to a deflection of the laser which in 
turn will violate the assumption of a symmetric optical potential. The 
magnitude of the violation is proportional to the index difference. For 
matching indices of the device material and the fluid no deflection at any of 
those interfaces occurs.

Secondly, with a OT consisting of two lasers where one is creating the optical 
potential for trapping and the other is the light source for the position 
detection at the QPDs one is able to re-perform the build up experiments.  
Besides a validation of our results with another setup and device, one would 
also eliminate the lead time before the acoustic excitation starts. This lead 
time was necessary in our case because we have a single beam OT and needed to 
open the optical shutter before we could start the acoustic excitation.

Lastly, our device is a bulk acoustic wave (BAW) device where a standing 
pressure waves is formed in the bulk of the fluid. The other major kind of 
devices are surface acoustic wave (SAW) devices. In those devices opposing 
traveling waves at the coverplate-fluid interface create a standing pressure 
field inside the fluid that can also manipulate objects. Up to our knowledge no 
direct force measurement were performed with SAW devices so far.
