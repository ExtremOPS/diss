\chapter[Discussion \& Outlook]{Discussion and Outlook}\label{ch:discussion}

\section{Discussion}
In this thesis we presented the application of the fast position readout of a 
trapped spherical particle within our optical trapping (OT) setup to two micro 
scale acoustofluidics (MSAF) phenomena: 1) the transient build up behaviour of 
acoustic streaming (AS) as well as the acoustic radiation force (ARF); 2) the 
viscous torque induced rotation of spherical particles in a viscous fluid where 
the viscous boundary layer is as large as the particle radius itself $\delta 
\approx R$.

The investigations of both phenomena are possible because the laser beam which 
is the cause for the trapping potential \emph{transfers} positional information 
about the particles relative position to the laser focal point after the actual 
trapping takes place. In order to extract this information from the light beam, 
the beam must be collimated after the trapping and focused onto photo 
detectors. For the resolution of two orthogonal directions of the in-plane 
movement, the photo detector (PD) must be a quadrant photo detector (QPD). For 
the axial movement a single PD is sufficient because only the total intensity 
shine onto the PD is needed.

The physical unit of the (Q)PDs output is \si{\volt}. In order to convert the 
voltage to the unit of \si{\meter}, we take advantage of the linear regime of 
(Q)PDs where a change in measured voltage is proportional to the movement of 
the particle within the trap. In this regime, the OT can also be used as force 
measurement device because it has the same properties as a linear mechanical 
spring. The respective start and end point of this regime is dependent on a 
multitude of parameters. For our setup movements of less than \SI{100}{\nm} are 
still linear related. The voltage-conversion factor as well as the stiffness of 
the OT can be derived from the single-sided frequency spectrum of the trapped 
particle. The in the fluid suspended particles are so lightweight and small 
that they undergo visible Brownian motion. Brownian motion is a random process 
with the property that over a long time period the particle will be back at its 
initial position. Also while being trapped the particle undergoes Brownian 
motion. The frequency content of a trapped particle in any spatial direction 
follows the curve of a low-pass filter. With the amplitude at zero frequency 
$A_{0}$ and the cutoff frequency $f_{\MR{c}}$ where 
$A\vert_{f=f_{\MR{c}}}=\sfrac{A_{0}}{2}$ it is possible to calculate the 
voltage-meter conversion factor as well as the stiffness for each spatial 
direction separately.

Besides the size of the trapped particle the conversion factor is dependent on 
the viscosity of the surrounding fluid. Exact knowledge of the viscosity is 
unavoidable for quantitative precise measurements. The viscosity of the fluid 
is in general a function of its temperature. While the ambient temperature is 
straightforward to measure, the measurement of the temperature within a MSAF 
device or at the focal point of the trapping laser is cumbersome. Although the 
intensity of our laser exceeds the sun's intensity by order of magnitudes, we 
showed in \cref{sec:TO-temperature} that for our setup the laser induced 
temperature change and therefore the temperature induced viscosity change is 
negligible small for all our experiments. However, this is not general for 
every OT the case because the temperature change does not only dependent on the 
laser power itself but even more on the laser wavelength and fluid combination, 
and therefore the absorption at this specific wavelength.

To investigate the first MSAF phenomena, we develop a new measurement method in 
\cref{ch:timeconstant} to visualize the transient build up of the ARF and AS 
and then study in \cref{ch:pulsing} the effects of a pulsed excitation on the 
respective build up times. Key for those experiments is, that the ARF and AS 
induced particle displacement is along orthogonal spatial direction. Moreover, 
the ARF displacement is along on of the axis of the in-plane QPD and the AS 
displacement is along the axial direction on another QPD. Hence, the measured 
displacement data is also separated and cross-talk free. We validated the 
orthogonality of the ARF and the drag force from AS by steady state force 
measurements with two different sized particles where all experimental 
parameters besides the particle radius kept the same. The drag force from AS 
scales with the particle radius, $\FAS\propto R$ and the ARF with the volume of 
the particle, $\FARF\propto R^{3}$. The ratio of the measured forces along same 
directions but of different particles sizes showed the same scaling laws. 

However, without modifications on the OT we are not able to measure any of 
those effects because the timeconstant of the OT is much larger than the 
theoretical build up time of the ARF. In fact, for our set of experimental 
parameters and our device configuration $\tOT \approx \tas \gg \tarf$ with 
$\tas = \SI{1.59}{\ms}$ and $\tarf = \SI{1.4}{\us}$. We reduced $\tOT$ 
effectively to zero by reducing the laser power to almost zero such that there 
is no effective trapping by the OT. We still need some laser power because the 
beam is the cause for the measured intensity at any of the (Q)PDs.

Because of the laser power reduction, we also needed to install an optical 
shutter right before the (Q)PDs. The laser intensity is too high for the photo 
detectors. In normal trapping mode a set of filters reduces the intensity to 
less than 0.01\% of the incoming intensity. The optical filter is actuated to 
be almost completely closed such that it acts as intensity filter if the laser 
power is high (trapping) and opens as soon as the laser power is reduced for 
the measurement itself.

The laser power change occurs almost instantaneously but the shutter has a 
opening time of less than \SI{15}{\ms}. To ensure undisturbed data we wait 
\SI{25}{\ms} before starting the acoustic excitation. During this time the 
particle will start to sediment because it is not trapped stably anymore. 
However, this movement is only $\approx 0.05\,R$ along the axial direction and, 
hence, does not hinder the measurement; the in-plane movement is zero because 
no forces besides gravity are acting on the particle.

The (pulsed) acoustic excitation is then switched on for only \SI{30}{\ms} 
because a longer excitation would lead to such large displacements -- primarily 
along the ARF direction -- that the particle could not be re-trapped. For our 
studies we were not interested in the actual displacements but in the time it 
takes until the movement along any direction starts. For actual displacements 
magnitudes one would need to retrieve the voltage-meter conversion factor but 
for that the particle needs to be trapped. Hence, only the voltage at the 
(Q)PDs can be investigated.

Because the particle is free-floating and, therefore, uncontrolled in its 
location for the first \SI{25}{\ms} we repeat the measurement multiple times 
per location and average the measured data. To further subtract the gravity 
induced movement from the data we perform the same amount of measurements with 
no acoustics at all and then subtract the averaged \emph{no-acoustic} data from 
the \emph{acoustics} data.

The data for both types of experiments (non-pulsed and pulsed) shows that the 
ARF induced displacement starts immediately whereas the AS induced displacement 
takes significantly longer. The measured time offset between ARF and AS is an 
explanation why in experiments~\cite{Castro2016,Hoyos2013} a pulsed acoustic 
excitation suppresses the build up of AS whereas numerical 
investigations~\cite{Muller2015} with an ideal fluid cavity reveal a much 
smaller offset which does not suppress streaming. Although the data has the 
unit of \si{\volt} and not \si{\meter}, the slope the AS and ARF induced 
displacement is showing that the acceleration after the build up by any of two 
effects occurs fast and then transitions into constant velocity (linear slope). 
It is a valid statement because the measured voltage (within the linear regime 
of the (Q)PDs) is the displacement divided by the voltage-meter conversion 
factor. The slope of the voltage-over-time data equals the slope of the 
displacement which is in turn the velocity of the particle.

Additionally, the pulsed excitation measurements revealed that a pulsed 
excitation leads to a greater reductions in terms of the final displacement at 
the end of the measurement on the AS than it does on the ARF. We measure this 
observation for all pulsing frequencies. The ARF and AS are linear dependent on 
the acoustic energy density within the fluid cavity. All our data also shows 
this relation but with a steeper slope for AS than the ARF. The pulsed 
excitation results help to further explain the experimental shown suppression 
of AS. However, with our measurement protocol we can only measure for the 
first \SI{30}{\ms} of acoustic excitation. However, any MSAF application is 
operated continuously where all fields have developed fully. At the moment, we 
cannot measure these effects during steady state.

For the second MSAF phenomena, the rotational speed measurement of a spherical 
particle driven by the viscous torque (\cref{ch:viscoustorque}), no set up 
modifications at the OT are necessary. The two main requirements for the 
generation of a viscous torque that is sufficiently big to initiate a particle 
rotation are: firstly, the formation of a viscous boundary layer $\delta$ 
around the particle; secondly, a two dimensional orthogonal acoustic excitation 
with phase difference such that the acoustic pressure at the particle surface 
has a continuous phase change in circumferential direction.

The second requirement is straightforward to accomplish by the use of a 
quadratic fluid cavity where the excitation is applied to the system from two 
orthogonal directions. The phase difference of the two excitation signals is 
easily realized with most dual channel function generators. The viscosity 
increase of the fluid and therefore the larger viscous boundary layer around 
the particle can be created by a glycerol water mixture (3:7 for our 
experiments). A greater viscosity increase was not possible, because glycerol 
has the side effect that it also increases the refractive index of the mixture.  
For stable trapping the refractive index of the particle must be larger than of 
the surrounding medium. For equal indices the particle is \emph{invisible} for 
the laser beam because no path change of the light occurs at the particle-fluid 
interface and a greater particle index leads to repulsion from the laser focal 
point.

Our measurement utilizes the property that a distinct amplitude peak appears at 
the rotational frequency of the particle in the one sided power spectrum of the 
trapped rotating particle. We validated this finding with deformed and partial 
gold coated particles where the rotation is also visible optically. By having a 
highspeed camera mounted to the OT and choosing the excitation parameters such 
that the viscous torque induced rotation is clearly visible by analysis of the 
single frames from the video. The deformed particles did not rotate with a 
single constant velocity because of its non-spherical shape. Therefore, 
multiple additional peaks were also visible in the power spectrum. The 
partially gold coated particles, however, showed a single peak that matched the 
rotational speed from the optical analysis.

With a validated method for measuring the rotational speed of a spherical 
particle we investigated the situation where the particle radius is as big as 
the viscous boundary layer itself $R\approx\delta$. In the regime where $\delta 
> \sfrac{R}{15}$ a numerical study by \cname{Hahn2016} predicts the final 
terminal rotational velocity order of magnitudes lower than a theoretical 
investigation by \cname{Lamprecht2015}. Note here, that~\cite{Lamprecht2015} 
restrict their calculations to the assumption that $R\gg\delta$. Our 
measurements of rotations up to \SI{13'700}{\rpm} confirm the results from 
\cname{Hahn2016} and prove the invalidity of the theoretical formula of 
\cname{Lamprecht2015} in the regime where $R\approx\delta$.

\section{Outlook}

outlook three sizes and the visualize AS and ARF

scaling absolute value with amplitdue of spectrum -> first guess of A0 when 
fitting

pulsed excitation measurement in steady-state! synchronization needed that 
laser shuts of when there is no acoustic excitation happening

outlook for our setup specific and more open minded outlookl
 - heavy particle force scaling
 - OST devices --> near walls
 - heavy particles in viscous fluids
 - SAW
 - dual beam/own detection light source for AS measurement
 - other particle materials
 - refractive index limitation
 - longer AS measurements
 - zurich instruments laser disruption measurements
