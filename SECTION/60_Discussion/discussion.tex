\chapter[Discussion \& Outlook]{Discussion and Outlook}\label{ch:discussion}

\section{Discussion}
In this thesis we presented the application of the fast position readout of a 
trapped spherical particle within our optical trapping (OT) setup to two MSAF 
phenomena: 1) the transient build up behaviour of acoustic streaming (AS) as 
well as the acoustic radiation force (ARF); 2) the viscous torque induced 
rotation of spherical particles in a viscous fluid such that the viscous 
boundary layer is as large as the particle itself $\delta \approx R$.

The investigations of both phenomena are possible because the laser beam which 
is the cause for the trapping potential \emph{transfers} positional information 
about the particles relative to the laser focal point after the actual trapping 
takes place. In order to extract this information from the light beam, the beam 
must be collimated after the trapping and focussed onto photo detectors. For 
the detection of the in-plane movement, the photo detector (PD) must be a 
quadrant photo detector (QPD) for resolving the spatial movement along two 
orthogonal directions. For the axial movement a single photo detector is 
sufficient because only the total intensity is needed.

The physical unit of the (Q)PDs output is \si{\volt}. In order to convert the 
voltage to the unit of \si{\meter}, we take advantage of the linear regime of 
(Q)PDs where a change in measured voltage is proportional to the movement of 
the particle within the trap. The respective start and end point of this 
regime is dependent on a multitude of parameters. But for our setup movements 
of less than \SI{100}{\nm} are still linear related. The voltage-conversion 
factor can be derived from the single-sided frequency spectrum of the trapped 
particle. The in the fluid suspended particles are so small and light that they 
undergo visible Brownian motion. Brownian motion is a random process with the 
property that over a long time the particle will be back at its initial 
position. Also while being trapped the particle is displaced by the Brownian 
motion. The frequency content of any spatial direction follows the curve of a 
low-pass filter. With the amplitude at zero frequency $A_{0}$ and the cutoff 
frequency $f_{\MR{c}}$ where $A\vert_{f=f_{\MR{c}}}=\sfrac{A_{0}}{2}$ it is 
possible to calculate the voltage-meter conversion factor for each spatial 
direction separately.

Besides the size of the trapped particle the conversion factor is dependent on 
the viscosity of the surrounding fluid. Precise knowledge of the viscosity is 
unavoidable for quantitative precise measurements. The viscosity of the fluid 
is in general a function of its temperature. While the ambient temperature is 
straightforward to measure, the measurement of the temperature within a MSAF 
device or at the focal point of the trapping laser is cumbersome. Although the 
intensity of our laser exceeds the sun's intensity by order of magnitudes, we 
show in \cref{sec:TO-temperature} that for our setup the laser induced 
temperature change and therefore the temperature induced viscosity change is 
negligible small for our experiments. However, this is not general for every OT 
the case because the temperature change does not only dependent on the laser 
power itself but even more on the laser wavelength and fluid combination, and 
therefore the absorption at this specific wavelength.

In \cref{ch:timeconstant} we first develop a new measurement method to 
visualize the transient build up of the ARF and AS and then study in 
\cref{ch:pulsing} the effects of a pulsed excitation on the respective build up 
times. Key for those experiments are, that the ARF and AS induced particle 
displacement is along orthogonal spatial direction. Moreover, the ARF 
displacement is along on of the axis of the in-plane QPD and the AS 
displacement is along the axial direction on another QPD. Because of that the 
measured displacement data is also separated and cross-talk free. We validated 
the orthogonality of the forces by force measurements with two different sized 
particles where all experimental parameters besides the particle radius kept 
the same. The drag force from AS scales with the particles radius, $\FAS\propto 
R$ and the ARF with the volume of the particle, $\FARF\propto R^{3}$.

However, without modifications on the OT we are not able to measure any of 
those effects because the timeconstant of the OT is much larger than the 
theoretical build up time of the ARF. In fact, for our set of experimental 
parameters and our device configuration $\tOT \approx \tas \gg \tarf$ with 
$\tas = \SI{1.59}{\ms}$ and $\tarf = \SI{1.4}{\us}$. We reduced $\tOT$ 
effectively to zero by reducing the laser power to almost zero such that there 
is no effective trapping by the OT. We still need some laser power because this 
beam is the cause for the measured intensity at any of the (Q)PDs.

Because of the laser power reduction, we also needed to install an optical 
shutter right before the (Q)PDs. The laser intensity is too high for the photo 
detectors. In normal trapping mode a set of filters reduces the intensity to 
less than 0.1\% of the incoming intensity. The optical filter is actuated to be 
almost completely closed such that it acts as intensity filter if the laser 
power is high (trapping) and opens as soon as the laser power is reduced for 
the measurement itself.

The laser power change occurs almost instantaneously but the shutter has a 
opening time of less than \SI{15}{\ms}. To ensure undisturbed data we wait 
\SI{25}{\ms} before starting the acoustic excitation. During this time the 
particle will start to sediment because it is not trapped stably anymore. 
However, this movement is only $\approx 0.05\,R$ along the axial direction and, 
hence, does not hinder the measurement; the in-plane movement is zero because 
no forces besides gravity are acting on the particle.

The (pulsed) acoustic excitation is then switched on for \SI{30}{\ms} only 
because a longer excitation would lead to such large displacements -- primarily 
along the ARF direction -- that the particle could not be re-trapped. For our 
studies we were not interested in the actual displacements but in the time it 
takes until the movement along any direction starts. For the actual 
displacements one would need to retrieve the voltage-meter conversion factor 
but for that the particle needs to be trapped.

Because the particle is free-floating and, hence, uncontrolled in its location 
for the first \SI{25}{\ms} we repeat the measurement multiple times per 
location and average the measured data. To further subtract the gravity induced 
movement from the data we perform the same amount of measurements with no 
acoustics at all and then subtract the averaged \emph{no-acoustic} data from 
the \emph{acoustics} data.

The data for both types of experiments (non-pulsed and pulsed) shows that the 
ARF induced displacement starts immediately whereas the AS induced displacement 
takes significantly longer. With this measured time offset between ARF and AS 
we are able to give an explanation why in 
experiments~\cite{Castro2016,Hoyos2013} a pulsed acoustic excitation suppresses 
the build up of AS whereas numerical investigations~\cite{Muller2015} with an 
ideal fluid cavity reveal a much smaller offset which could not suppress 
streaming. Although the data has the unit of \si{\volt} and not \si{\meter}, 
the slope the AS and ARF induced displacement is showing that the acceleration 
after the build up by any of two effects occurs fast and then transitions into 
constant velocity (linear slope). This can be stated because the measured 
voltage (within the linear regime of the (Q)PDs) is the displacement divided by 
the voltage-meter conversion factor. And hence, the slope of the 
voltage-over-time data equals the slope of the displacement which is the 
velocity of the particle.

The pulsed excitation measurements reveal that a pulsed excitation leads to a 
greater effect in terms of delaying the build up on the AS than it does on the 
ARF. We measure this for all three pulsing frequencies we used. The pulsed 
excitation results help to further explain the experimental shown suppression 
of AS. However, with our measurement protocol we can only measure for the first 
\SI{30}{\ms} of acoustic excitation. Any MSAF application is, however, operated 
continuously where all fields have developed fully. At the moment, we cannot 
measure these effects at this point in time.


\section{Outlook}

outlook three sizes and the visualize AS and ARF

scaling absolute value with amplitdue of spectrum -> first guess of A0 when 
fitting

pulsed excitation measurement in steady-state! synchronization needed that 
laser shuts of when there is no acoustic excitation happening

outlook for our setup specific and more open minded outlookl
 - OST devices --> near walls
 - heavy particles in viscous fluids
 - SAW
 - dual beam/own detection light source for AS measurement
 - other particle materials
 - refractive index limitation
 - longer AS measurements
 - zurich instruments laser disruption measurements
