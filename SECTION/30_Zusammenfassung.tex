\chapter*{Zusammenfassung}
\markboth{Zusammenfassung}{Zusammenfassung}
\addcontentsline{toc}{chapter}{Zusammenfassung}



Die typischen Dimensionen eines Flüssigkeitskanals für eine mikroskalierte 
akustofluidische (engl: \emph{micro-scale acoustofluidic} -- MSAF) Anwendung 
sind eine Höhe von weniger als \SI{200}{\um}, eine Breite von weniger als 
\SI{5}{\mm} und eine Länge von mehreren \si{\cm}, wobei die Länge nicht von 
grosser Wichtigkeit ist.  Die direkte Messung von physikalischen Grössen, wie 
zum Beispiel, dem Druck generiert durch die akustische Anregung ist unmöglich, 
da der Bereich der Messung klein ist.  Ausserdem sind die akustischen 
Frequenzen generell grösser als \SI{100}{\kilo\hertz}, so dass die Messfrequenz 
mindestens doppelt so hoch sein muss, wenn unter anderem auch das transiente 
Verhalten und der Aufbau des akustischen Drucks von Interesse ist und nicht nur 
dessen eingeschwungener Zustand.

Bis jetzt ist die häufigste und einfachste Methode den akustischen Druck 
innerhalb des Fluidkanals abzuschätzen, die Geschwindigkeiten von mehreren 
Objekten mit bekannten Dimensionen und Materialparametern optisch zu messen und 
dann anhand der Geschwindigkeiten auf den herrschenden akustischen Druck 
zurückzurechnen. Die Validität und Richtigkeit der Druckabschätzung hängt von 
mehreren Unsicherheiten ab. Neben den Dimensionen und den Materialparametern 
des Objekts und den Materialparametern des Fluides ist die grösste Unsicherheit 
die Validität der angewendeten Theorie für die Berechnung der akustischen 
Strahlungskraft (engl: \emph{acoustic radiation force} -- ARF). Es gibt viele 
Modelle für die Berechnung der ARF in der MSAF, welche sich grösstenteils bei 
der Annahme des physikalischen Modells für das Objekt und das Fluid 
unterscheiden. Jede Theorie für sich hat einen Parameterbereich, wo sie genauer 
ist als die anderen, weil sie zum Beispiel die Viskoelastizität des Fluides 
berücksichtigt.

In dieser Arbeit wird eine optische Falle (engl. \emph{optical trap} -- OT) 
verwendet, um zwei MSAF Phänomene zu untersuchen, bei welchen es ungeklärte 
Kontroversen gibt: 1) der transiente Aufbau der ARF und der Widerstandskraft 
aufgrund einer akustischen Strömung (engl: \emph{drag force from acoustic 
  streaming} -- AS) für eine ununterbrochene und eine gepulste akustische 
Anregung; 2) die Quantifizierung der stationären Rotationsgeschwindigkeit eines 
sphärischen Partikels, welches durch das akustisch-viskose Drehmoment (engl: 
\emph{acoustic viscous torque}) angetrieben wird und welches eine viskose 
Grenzschichtdicke (engl: \emph{viscous boundary layer thickness} -- VBL) hat, 
die so gross wie der Partikelradius selbst ist.

Bisher sind OT vor allem als Kraftsensoren für einzelne Partikel innerhalb von 
MSAF Geräten eingesetzt worden. Für die Messung der genannten Phänomene wird 
die genaue örtliche und zeitliche Auflösung der OT verwendet, wie auch die 
Möglichkeit Messungen an einzelnen Partikeln durchzuführen.

Für die Messung des Aufbaus der ARF und des AS wurde eine akustische 
Anregungsfrequenz und Messpunkte innerhalb der stehenden Druckwelle verwendet, 
bei deren Kombination die ARF und die Kräfte von AS senkrecht zueinander waren.  
Die Orthogonalität als auch die Aufteilung in ARF und AS wurde durch mehrere 
Kraftmessungen innerhalb des ganzen Fluidkanals mit unterschiedlichen 
Partikelgrössen validiert.

Die Ergebnisse der ununterbrochenen Anregung zeigten, dass der Aufbau der ARF 
sofort nach Beginn der akustischen Anregung beginnt, wohingegen AS deutlich 
langsamer ist. Der zügige Aufbau der ARF wird von der Theorie so vorhergesagt, 
während der Aufbau des AS um den Faktor 4 unterschätzt wird. Die Experimente 
mit einer gepulsten Anregung zeigten, dass abhängig von den Pulsparametern der 
Aufbau von AS weitestgehend verhindert werden kann, aber der Aufbau der ARF 
nicht im gleichen Masse beeinträchtigt wird. Daher werden kleinere Partikel zum 
grössten Teil wegen der ARF manipuliert, weil die relative Wichtigkeit von AS 
für eine gepulste Anregung schrumpft. Unsere Experimente bestärken andere 
experimentelle Ergebnisse mit einer gepulsten Anregung, welche bisher durch 
keine Theorie erklärt werden können.

Für die Messung der stationären Rotationsgeschwindigkeit wurde eine Mischung 
von Wasser und Glycerol (7 zu 3) hergestellt, dessen Viskosität so gross war, 
dass die VBL um das Partikel herum in etwa so gross wie der Partikelradius 
selbst war. Die Phasenverschiebung zwischen den beiden akustischen Feldern, die 
räumlich senkrecht zueinander standen, führte zu einem zeitgemittelten AS Feld 
innerhalb der VBL in Richtung des Umfangs des Partikels. Dieses AS Feld ist der 
Ursprung für ein antreibendes akustisch viskoses Drehmoment, welches das 
Partikel bis zu der Rotationsgeschwindigkeit beschleunigt, wo das 
Antriebsmoment gleich dem entgegengerichteten viskosen Widerstandsmoment ist. 

Eine theoretische Formel überschätzt die stationäre Rotationsgeschwindigkeit 
für diese experimentellen Parameter mit mehr als einer Grössenordnung. Dieser 
Fehler war erwartet, weil es bis jetzt noch keine Theorie gibt, welche für den 
Fall gilt, wo der Partikelradius so gross oder grösser ist wie die VBL selbst.  
Eine numerische Studie untersuchte unter anderem diesen Parameterbereich für 
sphärische Partikel und gab eine Formel für die stationäre 
Rotationsgeschwindigkeit an, welche die viskosen Effekte der VBL 
berücksichtigt. Die mit der OT gemessenen Rotationsgeschwindigkeiten 
bestätigten zwei Punkte: 1) die erwartete Ungültigkeit der vereinfachten 
Theorie für grosse Fluidviskositäten (VBL in der gleichen Grössenordnung wie 
das Partikel) und daher die notwendige Berücksichtigung der Viskosität für 
diesen Bereich; 2) die Richtigkeit der numerischen Ergebnisse.

