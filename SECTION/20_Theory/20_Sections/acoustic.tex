For the derivation and the discussion of the relevant forces coming from the 
scattering of the incident acoustic field on to the object surface, we will 
consider a viscous fluid and a rigid spherical object. We choose this 
combination of fluid and object material because the viscous property of the 
fluid causes the occurrence of the so-called acoustic streaming.

We will follow closely the derivations given by \cname{Doinikov1994Rigid} who 
solves this specific problem first \footnote{We will cite publications in this 
chapter only if they are different to \cname{Doinikov1994Rigid}. Per default, 
all information for this chapter if not stated otherwise is taken from there.}.

\section{Governing Equations}

Ultimately we want to compute the force acting on the particle. This force is 
called acoustic radiation force (ARF) and is defined as the timeaveraged 
surface integral of the stress tensor over the time varying surface $S(t)$ of 
the particle (see \cref{fig:TA-deformed_circle})
\begin{equation}
  F^{\MR{rad}}_{i} \coloneqq
  \timeaverage{\int_{S(t)}\,\stress\,\normal(t)\,\dd{S(t)}}
  \label{eq:TA-def-rad}
\end{equation}
where $\normal(t)$ is the time dependent outward facing normal, repetitive 
indices imply Einstein summation notation, and the angle bracket define the 
timeaverage over one period $T$
\begin{equation}
  \timeaverage{\Box(t)} = \frac{1}{T}\,\int\limits_{0}^{T}\Box(t)\dd{t}.
  \label{eq:TA-timeaverage}
\end{equation}

\begin{figure}[tbp]
  \centering
  % \tikzsetnextfilename{deformed_circle}

\pgfmathsetmacro{\R}{2.25}     % COS
\begin{tikzpicture}
  \fill[colFluid,opacity=0.3] (-4.6,-2.6) rectangle (4.6, 2.6);
  \pgfmathsetseed{1} % choose a number which give a good shape to your circle
  \path[draw, fill=white,color=white] plot[domain=0:350,smooth cycle] 
  (\x:2+rnd*0.5);

  % axes
  \draw[ultra thin] (-2.5, 0) -- (2.5, 0);
  \draw[ultra thin] (0, -2.5) -- (0, 2.5);
  % undistorted sphere
  \draw[ultra thin, dashed] (0, 0) circle (\R);
  % distrorted sphere
  \pgfmathsetseed{1} % choose a number which give a good shape to your circle
  \path[draw, shading=ball, ball color=colParticle, fill opacity=0.3, thick, 
  name path=curve] plot[domain=0:350,smooth cycle] (\x:2+rnd*0.5);

  \draw[ultra thin, dashed, ->] (0:0) -- node[below, midway] {$R$} (190:\R);

  \draw[-] (140:\R) to[out=50, in=0] ++(-.4*\R,.25*\R) node[left] 
  {$\zeOrder{S}$};

  \path[very thick, name path=line] (0:0) -- (60:3.5);
  \draw [name intersections={of=curve and line, by=x}];
  \draw[-] (x) to[out=-35, in=180] ++(.6*\R,-.25*\R) node[right] {$S(t)$};

  \path[very thick, name path=line] (0:0) -- (10:3.5);
  \draw [name intersections={of=curve and line, by=x}];
  \draw[thick, |->] (x) to (13:3.5) node[right] {$\normal(t)$};
\end{tikzpicture}

  \includegraphics[]{Plots/cache/deformed_circle.pdf}
  \caption{Sketch of deformed time dependent particle surface $S(t)$ and 
  particle surface at rest $\zeOrder{S}$.}
  \label{fig:TA-deformed_circle}
\end{figure}

For a viscous fluid the stress tensor is given as
\begin{equation}
  \stress = \muef\left[ \pdv{\vel_{i}}{x_{j}} + \pdv{\vel_{j}}{x_{i}} \right] + 
  \left[ \mu_{\MR{B}} - \frac{2}{3}\,\muef \right] \pdv{\vel_{k}}{x_{k}} 
  \,\delta_{ij} - \pre\,\delta_{ij}
  \label{eq:TA-stress}
\end{equation}
where $\muef$ is the fluid dynamic viscosity, $\mu_{\MR{B}}$ the fluid bulk 
viscosity, $\vel_{i}$ the $i$-th component of the fluid velocity, $\pre$ the 
acoustic pressure, and $\delta_{ij}$ the Kronecker delta.

For the stress tensor the fluid velocity $\vel_{i}$ and the acoustic pressure 
$\pre$ must be computed. These four unknowns (three as $\vel_{i}$ and one as 
$\pre$) are linked by the scalar equation for the mass conservation
\begin{equation}
  \pdv{\density}{t} + \pdv{x_{i}}(\density\,\vel_{i}) = 0
  \label{eq:TA-massconservation}
\end{equation}
where $\density$ is the density. This equation introduces the additional 
unknown density $\density$. With the assumption of a barotropic fluid
\begin{equation}
  \pre \coloneqq \pre\left( \density \right)
  \label{eq:TA-state}
\end{equation}
the acoustic pressure $\pre$ is defined solely by the density $\density$. The 
missing three equations are given by the Navier-Stokes equations for a viscous 
fluid
\begin{equation}
  \pdv{t}\left( \density\,\vel_{i} \right) = \pdv{x_{j}} \left( \stress - 
  \density\,\vel_{i}\,\vel_{j} \right).
    \label{eq:TA-navierconservative}
\end{equation}

With the assumption, that the density $\density$ is constant in time and space 
($\pdv{\density}{t} = 0$, $\pdv{\density}{x_{i}} =0$)
\cref{eq:TA-navierconservative} can be rearranged to
\begin{equation}
  \density\left( \pdv{\vel_{i}}{t} + \vel_{i}\pdv{\vel_{j}}{x_{j}}\right) = 
  \pdv{\stress}{x_{j}}.
  \label{eq:TA-navier}
\end{equation}

This set of equations (\cref{eq:TA-massconservation,eq:TA-state} and 
\cref{eq:TA-navierconservative} or \cref{eq:TA-navier})solve the problem of a 
free floating rigid particle in a viscous fluid but are impossible to solve 
analytically without further simplification. Most common is the expansion of 
the unknowns into a converging -- not guaranteed~\cite{Baasch2020} -- series of 
increasing order
\begin{subequations}
\begin{alignat}{4}
  \pre & = \zeOrder{\pre} & + \stOrder{\pre} & + \ndOrder{\pre} & + \ldots 
  \label{eq:TA-per-pre}\\
  \density & = \zeOrder{\density} & + \stOrder{\density} & + \ndOrder{\density} & + \ldots 
  \label{eq:TA-per-rho}\\
  \stress & = \zeOrder{\stress} & + \stOrder{\stress} & + \ndOrder{\stress} & + 
  \ldots \label{eq:TA-per-stress} \\
  \vel_{i} & =  & \phantom{+} \stOrder{\vel_{i}} & + \ndOrder{\vel_{i}} & + 
  \ldots \label{eq:TA-per-vel}.
\end{alignat}
\end{subequations}
This method of expanding the variables is often called Pertubation expansion. 
The superscript $\Box^{(i)}$ indicates the $i$-th order of the respective 
parameter. The zero order fields $\zeOrder{\Box}$ are defined to be constant in 
space and time. Additionally, we assume no constant flow of the fluid and, 
hence, the zeroth order of the velocity $\zeOrder{\vel_{i}} = 0$.

This series expansion is useful because it is known that the ARF occurs due to 
second order effects. With that is possible to approximate 
\cref{eq:TA-def-rad} as
\begin{equation}
  F^{\MR{rad}}_{i} \approx
  \timeaverage{\int_{S(t)}\,\stOrder{\stress}\,\normal(t)\,\dd{S(t)}} +
  \int_{\zeOrder{S}}\,\timeaverage{\ndOrder{\stress}}\,
  \normal\,\dd{\zeOrder{S}}
  \label{eq:TA-def-approx-ARF}
\end{equation}
where $\zeOrder{S}$ is the particle surface at rest (see 
\cref{fig:TA-deformed_circle}).

It is possible to solve first for the first order fluid field, which is also 
called acoustic scattering, and then use the first order solution to compute 
the second order fields. In the following two section, we will solve those 
sequentially.

\section{First Order Solution\label{sec:TA-firstorder}}

By applying the Pertubation expansion to \cref{eq:TA-massconservation} and by 
only collection terms that are equal or less than first order the mass 
conservation is
\begin{equation}
  \pdv{\stOrder{\density}}{t} + 
  \zeOrder{\density}\,\pdv{\stOrder{\vel_{i}}}{x_{i}} = 0.
  \label{eq:TA-st-massconservation}
\end{equation}

With the same procedure the general Navier Stockes equation 
(\cref{eq:TA-navier}) are
\begin{equation}
  \zeOrder{\density} \pdv{\stOrder{\vel_{i}}}{t} = 
  \pdv{\stOrder{\stress}}{x_{j}}.
  \label{eq:TA-st-navier}
\end{equation}
with the stress $\stOrder{\stress}$ being
\begin{equation}
  \stOrder{\stress} = \muef\left[ \pdv{\stOrder{\vel_{i}}}{x_{j}} + 
  \pdv{\stOrder{\vel_{j}}}{x_{i}} \right] + \left[ \mu_{\MR{B}} - 
  \frac{2}{3}\,\muef \right] \pdv{\stOrder{\vel_{k}}}{x_{k}} \,\delta_{ij} - 
  \stOrder{\pre}\,\delta_{ij}.
  \label{eq:TA-st-stress}
\end{equation}

The acoustic pressure of first order is
\begin{equation}
  \stOrder{\pre} = c^{2}\,\stOrder{\density}
  \label{eq:TA-st-pre}
\end{equation}
with $c$ being the speed of sound.

The first order velocity field $\stOrder{\vel_{i}}$ can be approximated as
\begin{equation}
  \stOrder{\vel}_{i} = \pdv{\stOrder{\phi}}{x_{i}} + 
  \epsilon_{kli}\,\pdv{\stOrder{\psi}_{l}}{x_{k}}
  \label{eq:TA-st-vel}
\end{equation}
where $\phi$ is the scalar velocity potential, $\psi_{i}$ the vector 
velocity potential, and $\psi_{kli}$ the Levi-Civita symbol to express the curl 
of the vector velocity potential in index notation. Furthermore, the scalar can 
be split up into the contributions from the incident velocity potential 
$\phi_{\MR{in}}$ and the scattered velocity potential $\phi_{\MR{sc}}$
\begin{equation}
  \stOrder{\phi} = \phi_{\MR{in}} + \phi_{\MR{sc}}.
  \label{eq:TA-scalar-potential}
\end{equation}

For the derivations we assume an axisymmetric incoming field and with the 
wavevector $\vb{k}$ pointing along the $\ez$ direction (see 
\cref{fig:TA-plane_wave}). Additionally, we use spherical coordinates $r, 
\theta$, and $\varpi$ with the origin at the center of the particle in its 
equilibrium configuration (see \cref{fig:TA-coordinate}).

\begin{figure}[tbp]
  \centering
  % \tikzsetnextfilename{plane_wave}
%----------------------
%Plot display orientation
\pgfmathsetmacro{\thetaC}{80}
\pgfmathsetmacro{\phiC}{100}
\tdplotsetmaincoords{\thetaC}{\phiC}
%----------------------
%Parameters 
\pgfmathsetmacro{\R}{4}     % COS
\pgfmathsetmacro{\k}{2.7}   % distance k vector
\pgfmathsetmacro{\rvec}{2}  % Particle's radius
\pgfmathsetmacro{\l}{4}   % width of planar wave
\pgfmathsetmacro{\endl}{7}  % Azimuthal dashed line

\begin{tikzpicture}[tdplot_main_coords]
\coordinate (O) at (0,0,0);
\coordinate (X) at (\R,0,0);
\coordinate (Y) at (0,0,\R);
\coordinate (Z) at (0,\R,0);
\tdplotsetcoord{P}{2*\R}{-90}{0}

%----------------------
% Particle
\shadedraw[tdplot_screen_coords,particleBall, opacity=.4](O) circle (\rvec);
% Edges for labels
\draw[-](0,.3*\R,.2*\R) to[out=40, in=180] ++(0,.4*\R,.45*\R)
    node[right] {Particle};
%-----------------------
% Radial
    \draw[thick,dashed] (O) -- (P);
    %\tdplotdrawarc[->]{(P)}{0.7}{0}{90}{anchor=north}{$\theta$}
\pic [draw, ->, "$\theta$", angle eccentricity=1.2,angle radius=1cm] {angle = Z--O--P};
%-----------------------
% Equator
\draw[dashed] (\rvec,0,0) arc (0:360:\rvec);
% Draw the arc which center is (O) from \phiC to \phiC-180 deg
\draw[thick] ([shift=(\phiC:\rvec)]O) arc (\phiC:\phiC-180:\rvec);
%-----------------------
% Cartesian COS
\begin{scope}[->,thick]
    \draw (O) -- (X)
        node[anchor=north east]{$\ey$};
    \draw (O) -- (Y)
        node[anchor=north west]{$\ex$};
  \draw (O) -- (Z)
    node[anchor=south east]{$\ez$};
\end{scope}
%----------------------
%----------------------
% Axis of symmetry
\draw[dashed] ($(O)+(0,\endl-1.5,0)$) -- (0,-\endl+1,0);
% Edges for labels
\draw[-] ($(O)+(0,\endl-2,0)$) to[out=120, in=-90] ++($(O)+(0,\endl-8,.25*\R)$)
    node[above]{Axis of symmetry};
%----------------------
% Incoming wave vector
\coordinate (P) at (0,-\k,0);
\draw[->,thick,color=red] (0,-1.5*\k,0) -- (P) 
node[above,pos=.6]{$\vb{k}=k\vb{e}_{\MR{k}}$};
 %----------------------
% Incoming wave fronts
\foreach \z in {-7,-6,-5}
\draw [red, fill=red!20, opacity=.4] plot (\l,\z,1) -- (-\l,\z,1) --(-\l,\z,-1) -- (\l,\z,-1) -- cycle;
\end{tikzpicture}



  \includegraphics[]{Plots/cache/plane_wave.pdf}
  \caption{Schematic of light path from laser head to quadrant photo detectors. 
  In the top half sketch red shading and the black lines the laser intensity 
profile qualitatively. In the bottom half just the outer most laser rays are 
depicted.}
  \label{fig:TA-plane_wave}
\end{figure}

The scaler incident velocity potential can be for a general case represented as
\begin{equation}
  \phi_{\MR{in}} = \exp\left( -\iu\omega t \right) \sum\limits_{n=0}^{\infty} 
  A_{n}\,\bessel{n}{kr}\,\legendre{n}{\cos\,\theta}
  \label{eq:TA-incident-field}
\end{equation}
where $\omega$ is the angular velocity of the incoming acoustic wave, $t$ the 
time, $A_{n}$ the amplitude of the incoming wave, $\bessel{n}{\Box}$ the $n$-th 
order spherical Bessel function of first kind, $\legendre{n}{\Box}$ the 
Legendre polynomial of order $n$, and $k$ the wavenumber
\begin{equation}
  k = 
  \frac{\omega}{c}\,\frac{1}{\sqrt{1-\frac{\iu\,\omega}{\zeOrder{\density}\,c^{2}} 
  \left( \mu_{\MR{B}} + \frac{4}{3}\,\muef \right)}}.
  \label{eq:TA-wavenumber}
\end{equation}

\cref{eq:TA-incident-field} is given by the type of incident wave. 
$\phi_{\MR{sc}}$ and $\psi_{i}$ can be separated into two independent 
differential equations. For $\phi_{\MR{sc}}$ we take first 
\cref{eq:TA-scalar-potential} and plug it into the mass conservation 
\cref{eq:TA-st-massconservation}. With the property that the divergence of the 
curl ($\pdv{\epsilon_{kli}\,\Box_{i}}{x_{i}} = 0$) is equal to zero and the 
definition of the first order pressure \cref{eq:TA-st-pre} one can eliminate 
$\psi_{i}$
\begin{equation}
  \pdv{\stOrder{\pre}}{t} = \zeOrder{\density}\,c^{2}\,\pdv[2]{\phi}{x_{i}}.
  \label{eq:TA-st-tmp-step}
\end{equation}
The next step is to take \cref{eq:TA-st-tmp-step} and plug it into the partial 
time derivative ($\pdv{\Box}{t}$) of \cref{eq:TA-st-navier}. This results in a 
wave equation which can be further simplified because we assume time harmonic 
acoustics fields ($\vel_{i}\propto\exp -\iu\omega t$) that have the property 
$\pdv{\vel_{i}}{t}\propto -\iu\omega$ and
$\pdv[2]{\vel_{i}}{t}\propto \omega^{2}$
\begin{equation}
  \label{eq:TA-st-nd-tmp-step}
\begin{split}
  \zeOrder{\density}\,\pdv[2]{\vel_{i}}{t} = & 
  \zeOrder{\density}\,c^{2}\pdv{x_{i}}\left( \pdv[2]{\phi}{x_{j}} \right) \\
  + & \mu\,\pdv[2]{x_{i}}\left( \pdv{\vel_{k}}{t} \right) \\
    + & \left(\mu_{\MR{B}} + \frac{\mu}{3}\right)\pdv{t}
    \left( \pdv{x_{i}}\left( \pdv[2]{\phi}{x_{j}} \right) \right).
\end{split}
\end{equation}

The last step is to take the divergence of \cref{eq:TA-st-nd-tmp-step} to 
eliminate $\psi_{i}$. Finally, the equations for $\phi$ is given as a Helmholtz 
equation
\begin{equation}
  \pdv[2]{\phi}{x_{i}} + k^{2}\,\phi = 0.
  \label{eq:TA-st-phi}
\end{equation}

\begin{figure}[tbp]
  \centering
  % \tikzsetnextfilename{coordinate}
\tdplotsetmaincoords{60}{110}
%
% Parameters for vector
\pgfmathsetmacro{\rvec}{1}
\pgfmathsetmacro{\thetavec}{30}
\pgfmathsetmacro{\phivec}{60}
%
%Parameters for COS
\pgfmathsetmacro{\R}{1.2}
\pgfmathsetmacro{\r}{1}
%
\begin{tikzpicture}[scale=3,tdplot_main_coords]
    \coordinate (O) at (0,0,0);
    \coordinate (X) at (\R,0,0);
    \coordinate (Y) at (0,\R,0);
    \coordinate (Z) at (0,0,\R);

    % Cartesian COS
    \draw[thick,->] (O) -- (X)
        %node[midway,sloped,above,color=note]{$x$}
        node[anchor=north east]{$\ex$};
    \draw[thick,->] (O) -- (Y)
        %node[pos=.6,sloped,above,color=note]{$y$}
        node[anchor=north west]{$\ey$};
  \draw[thick,->] (O) -- (Z)
    %node[midway,sloped,above,color=note]{$z$}
    node[anchor=south]{$\ez$};

  %Shade
  \shade[particle,opacity=0.3] (\r,0) arc (0:90:\r) {[x={(0,0,\r)}] arc (90:0:\r)} {[y={(0,0,\r)}] arc (90:0:\r)};

    % Spherical coordinates
    \tdplotdrawarc[->]{(O)}{0.2}{0}{\phivec}{anchor=north}{$\varphi$}
    \tdplotsetthetaplanecoords{\phivec}
    \tdplotdrawarc[->,tdplot_rotated_coords]{(0,0,0)}{.5}{0}%
        {\thetavec}{anchor=south west}{$\theta$}

     %Hilfslinien
  \draw[dashed,tdplot_rotated_coords] (\rvec,0,0) arc (0:90:\rvec)node(Rxy){};
    %\draw[dashed] (\rvec,0,0) arc (0:90:\rvec);
    \draw[dashed] (O) -- (Rxy);

  % Vector
    \tdplotsetcoord{P}{\rvec}{\thetavec}{\phivec}
    \draw[->,thick,color=red] (O) -- (P)
    % label vector
    node[above,left,color=red] at (P) {$\vb{r}$}
    %label coordinate r
        node[midway,below,sloped]{$r$};
    \draw[dashed, color=red] (O) -- (Pxy);
    \draw[dashed, color=red] (P) -- (Pxy);

   %  New coordinate origin
    \tdplotsetrotatedcoords{\phivec}{\thetavec}{0}
    \tdplotsetrotatedcoordsorigin{(P)}
    % Spherical COS
    \draw[thick,tdplot_rotated_coords,->] (P)
        -- (.5,0,0) node[anchor=north west]{$\vb{e}_\theta$};
    \draw[thick,tdplot_rotated_coords,->] (0,0,0)
        -- (0,.5,0) node[anchor=west]{$\vb{e}_\varphi$};
    \draw[thick,tdplot_rotated_coords,->] (0,0,0)
        -- (0,0,.5) node[anchor=south]{$\vb{e}_r$};
\end{tikzpicture}


  \includegraphics[]{Plots/cache/coordinate.pdf}
  \caption{Sketch of Cartesian and spherical coordinate system}
  \label{fig:TA-coordinate}
\end{figure}

Instead of taking the divergence of \cref{eq:TA-st-nd-tmp-step} but taking the 
curl one gets a Helmholtz equation for $\psi_{i}$
\begin{equation}
  \pdv[2]{\psi_{j}}{x_{i}} + k_{\MR{v}}^{2}\,\psi_{i} = 0
  \label{eq:TA-st-psi}
\end{equation}
by using $\epsilon_{kli}\,\pdv{x_{k}}\left( \pdv{\Box}{x_{l}} \right) = 0$ 
(curl of the gradient of a scalar). In \cref{eq:TA-st-psi} $k_{\MR{v}}$ is 
called the viscous wavenumber and is defined as
\begin{equation}
  k_{\MR{v}} = \frac{1 + \iu}{\delta} = \left( 1 + \iu \right)\, 
  \sqrt{\frac{\omega\,\mu}{\zeOrder{\density}}}
  \label{eq:TA-st-visc-wavenumber}
\end{equation}
where $\delta$ is also known as \emph{viscous penetration depth}.

With the requirement that the solutions to \cref{eq:TA-st-phi,eq:TA-st-psi} 
satisfy Sommerfeldt's radiation condition one finds
\begin{equation}
  \phi_{\MR{sc}} = \exp\left( -\iu\omega t \right)\,
  \sum\limits^{\infty}_{n=0} 
  \alpha_{n}\,A_{n}\,\hankel{n}{k\,r}\,\legendre{n}{\cos\theta}
  \label{eq:TA-st-sol-phi}
\end{equation}
and
\begin{equation}
  \vb{\psi}^{(1)} = \exp\left( -\iu\omega t \right)\,\vb{e}_{\varphi}\,
  \sum\limits^{\infty}_{n=1} 
  \beta_{n}\,A_{n}\,\hankel{n}{k_{\MR{v}}\,r}\,P^{1}_{n}(\cos\theta)
  \label{eq:TA-st-sol-psi}
\end{equation}
where $\alpha_{n}$ and $\beta_{n}$ are constants defined by the boundary 
conditions, $\vb{e}_{\varphi}$ the unit vector of the spherical coordinate 
system (see \cref{fig:TA-coordinate}), $\hankel{n}{\Box}$ the spherical Hankel 
functions of the first kind, and $P^{1}_{n}\left( \Box \right)$ the associated 
Legendre polynomial. The sum of \cref{eq:TA-st-sol-psi} starts with $n=1$ 
because $\beta_{0}=0$.

The boundary condition for the combination of a rigid spherical particle and a 
viscous fluid is that the velocity of particle $\vel_{\MR{p}}$ surface must 
match the fluid velocity
\begin{equation}
  \vel_{i,\MR{fluid}} = \vel_{i,\MR{p}}\quad\text{at}\quad r = R.
  \label{eq:TA-st-BC}
\end{equation}

Using Newtown's second law and the first order stress tensor
\begin{equation}
  \frac{4}{3}\,\pi 
  R^{3}\,\density_{\MR{p}}\,\dv{\vel_{i,\MR{p}}}{t} = 
  \int\limits_{\zeOrder{S}}\stOrder{\stress}\dd{\zeOrder{S}}
  \label{eq:TA-st-newton}
\end{equation}
where $\density_{\MR{p}}$ is the particle density, one finds the particle 
velocity along the direction of the wave vector $\vb{e}_{k}$ as
\begin{equation}
  \vel_{\MR{p}} = \frac{\zeOrder{\density}A_{1}k}{x}\,
  \left[
    \bessel{1}{x} + \alpha_{1}\hankel{1}{x} + 2\beta_{1}\hankel{1}{\xv}
  \right]
  \,\exp\left( -\iu\omega t \right)
  \label{eq:TA-st-particle-vel}
\end{equation}
with $x=kR$ and $\xv=k_{\MR{v}}R$.

In order to calculate the unknown constants $\alpha_{n}$ and $\beta_{n}$ one 
needs to enforce the boundary condition of \cref{eq:TA-st-BC} for every $n$ 
separately. Besides for $n=1$, $\vel_{\MR{p}}$ is always zero; for $n=1$, 
$\vel_{\MR{p}}$ is given by \cref{eq:TA-st-particle-vel}.

For $n=0$ and $n=1$ the constant are
\begin{subequations}
\begin{align}
  \alpha_{0} & = -\frac{\bessel{1}{x}}{\hankel{1}{x}} \\
  \beta_{0} & = 0 \\
  \alpha_{1} & = - \frac{1}{\chi_{4}} \left[ \chi_{1}\,\chi_{3} + 
  2(1-\trho^{2})\bessel{1}{x}\hankel{1}{\xv} \right]
    \label{eq:TA-alpha-one}\\
    \beta_{1} &= - \frac{1-\trho}{\chi_{4}} \left[ \chi_{1}\,\hankel{1}{x} - 
    \chi_{2}\,\bessel{1}{x} \right]
    \label{eq:TA-beta-one}
\end{align}
\end{subequations}
where $\trho = \sfrac{\rhof}{\density_{\MR{p}}}$ and
\begin{subequations}
\begin{align}
\chi_{1} & = \trho\,\bessel{1}{x} - \trho\,\dbessel{1}{x}
\label{eq:TA-chi-one} \\
\chi_{2} & = \trho\,\hankel{1}{x} - \trho\,\dhankel{1}{x}
\label{eq:TA-chi-two} \\
\chi_{3} & = \left( 1 + 2 \trho\right)\hankel{1}{\xv} + 
\xv\,\dhankel{1}{x}
\label{eq:TA-chi-three} \\
\chi_{4} & = \chi_{2}\,\chi_{3} + 2\left( 1 - \trho^{2} \right) 
\hankel{1}{x}\,\hankel{1}{\xv}
\label{eq:TA-chi-four}
\end{align}
\end{subequations}
where the prime $\Box^{\prime}$ indicates the differentiation, e.g. 
$\dbessel{n}{\xv}=\dv{\bessel{n}{\xv}}{\xv}$.

For $n > 1$, one finds the coefficient $\alpha_{n}$ and $\beta_{n}$ to be
\begin{subequations}
\begin{align}
  \alpha_{n} & =
  - \frac{1}{\xi_{n}} \left[
    n(n+1)\bessel{n}{x}\hankel{n}{\xv} - x \gamma_{n}\dbessel{n}{x}
  \right]
  \label{eq:TA-alpha-n}\\
    \beta_{n} &= - \frac{1}{\xi_{n}} \left[
      x\dbessel{n}{x}\hankel{n}{x} - x\bessel{n}{x}\dhankel{n}{x}
    \right]
    \label{eq:TA-beta-n}
\end{align}
\end{subequations}
with
\begin{subequations}
\begin{align}
  \gamma_{n} & =
  \hankel{n}{\xv}+\xv\dhankel{n}{\xv}
  \label{eq:TA-gamma-n}\\
  \xi_{n} &= x\dhankel{n}{x}\gamma_{n} - n(n+1)\hankel{n}{x}\hankel{n}{\xv}.
    \label{eq:TA-xi-n}
\end{align}
\end{subequations}

Note here, that 
\cref{eq:TA-alpha-one,eq:TA-beta-one,eq:TA-chi-three,eq:TA-chi-four,eq:TA-beta-n} 
are different to the ones in \cite{Doinikov1994Rigid}; Equations (3.14), 
(3.15), (3.18), (3.19), and (3.21) respectively. We found these typos in the 
original publication and discussed it more extensively in \cname{Fankhauser2022}.

\section{Second Order Solution\label{sec:TA-secondorder}}

\begin{equation}
  \pdv{\ndOrder{\density}}{t} + \pdv{x_{i}}\left(\stOrder{\density}\,\stOrder{\vel_{i}} 
  + \zeOrder{\density}\,\ndOrder{\vel_{i}} \right) = 0
  \label{eq:TA-nd-massconservatio}
\end{equation}

\begin{equation}
  \stOrder{\density}\,\pdv{\stOrder{\vel_{i}}}{t} + \zeOrder{\density} 
  \pdv{\ndOrder{\vel_{i}}}{t} + 
  \zeOrder{\density}\,\stOrder{\vel_{j}}\,\pdv{\stOrder{\vel_{j}}}{x_{j}} = 
  \pdv{\ndOrder{\stress}}{x_{j}}
  \label{eq:TA-nd-navier}
\end{equation}

\begin{equation}
  \ndOrder{\stress} = \muef\left[ \pdv{\ndOrder{\vel_{i}}}{x_{j}} + 
  \pdv{\ndOrder{\vel_{j}}}{x_{i}} \right] + \left[ \mu_{\MR{B}} - 
  \frac{2}{3}\,\muef \right] \pdv{\ndOrder{\vel_{k}}}{x_{k}} \,\delta_{ij} - 
  \ndOrder{\pre}\,\delta_{ij}
  \label{eq:TA-nd-stress}
\end{equation}





\section{Time-averaged Second Order Solution\label{Th-avg-nd-order}}

\begin{equation}
  \timeaverage{\pdv{t}\,f(\vb{x};t)} = \lim\limits_{\tau \rightarrow \infty}{\,
    \left[ \frac{1}{\tau}\,\left( f(\vb{x}; \tau) - f(\vb{x}; 0) \right) 
  \right]} = 0
  \label{eq:TA-avg-property}
\end{equation}

timeaverage of first and second order fields is zero --> but not the product of 
them BAASCH2020

\begin{equation}
  \pdv{x_{i}}\left( \timeaverage{\ndOrder{\vel_{i}}} \right) = 
  -\frac{1}{\zeOrder{\density}} \pdv{x_{i}}\left(\timeaverage{ 
  \stOrder{\density}\,\stOrder{\vel_{i}} }\right)
  \label{eq:TA-avg-massconservatio}
\end{equation}

\begin{equation}
  \timeaverage{\stOrder{\density} \pdv{\stOrder{\vel_{i}}}{t}} + 
  \zeOrder{\density}\,\timeaverage{\stOrder{\vel_{j}}\,\pdv{\stOrder{\vel_{j}}}{x_{j}}} 
  = \pdv{x_{j}}\left( \timeaverage{\ndOrder{\stress}} \right)
  \label{eq:TA-avg-navier}
\end{equation}

first term remains because it is a product of first order fields

timeaverage of a constant function is the function itself

\section{Acoustic Radiation Force\label{sec:TA-ARF}}



