For the derivation and the discussion of the relevant forces coming from the 
scattering of an axisymmetric incident acoustic field at the object surface, we 
will consider a viscous fluid and a rigid spherical particle. The spherical 
object is freely floating in the fluid and far away from any boundaries. We 
choose this combination of fluid and object material because the viscous 
property of the fluid causes the occurrence of the so-called acoustic streaming 
(AS).

We will follow closely the derivations given by \cname{Doinikov1994Rigid} who 
solves this specific problem first \footnote{We will cite publications in this 
chapter only if they are different to \cname{Doinikov1994Rigid}. Per default, 
all information for this chapter if not stated otherwise is taken from there.}.

\section{Governing Equations}

Ultimately we want to compute the force acting on the particle. This force is 
called acoustic radiation force (ARF) and is defined as the time-averaged 
surface integral of the stress tensor over the time varying surface $S(t)$ of 
the particle (see \cref{fig:TA-deformed_circle})
\begin{equation}
  F^{\MR{rad}}_{i} \coloneqq
  \timeaverage{\int_{S(t)}\,\stress\,\normal(t)\,\dd{S(t)}}
  \label{eq:TA-def-rad}
\end{equation}
where $\normal(t)$ is the time dependent outward facing normal, repetitive 
indices imply Einstein summation notation, and the angle bracket define the 
time-average over one period $T_{\MR{ex}}$ of the fundamental excitation 
frequency of the acoustic field
\begin{equation}
  \timeaverage{\Box(t)} = 
  \frac{1}{T_{\MR{ex}}}\,\int\limits_{0}^{T_{\MR{ex}}}\Box(t)\dd{t}.
  \label{eq:TA-time-average}
\end{equation}

\begin{figure}[tbp]
  \centering
  % \tikzsetnextfilename{deformed_circle}

\pgfmathsetmacro{\R}{2.25}     % COS
\begin{tikzpicture}
  \fill[colFluid,opacity=0.3] (-4.6,-2.6) rectangle (4.6, 2.6);
  \pgfmathsetseed{1} % choose a number which give a good shape to your circle
  \path[draw, fill=white,color=white] plot[domain=0:350,smooth cycle] 
  (\x:2+rnd*0.5);

  % axes
  \draw[ultra thin] (-2.5, 0) -- (2.5, 0);
  \draw[ultra thin] (0, -2.5) -- (0, 2.5);
  % undistorted sphere
  \draw[ultra thin, dashed] (0, 0) circle (\R);
  % distrorted sphere
  \pgfmathsetseed{1} % choose a number which give a good shape to your circle
  \path[draw, shading=ball, ball color=colParticle, fill opacity=0.3, thick, 
  name path=curve] plot[domain=0:350,smooth cycle] (\x:2+rnd*0.5);

  \draw[ultra thin, dashed, ->] (0:0) -- node[below, midway] {$R$} (190:\R);

  \draw[-] (140:\R) to[out=50, in=0] ++(-.4*\R,.25*\R) node[left] 
  {$\zeOrder{S}$};

  \path[very thick, name path=line] (0:0) -- (60:3.5);
  \draw [name intersections={of=curve and line, by=x}];
  \draw[-] (x) to[out=-35, in=180] ++(.6*\R,-.25*\R) node[right] {$S(t)$};

  \path[very thick, name path=line] (0:0) -- (10:3.5);
  \draw [name intersections={of=curve and line, by=x}];
  \draw[thick, |->] (x) to (13:3.5) node[right] {$\normal(t)$};
\end{tikzpicture}

  \includegraphics[]{External/deformed_circle.pdf}
  \caption{Sketch of deformed time dependent particle surface $S(t)$, particle 
  surface at rest $\zeOrder{S}$, and time-varying surface normal $\normal(t)$.}
  \label{fig:TA-deformed_circle}
\end{figure}

For a viscous fluid the stress tensor is given as
\begin{equation}
  \stress = \muef\left[ \pdv{\vel_{i}}{x_{j}} + \pdv{\vel_{j}}{x_{i}} \right] + 
  \left[ \mu_{\MR{B}} - \frac{2}{3}\,\muef \right] \pdv{\vel_{k}}{x_{k}} 
  \,\delta_{ij} - \pre\,\delta_{ij}
  \label{eq:TA-stress}
\end{equation}
where $\muef$ is the fluid dynamic viscosity, $\mu_{\MR{B}}$ the fluid bulk 
viscosity, $\vel_{i}$ the $i$-th component of the fluid velocity, $\pre$ the 
pressure, and $\delta_{ij}$ the Kronecker delta.

For the stress tensor the fluid velocity $\vel_{i}$ and the acoustic pressure 
$\pre$ must be computed. These four unknowns (three as $\vel_{i}$ and one as 
$\pre$) are linked by the scalar equation for the mass conservation
\begin{equation}
  \pdv{\density}{t} + \pdv{x_{i}}(\density\,\vel_{i}) = 0
  \label{eq:TA-massconservation}
\end{equation}
where $\density$ is the density. This equation introduces the additional 
unknown density $\density$. With the assumption of a barotropic fluid
\begin{equation}
  \pre \coloneqq \pre\left( \density \right)
  \label{eq:TA-state}
\end{equation}
the acoustic pressure $\pre$ is defined solely by the density $\density$. The 
missing three equations are given by the Navier-Stokes equations for a viscous 
fluid
\begin{equation}
  \pdv{t}\left( \density\,\vel_{i} \right) = \pdv{x_{j}} \left( \stress - 
  \density\,\vel_{i}\,\vel_{j} \right).
    \label{eq:TA-navierconservative}
\end{equation}

% With the assumption, that the density $\density$ is constant in time and space 
% ($\pdv{\density}{t} = 0$, $\pdv{\density}{x_{i}} =0$)
% \cref{eq:TA-navierconservative} can be rearranged to
% \begin{equation}
  % \density\left( \pdv{\vel_{i}}{t} + \vel_{i}\pdv{\vel_{j}}{x_{j}}\right) = 
  % \pdv{\stress}{x_{j}}.
  % \label{eq:TA-navier}
% \end{equation}

This set of equations (\cref{eq:TA-massconservation,eq:TA-state} and 
\cref{eq:TA-navierconservative}) solve the problem of a free floating rigid 
particle in a viscous fluid but are impossible to solve analytically without 
further simplification. Most common is the expansion of the unknowns into a 
converging 
% -- not guaranteed~\cite{Baasch2020} --
series of increasing order
\begin{subequations}
\begin{alignat}{4}
  \pre & = \zeOrder{\pre} & + \stOrder{\pre} & + \ndOrder{\pre} & + \ldots 
  \label{eq:TA-per-pre}\\
  \density & = \zeOrder{\density} & + \stOrder{\density} & + \ndOrder{\density} & + \ldots 
  \label{eq:TA-per-rho}\\
  \stress & = \zeOrder{\stress} & + \stOrder{\stress} & + \ndOrder{\stress} & + 
  \ldots \label{eq:TA-per-stress} \\
  \vel_{i} & =  & \phantom{+} \stOrder{\vel_{i}} & + \ndOrder{\vel_{i}} & + 
  \ldots \label{eq:TA-per-vel}.
\end{alignat}
\end{subequations}
This method of expanding the variables is called pertubation expansion. The 
superscript $\Box^{(i)}$ indicates the $i$-th order of the respective 
parameter. The zeroth order fields $\zeOrder{\Box}$ are defined to be constant 
in space and time. Additionally, we assume no constant flow of the fluid and, 
hence, the zeroth order of the velocity $\zeOrder{\vel_{i}} = 0$.

This series expansion is useful because it is known that the ARF occurs due to 
second order effects. With that it is possible to approximate 
\cref{eq:TA-def-rad} as
\begin{equation}
  F^{\MR{rad}}_{i} \approx
  \timeaverage{\int_{S(t)}\,\stOrder{\stress}\,\normal(t)\,\dd{S(t)}} +
  \int_{\zeOrder{S}}\,\timeaverage{\ndOrder{\stress}}\,
  \normal\,\dd{\zeOrder{S}}
  \label{eq:TA-def-approx-ARF}
\end{equation}
where $\zeOrder{S}$ is the particle surface at rest (see 
\cref{fig:TA-deformed_circle}).

It is possible to solve first for the first order fluid field, which is also 
called acoustic scattering, and then use the first order solution to compute 
the second order fields, also known as AS. In the following two section, we 
will solve those two orders sequentially.

\section{First Order Solution\label{sec:TA-firstorder}}

By applying the Pertubation expansion to \cref{eq:TA-massconservation} and by 
only collecting terms that are equal to the first order the mass conservation 
is
\begin{equation}
  \pdv{\stOrder{\density}}{t} + 
  \zeOrder{\density}\,\pdv{\stOrder{\vel_{i}}}{x_{i}} = 0.
  \label{eq:TA-st-massconservation}
\end{equation}

With the same procedure the general Navier Stockes equation 
(\cref{eq:TA-navierconservative}) is
\begin{equation}
  \zeOrder{\density} \pdv{\stOrder{\vel_{i}}}{t} = 
  \pdv{\stOrder{\stress}}{x_{j}}.
  \label{eq:TA-st-navier}
\end{equation}
with the first order stress tensor $\stOrder{\stress}$ being
\begin{equation}
  \stOrder{\stress} = \muef\left[ \pdv{\stOrder{\vel_{i}}}{x_{j}} + 
  \pdv{\stOrder{\vel_{j}}}{x_{i}} \right] + \left[ \mu_{\MR{B}} - 
  \frac{2}{3}\,\muef \right] \pdv{\stOrder{\vel_{k}}}{x_{k}} \,\delta_{ij} - 
  \stOrder{\pre}\,\delta_{ij}.
  \label{eq:TA-st-stress}
\end{equation}

The acoustic pressure of first order is
\begin{equation}
  \stOrder{\pre} = c^{2}\,\stOrder{\density}
  \label{eq:TA-st-pre}
\end{equation}
with $c$ being the speed of sound.

The first order velocity field $\stOrder{\vel_{i}}$ can be approximated as
\begin{equation}
  \stOrder{\vel}_{i} = \pdv{\stOrder{\phi}}{x_{i}} + 
  \epsilon_{kli}\,\pdv{\stOrder{\psi}_{l}}{x_{k}}
  \label{eq:TA-st-vel}
\end{equation}
where $\phi$ is the scalar velocity potential, $\psi_{i}$ the vector velocity 
potential, and $\epsilon_{kli}$ the Levi-Civita symbol to express the curl of 
the vector velocity potential in index notation. Furthermore, the scalar can be 
split up into the contributions from the incident velocity potential 
$\phi_{\MR{in}}$ and the scattered velocity potential $\phi_{\MR{sc}}$
\begin{equation}
  \stOrder{\phi} = \phi_{\MR{in}} + \phi_{\MR{sc}}.
  \label{eq:TA-scalar-potential}
\end{equation}

For the derivations we assume an axisymmetric incoming field and with the 
wavevector $\vb{k}$ pointing along the $\ez$ direction (see 
\cref{fig:TA-plane_wave}). Additionally, we use spherical coordinates $r, 
\theta$, and $\varphi$ with the origin at the center of the particle in its 
equilibrium configuration (see \cref{fig:TA-coordinate}).

\begin{figure}[tbp]
  \centering
  % \tikzsetnextfilename{plane_wave}
%----------------------
%Plot display orientation
\pgfmathsetmacro{\thetaC}{80}
\pgfmathsetmacro{\phiC}{100}
\tdplotsetmaincoords{\thetaC}{\phiC}
%----------------------
%Parameters 
\pgfmathsetmacro{\R}{4}     % COS
\pgfmathsetmacro{\k}{2.7}   % distance k vector
\pgfmathsetmacro{\rvec}{2}  % Particle's radius
\pgfmathsetmacro{\l}{4}   % width of planar wave
\pgfmathsetmacro{\endl}{7}  % Azimuthal dashed line

\begin{tikzpicture}[tdplot_main_coords]
\coordinate (O) at (0,0,0);
\coordinate (X) at (\R,0,0);
\coordinate (Y) at (0,0,\R);
\coordinate (Z) at (0,\R,0);
\tdplotsetcoord{P}{2*\R}{-90}{0}

%----------------------
% Particle
\shadedraw[tdplot_screen_coords,particleBall, opacity=.4](O) circle (\rvec);
% Edges for labels
\draw[-](0,.3*\R,.2*\R) to[out=40, in=180] ++(0,.4*\R,.45*\R)
    node[right] {Particle};
%-----------------------
% Radial
    \draw[thick,dashed] (O) -- (P);
    %\tdplotdrawarc[->]{(P)}{0.7}{0}{90}{anchor=north}{$\theta$}
\pic [draw, ->, "$\theta$", angle eccentricity=1.2,angle radius=1cm] {angle = Z--O--P};
%-----------------------
% Equator
\draw[dashed] (\rvec,0,0) arc (0:360:\rvec);
% Draw the arc which center is (O) from \phiC to \phiC-180 deg
\draw[thick] ([shift=(\phiC:\rvec)]O) arc (\phiC:\phiC-180:\rvec);
%-----------------------
% Cartesian COS
\begin{scope}[->,thick]
    \draw (O) -- (X)
        node[anchor=north east]{$\ey$};
    \draw (O) -- (Y)
        node[anchor=north west]{$\ex$};
  \draw (O) -- (Z)
    node[anchor=south east]{$\ez$};
\end{scope}
%----------------------
%----------------------
% Axis of symmetry
\draw[dashed] ($(O)+(0,\endl-1.5,0)$) -- (0,-\endl+1,0);
% Edges for labels
\draw[-] ($(O)+(0,\endl-2,0)$) to[out=120, in=-90] ++($(O)+(0,\endl-8,.25*\R)$)
    node[above]{Axis of symmetry};
%----------------------
% Incoming wave vector
\coordinate (P) at (0,-\k,0);
\draw[->,thick,color=red] (0,-1.5*\k,0) -- (P) 
node[above,pos=.6]{$\vb{k}=k\vb{e}_{\MR{k}}$};
 %----------------------
% Incoming wave fronts
\foreach \z in {-7,-6,-5}
\draw [red, fill=red!20, opacity=.4] plot (\l,\z,1) -- (-\l,\z,1) --(-\l,\z,-1) -- (\l,\z,-1) -- cycle;
\end{tikzpicture}



  \includegraphics[]{External/plane_wave.pdf}
  \caption{Plane acoustic wave incident on spherical particle. Wavevector 
  direction $\vb{e}_{\MR{k}}$ aligned with $\ez$.}
  \label{fig:TA-plane_wave}
\end{figure}

For the general case, the scaler incident velocity potential can be represented 
as
\begin{equation}
  \phi_{\MR{in}} = \exp\left( -\iu\omega t \right) \sum\limits_{n=0}^{\infty} 
  A_{n}\,\bessel{n}{kr}\,\legendre{n}{\cos\,\theta}
  \label{eq:TA-incident-field}
\end{equation}
where $\omega=2\pi\,\fex$ is the angular frequency of the incoming acoustic 
wave, $t$ the time, $A_{n}$ the amplitude of the incoming wave, 
$\bessel{n}{\Box}$ the $n$-th order spherical Bessel function of first kind, 
$\legendre{n}{\Box}$ the Legendre polynomial of order $n$, and $k$ the 
wavenumber
\begin{equation}
  k = 
  \frac{\omega}{c}\,\frac{1}{\sqrt{1-\frac{\iu\,\omega}{\zeOrder{\density}\,c^{2}} 
  \left( \mu_{\MR{B}} + \frac{4}{3}\,\muef \right)}}.
  \label{eq:TA-wavenumber}
\end{equation}

The amplitudes $A_{n}$ of \cref{eq:TA-incident-field} are known by the type of 
incident wave (traveling, standing, spherical). $\phi_{\MR{sc}}$ and 
$\psi_{i}$ can be separated into two independent differential equations. For 
$\phi_{\MR{sc}}$ we take first \cref{eq:TA-scalar-potential} and plug it into 
the mass conservation \cref{eq:TA-st-massconservation}. With the property that 
the divergence of the curl ($\pdv{\epsilon_{kli}\,\Box_{i}}{x_{i}} = 0$) is 
equal to zero and the definition of the first order pressure 
\cref{eq:TA-st-pre} one can eliminate $\psi_{i}$ completely
\begin{equation}
  \pdv{\stOrder{\pre}}{t} = \zeOrder{\density}\,c^{2}\,\pdv[2]{\phi}{x_{i}}.
  \label{eq:TA-st-tmp-step}
\end{equation}
The next step is to take \cref{eq:TA-st-tmp-step} and plug it into the partial 
time derivative ($\pdv{\Box}{t}$) of \cref{eq:TA-st-navier}. This results in a 
wave equation which can be further simplified because we assume time harmonic 
acoustics fields ($\vel_{i}\propto\exp -\iu\omega t$) that have the property 
$\pdv{\vel_{i}}{t}\propto -\iu\omega$ and
$\pdv[2]{\vel_{i}}{t}\propto \omega^{2}$
\begin{equation}
  \label{eq:TA-st-nd-tmp-step}
\begin{split}
  \zeOrder{\density}\,\pdv[2]{\vel_{i}}{t} = & 
  \zeOrder{\density}\,c^{2}\pdv{x_{i}}\left( \pdv[2]{\phi}{x_{j}} \right) \\
  + & \mu\,\pdv[2]{x_{i}}\left( \pdv{\vel_{k}}{t} \right) \\
    + & \left(\mu_{\MR{B}} + \frac{\mu}{3}\right)\pdv{t}
    \left( \pdv{x_{i}}\left( \pdv[2]{\phi}{x_{j}} \right) \right).
\end{split}
\end{equation}

The last step is to take the divergence of \cref{eq:TA-st-nd-tmp-step} to 
eliminate $\psi_{i}$. Finally, the equations for $\phi$ is given as a Helmholtz 
equation
\begin{equation}
  \pdv[2]{\phi}{x_{i}} + k^{2}\,\phi = 0.
  \label{eq:TA-st-phi}
\end{equation}

\begin{figure}[tbp]
  \centering
  % \tikzsetnextfilename{coordinate}
\tdplotsetmaincoords{60}{110}
%
% Parameters for vector
\pgfmathsetmacro{\rvec}{1}
\pgfmathsetmacro{\thetavec}{30}
\pgfmathsetmacro{\phivec}{60}
%
%Parameters for COS
\pgfmathsetmacro{\R}{1.2}
\pgfmathsetmacro{\r}{1}
%
\begin{tikzpicture}[scale=3,tdplot_main_coords]
    \coordinate (O) at (0,0,0);
    \coordinate (X) at (\R,0,0);
    \coordinate (Y) at (0,\R,0);
    \coordinate (Z) at (0,0,\R);

    % Cartesian COS
    \draw[thick,->] (O) -- (X)
        %node[midway,sloped,above,color=note]{$x$}
        node[anchor=north east]{$\ex$};
    \draw[thick,->] (O) -- (Y)
        %node[pos=.6,sloped,above,color=note]{$y$}
        node[anchor=north west]{$\ey$};
  \draw[thick,->] (O) -- (Z)
    %node[midway,sloped,above,color=note]{$z$}
    node[anchor=south]{$\ez$};

  %Shade
  \shade[particle,opacity=0.3] (\r,0) arc (0:90:\r) {[x={(0,0,\r)}] arc (90:0:\r)} {[y={(0,0,\r)}] arc (90:0:\r)};

    % Spherical coordinates
    \tdplotdrawarc[->]{(O)}{0.2}{0}{\phivec}{anchor=north}{$\varphi$}
    \tdplotsetthetaplanecoords{\phivec}
    \tdplotdrawarc[->,tdplot_rotated_coords]{(0,0,0)}{.5}{0}%
        {\thetavec}{anchor=south west}{$\theta$}

     %Hilfslinien
  \draw[dashed,tdplot_rotated_coords] (\rvec,0,0) arc (0:90:\rvec)node(Rxy){};
    %\draw[dashed] (\rvec,0,0) arc (0:90:\rvec);
    \draw[dashed] (O) -- (Rxy);

  % Vector
    \tdplotsetcoord{P}{\rvec}{\thetavec}{\phivec}
    \draw[->,thick,color=red] (O) -- (P)
    % label vector
    node[above,left,color=red] at (P) {$\vb{r}$}
    %label coordinate r
        node[midway,below,sloped]{$r$};
    \draw[dashed, color=red] (O) -- (Pxy);
    \draw[dashed, color=red] (P) -- (Pxy);

   %  New coordinate origin
    \tdplotsetrotatedcoords{\phivec}{\thetavec}{0}
    \tdplotsetrotatedcoordsorigin{(P)}
    % Spherical COS
    \draw[thick,tdplot_rotated_coords,->] (P)
        -- (.5,0,0) node[anchor=north west]{$\vb{e}_\theta$};
    \draw[thick,tdplot_rotated_coords,->] (0,0,0)
        -- (0,.5,0) node[anchor=west]{$\vb{e}_\varphi$};
    \draw[thick,tdplot_rotated_coords,->] (0,0,0)
        -- (0,0,.5) node[anchor=south]{$\vb{e}_r$};
\end{tikzpicture}


  \includegraphics[]{External/coordinate.pdf}
  \caption{Sketch of Cartesian and spherical coordinate system}
  \label{fig:TA-coordinate}
\end{figure}

Instead of taking the divergence of \cref{eq:TA-st-nd-tmp-step} but taking the 
curl one gets a Helmholtz equation for $\psi_{i}$
\begin{equation}
  \pdv[2]{\psi_{j}}{x_{i}} + k_{\MR{v}}^{2}\,\psi_{i} = 0
  \label{eq:TA-st-psi}
\end{equation}
by using $\epsilon_{kli}\,\pdv{x_{k}}\left( \pdv{\Box}{x_{l}} \right) = 0$ 
(curl of the gradient of a scalar). In \cref{eq:TA-st-psi} $k_{\MR{v}}$ is 
called the viscous wavenumber and is defined as
\begin{equation}
  k_{\MR{v}} = \frac{1 + \iu}{\delta} = \left( 1 + \iu \right)\, 
  \sqrt{\frac{\omega\,\zeOrder{\density}}{2\rhof}}
  \label{eq:TA-st-visc-wavenumber}
\end{equation}
where $\delta$ is also known as the VBL thickness.

With the requirement that the solutions to \cref{eq:TA-st-phi,eq:TA-st-psi} 
satisfy Sommerfeld's radiation condition and using multipole expansion one 
finds
\begin{equation}
  \phi_{\MR{sc}} = \exp\left( -\iu\omega t \right)\,
  \sum\limits^{\infty}_{n=0} 
  \alpha_{n}\,A_{n}\,\hankel{n}{k\,r}\,\legendre{n}{\cos\theta}
  \label{eq:TA-st-sol-phi}
\end{equation}
and
\begin{equation}
  \vb*{\psi}^{(1)} = \exp\left( -\iu\omega t \right)\,\vb{e}_{\varphi}\,
  \sum\limits^{\infty}_{n=1} 
  \beta_{n}\,A_{n}\,\hankel{n}{k_{\MR{v}}\,r}\,P^{1}_{n}(\cos\theta)
  \label{eq:TA-st-sol-psi}
\end{equation}
where $\alpha_{n}$ and $\beta_{n}$ are constants defined by the boundary 
conditions, $\vb{e}_{\varphi}$ the unit vector of the spherical coordinate 
system (see \cref{fig:TA-coordinate}), $\hankel{n}{\Box}$ the spherical Hankel 
functions of the first kind, and $P^{1}_{n}\left( \Box \right)$ the associated 
Legendre polynomial. The sum of \cref{eq:TA-st-sol-psi} starts with $n=1$ 
because $\beta_{0}=0$.

The boundary condition for the combination of a rigid spherical particle and a 
viscous fluid is that the velocity of the particle $\vel_{\MR{p}}$ at the 
surface must match the fluid velocity there
\begin{equation}
  \vel_{i,\MR{fluid}} = \vel_{i,\MR{p}}\quad\text{at}\quad r = R.
  \label{eq:TA-st-BC}
\end{equation}

Using Newtown's second law and the first order stress tensor
\begin{equation}
  \underbrace{\frac{4}{3}\,\pi\,R^{3}\,\density_{\MR{p}}}_{m}
  \underbrace{\dv{\vel_{i,\MR{p}}}{t}}_{a} = 
  \int\limits_{\zeOrder{S}}\stOrder{\stress}\dd{\zeOrder{S}}
  \label{eq:TA-st-newton}
\end{equation}
where $\density_{\MR{p}}$ is the particle density, one finds the particle 
velocity along the direction of the wave vector $\vb{e}_{k}$ as
\begin{equation}
  \vel_{\MR{p}} = \frac{\zeOrder{\density}A_{1}k}{x}\,
  \left[
    \bessel{1}{x} + \alpha_{1}\hankel{1}{x} + 2\beta_{1}\hankel{1}{\xv}
  \right]
  \,\exp\left( -\iu\omega t \right)
  \label{eq:TA-st-particle-vel}
\end{equation}
with $x=kR$ and $\xv=k_{\MR{v}}R$.

In order to calculate the unknown constants $\alpha_{n}$ and $\beta_{n}$ one 
needs to enforce the boundary condition of \cref{eq:TA-st-BC} for every $n$ 
separately. Besides for $n=1$, $\vel_{\MR{p}}$ is always zero because the 
multipole expansion for a rigid particle has only a dipole contribution; for 
$n=1$, $\vel_{\MR{p}}$ is given by \cref{eq:TA-st-particle-vel}.

For $n=0$ and $n=1$ the constants $\alpha_{0}, \alpha_{1}, \beta_{0},$ and 
$\beta_{1}$ are
\begin{subequations}
\begin{align}
  \alpha_{0} & = -\frac{\bessel{1}{x}}{\hankel{1}{x}} \\
  \beta_{0} & = 0 \\
  \alpha_{1} & = - \frac{1}{\chi_{4}} \left[ \chi_{1}\,\chi_{3} + 
  2(1-\trho^{2})\bessel{1}{x}\hankel{1}{\xv} \right]
    \label{eq:TA-alpha-one}\\
    \beta_{1} &= - \frac{1-\trho}{\chi_{4}} \left[ \chi_{1}\,\hankel{1}{x} - 
    \chi_{2}\,\bessel{1}{x} \right]
    \label{eq:TA-beta-one}
\end{align}
\end{subequations}
where $\trho = \sfrac{\rhof}{\density_{\MR{p}}}$ and
\begin{subequations}
\begin{align}
\chi_{1} & = \trho\,\bessel{1}{x} - \trho\,\dbessel{1}{x}
\label{eq:TA-chi-one} \\
\chi_{2} & = \trho\,\hankel{1}{x} - \trho\,\dhankel{1}{x}
\label{eq:TA-chi-two} \\
\chi_{3} & = \left( 1 + 2 \trho\right)\hankel{1}{\xv} + 
\xv\,\dhankel{1}{x}
\label{eq:TA-chi-three} \\
\chi_{4} & = \chi_{2}\,\chi_{3} + 2\left( 1 - \trho^{2} \right) 
\hankel{1}{x}\,\hankel{1}{\xv}
\label{eq:TA-chi-four}
\end{align}
\end{subequations}
where the prime $\Box^{\prime}$ indicates the differentiation, e.g. 
$\dbessel{n}{\xv}=\dv{\bessel{n}{\xv}}{\xv}$.

For $n > 1$, one finds the coefficients $\alpha_{n}$ and $\beta_{n}$ to be
\begin{subequations}
\begin{align}
  \alpha_{n} & =
  - \frac{1}{\xi_{n}} \left[
    n(n+1)\bessel{n}{x}\hankel{n}{\xv} - x \gamma_{n}\dbessel{n}{x}
  \right]
  \label{eq:TA-alpha-n}\\
    \beta_{n} &= - \frac{1}{\xi_{n}} \left[
      x\dbessel{n}{x}\hankel{n}{x} - x\bessel{n}{x}\dhankel{n}{x}
    \right]
    \label{eq:TA-beta-n}
\end{align}
\end{subequations}
with
\begin{subequations}
\begin{align}
  \gamma_{n} & =
  \hankel{n}{\xv}+\xv\dhankel{n}{\xv}
  \label{eq:TA-gamma-n}\\
  \xi_{n} &= x\dhankel{n}{x}\gamma_{n} - n(n+1)\hankel{n}{x}\hankel{n}{\xv}.
    \label{eq:TA-xi-n}
\end{align}
\end{subequations}

\begin{table}
  \centering
  \begin{tabular}{lccr}
    \toprule
    \toprule
    {\bfseries Parameter} & {\bfseries Symbol} & {\bfseries Value} & {\bfseries 
    Unit}\\
    \midrule
    \textbf{Fluid} & & \\
    Bulk viscosity & $\mu_{\MR{B}}$ & 2.47 & \si{\milli\pascal\second} \\
    \midrule
    \textbf{Wave} & & \\
    Wavetype &  & travelling & - \\
    Pressure & $p_{\MR{a}}$ & 100.0 & \si{\kilo\pascal} \\
    Frequency & $f$ & 1.0 & \si{\MHz} \\
    \bottomrule
    \bottomrule
  \end{tabular}
  \caption{Symbols and physical properties of the fluid, the particle, and the 
  incident acoustic wave.}\label{tab:TA-parameters}
\end{table}

Note here, that 
\cref{eq:TA-alpha-one,eq:TA-beta-one,eq:TA-chi-three,eq:TA-chi-four,eq:TA-beta-n} 
are different to the equations in \cname{Doinikov1994Rigid}; Equations (3.14), 
(3.15), (3.18), (3.19), and (3.21) respectively. We discovered these typos in 
the original publication and discussed it more extensively in 
\cname{Fankhauser2022}.

\begin{figure}
  \centering
  \begin{subfigure}[b]{\textwidth}
    \centering
    \caption{Modes up to $n=5$}
    % \tikzsetnextfilename{SC_all}

\renewcommand{\tikzHelper}{\relPath/10_Figures/TikZ/scattering_data}

\begin{tikzpicture}
\begin{groupplot}[
  group style={
    vertical sep=6mm,
    horizontal sep=3mm,
    group size= 2 by 3,
  },
  title style={yshift=-2.5mm,},
  xmin=-5,
  xmax=5,
  ymin=0,
  ymax=5,
  point meta max=75,
  point meta min=4,
  view={0}{90},
  % colorbar,
  colormap = {whiteblack}{color(0cm)  = (black);color(1cm) = (white)},
  % colormap/bluered,                     % Colormap preset
  height=4cm,
  width=75mm,
  ]

  %%%%%%%%%%%%%%%%
  \nextgroupplot[
    title={$\omega t =0\,\pi$},
    ylabel={$\sfrac{x}{R}$},
    xticklabels={,,},
]
    \addplot3[
        contour filled={number=25,labels={false}},
        mesh/rows=110,
        mesh/check=false
    ] table[x=z, y=x, z=c] {\tikzHelper/phase_0.dat};

    \coordinate (top) at (rel axis cs:0,1);
  %%%%%%%%%%%%%%%%
  \nextgroupplot[
    title={$\omega t =0.2\,\pi$},
    xticklabels={,,},
    yticklabels={,,},
]
    \addplot3[
        contour filled={number=25,labels={false}},
        mesh/rows=110,
        mesh/check=false
    ] table[x=z, y=x, z=c] {\tikzHelper/phase_1.dat};

  %%%%%%%%%%%%%%%%
  \nextgroupplot[
    title={$\omega t =0.4\,\pi$},
    ylabel={$\sfrac{x}{R}$},
    xticklabels={,,},
]
    \addplot3[
        contour filled={number=25,labels={false}},
        mesh/rows=110,
        mesh/check=false
    ] table[x=z, y=x, z=c] {\tikzHelper/phase_2.dat};

  %%%%%%%%%%%%%%%%
  \nextgroupplot[
    title={$\omega t =0.6\,\pi$},
    xticklabels={,,},
    yticklabels={,,},
]
    \addplot3[
        contour filled={number=25,labels={false}},
        mesh/rows=110,
        mesh/check=false
    ] table[x=z, y=x, z=c] {\tikzHelper/phase_3.dat};

  %%%%%%%%%%%%%%%%
  \nextgroupplot[
    title={$\omega t =0.8\,\pi$},
    ylabel={$\sfrac{x}{R}$},
    xlabel={$\sfrac{z}{R}$},
]
    \addplot3[
        contour filled={number=25,labels={false}},
        mesh/rows=110,
        mesh/check=false
    ] table[x=z, y=x, z=c] {\tikzHelper/phase_4.dat};

  %%%%%%%%%%%%%%%%
  \nextgroupplot[
    title={$\omega t =1.0\,\pi$},
    yticklabels={,,},
    xlabel={$\sfrac{z}{R}$},
]
    \addplot3[
        contour filled={number=25,labels={false}},
        mesh/rows=110,
        mesh/check=false
    ] table[x=z, y=x, z=c] {\tikzHelper/phase_5.dat};

  \coordinate (bot) at (rel axis cs:1,0);

\end{groupplot}
  \path (top-|current bounding box.east)--
                    coordinate(legendposright)
                    (bot-|current bounding box.east);


\begin{axis}[%
  hide axis,
  scale only axis,
  %height=0.975\linewidth,
  %width=0.01\linewidth,
  at={(legendposright.east)},
  anchor=east,
  xshift=60mm,
  point meta min=4.0,
  point meta max=75.0,
  % colormap/bluered,                     % Colormap preset
  colormap = {whiteblack}{color(0cm)  = (black);color(1cm) = (white)},
  colorbar right,                  % Active colorbar
  colorbar sampled,                     % Steps in colorbar
  colorbar style={
    separate axis lines,
    samples=256,                        % Number of steps
    ylabel={Acoustic Velocity $\abs{\vb{v}}$ [\si{\mm\per\second}]},
  },
  every colorbar/.append style={
    height=\pgfkeysvalueof{/pgfplots/parent axis height}%+
               %\pgfkeysvalueof{/pgfplots/group/vertical sep}
  }
]
  \addplot [draw=none] coordinates {(0,0)};
\end{axis}

\end{tikzpicture}

    \includegraphics[]{External/SC_all.pdf}
    \label{fig:TA-SC_all}
  \end{subfigure}\\%
  \begin{subfigure}[b]{\textwidth}
    \centering
    \caption{Single mode $n=1$}
    % \input{\relPath/10_Figures/TikZ/SC_mode1.tikz}
    \includegraphics[]{External/SC_mode1.pdf}
    \label{fig:TA-SC_mode1}
  \end{subfigure}
  \caption{First order absolute fluid velocity field at different time steps. 
  For both plots the same scaling applies. a) all modes up to $n=5$; b) single 
  mode for $n=1$. Material properties for the particle and the fluid are given 
  in \cref{tab:TA-parameters,tab:TC-parameters}. The plotted data is generated 
  with the Pyhton module \emph{osaft}~\cite{FankhauserPython2022}.}
  \label{fig:TA-SC}
\end{figure}

We solved the first order velocity field for a rigid spherical particle and a 
viscous fluid. The field inside the particle has the same magnitude and 
direction for all points because the particle is modeled rigid and therefore it 
is only able to perform rigid-body motion. The rigid-body mode in the multipole 
expansion is the dipole mode ($n=1$). Since the boundary conditions enforce a 
matching velocity at the particle-fluid interface, the whole first order fluid 
velocity solution is also dominated by that mode. This is also depicted in 
\cref{fig:TA-SC} where the fluid velocity for water and a silicium dioxide 
particle (SiO$_{2}$) for the sum of all modes through $n=5$ and the fluid 
velocity of the first mode $n=1$ only is computed. \Cref{fig:TA-SC_all} and 
\Cref{fig:TA-SC_mode1} are indistinguishable from each other meaning that the 
solution to the whole first order velocity field is due to the rigid-body mode 
in the multipole expansion.

\section{Second Order Solution\label{sec:TA-secondorder}}

Similar to the first order equations one can collect all terms up to second 
order. The general mass conservation (\cref{eq:TA-massconservation}) transforms 
to
\begin{equation}
  \pdv{\ndOrder{\density}}{t} + \pdv{x_{i}}\left(\stOrder{\density}\,\stOrder{\vel_{i}} 
  + \zeOrder{\density}\,\ndOrder{\vel_{i}} \right) = 0
  \label{eq:TA-nd-massconservatio}
\end{equation}
and the second order Navier-Stokes (\cref{eq:TA-navierconservative}) becomes
\begin{equation}
  \stOrder{\density}\,\pdv{\stOrder{\vel_{i}}}{t} + \zeOrder{\density} 
  \pdv{\ndOrder{\vel_{i}}}{t} + 
  \zeOrder{\density}\,\stOrder{\vel_{j}}\,\pdv{\stOrder{\vel_{j}}}{x_{j}} = 
  \pdv{\ndOrder{\stress}}{x_{j}}
  \label{eq:TA-nd-navier}
\end{equation}
where the second order stress tensor (\cref{eq:TA-stress}) is
\begin{equation}
  \ndOrder{\stress} = \muef\left[ \pdv{\ndOrder{\vel_{i}}}{x_{j}} + 
  \pdv{\ndOrder{\vel_{j}}}{x_{i}} \right] + \left[ \mu_{\MR{B}} - 
  \frac{2}{3}\,\muef \right] \pdv{\ndOrder{\vel_{k}}}{x_{k}} \,\delta_{ij} - 
  \ndOrder{\pre}\,\delta_{ij}.
  \label{eq:TA-nd-stress}
\end{equation}

For the second order velocity field one is usually interested in the 
time-average of this field. This steady-state is referred to as AS field. 
Applying the time-average to \cref{eq:TA-nd-massconservatio,eq:TA-nd-navier} 
and using the property that the time-average of the time-derivative of a 
bounded, differentiable function is zero~\cite{Baasch2020}
\begin{equation}
  \timeaverage{\pdv{t}\,g(\vb{x};t)} = \lim\limits_{\tau \rightarrow T}{\,
    \left[ \frac{1}{\tau}\,\left( g(\vb{x}; \tau) - g(\vb{x}; 0) \right) 
  \right]} = 0,
  \label{eq:TA-avg-property}
\end{equation}
one can simplify the second order mass conservation and second order 
Navier-Stokes equation further to
\begin{equation}
  \pdv{x_{i}}\left( \timeaverage{\ndOrder{\vel_{i}}} \right) = 
  -\frac{1}{\zeOrder{\density}} \pdv{x_{i}}\left(\timeaverage{ 
  \stOrder{\density}\,\stOrder{\vel_{i}} }\right)
  \label{eq:TA-avg-massconservatio}
\end{equation}
and
\begin{equation}
  \timeaverage{\stOrder{\density} \pdv{\stOrder{\vel_{i}}}{t}} + 
  \zeOrder{\density}\,\timeaverage{\stOrder{\vel_{j}}\,\pdv{\stOrder{\vel_{j}}}{x_{j}}} 
  = \pdv{x_{j}}\left( \timeaverage{\ndOrder{\stress}} \right).
  \label{eq:TA-avg-navier}
\end{equation}

Note here, that the first term on the left hand side of \cref{eq:TA-avg-navier} 
does not vanish. Taking the partial time derivative of the time-averaged 
product of the two first order fields 
$\left(\stOrder{\density}\,\stOrder{\vel_{i}}\right)$
\begin{equation}
  \timeaverage{\pdv{t}\left( \stOrder{\density}\,\stOrder{\vel_{i}} \right)}
  =
  \timeaverage{\stOrder{\vel_{i}}\,\pdv{\stOrder{\density}}{t}}
  +
  \timeaverage{\stOrder{\density}\,\pdv{\stOrder{\vel_{i}}}{t}}
\end{equation}
and re-arranging it to
\begin{equation}
  \timeaverage{\stOrder{\density}\,\pdv{\stOrder{\vel_{i}}}{t}}
  =
  \underbrace{
  \timeaverage{\pdv{t}\left( \stOrder{\density}\,\stOrder{\vel_{i}} \right)}
}_{=0}
  -
  \underbrace{
  \timeaverage{\stOrder{\vel_{i}}\,\pdv{\stOrder{\density}}{t}}
}_{\neq 0}
  \label{eq:TA-non-zero-term}
\end{equation}
reveals that the first term on the right hand side of 
\cref{eq:TA-non-zero-term} vanishes because of the property explained in 
\cref{eq:TA-avg-property} and the second term is non-zero.

As for the first order field, the second order velocity can be decomposed into 
a sum of a streaming velocity driven solely by the known incident field and an 
unknown streaming velocity due to the scattered field
\begin{equation}
  \timeaverage{\ndOrder{\vel_{i}}} =
    \timeaverage{\ndOrder{\vel_{i,\MR{in}}}} +
    \timeaverage{\ndOrder{\vel_{i,\MR{sc}}}}.
  \label{eq:TA-nd-velocity}
\end{equation}

The time-averaged second order incident field 
$\timeaverage{\ndOrder{\vel_{i,\MR{in}}}}$ is the time-averaged motion of the 
fluid without particle. For our assumption of a spherical particle in a 
unbounded fluid this type of AS is called \emph{Eckart streaming} named after 
\cname{Eckart1948} who derived it for the first time. However, this type of AS 
is uncommon in MSAF devices because the fluid is always in some kind of a 
cavity. The AS pattern that usually appears in MSAF devices is a combination 
of \emph{Schlichting streaming} at the fluid-structure interface and 
\emph{Rayleigh streaming} in the bulk of the fluid. Rayleigh streaming is 
driven by the boundary near Schlichting streaming. This interaction of the two 
streaming fields is named after \cname{Schlichting1932} and 
\cname{Rayleigh1894}, respectively, and qualitatively depicted in 
\cref{fig:TA-acoustic_streaming} for a plane standing pressure wave with a 
single pressure nodal plane. Note here, that this interaction can also occur at 
the particle-fluid interface. Depending on the material parameter of the fluid 
and the particle, the streaming contribution around the particle can become 
large enough and substantially contribute to the total force~\cite{Baasch2019}. 
In order to visualize any of the mentioned streaming fields, one needs to 
perform a numerical simulation and solve for the time-averaged fields.


\begin{figure}[tbp]
  \centering
  % \tikzsetnextfilename{acoustic_streaming}
{\small
% Parameters for vector
\pgfmathsetmacro{\W}{12.0}
\pgfmathsetmacro{\H}{6.0}
\pgfmathsetmacro{\B}{0.5}

\pgfmathsetmacro{\d}{1.1}

\pgfmathsetmacro{\fac}{0.5}
\pgfmathsetmacro{\amplitude}{1.8}
\pgfmathsetmacro{\doublefac}{2*\fac}

% Syntax: [draw options] (center) (initial angle:final angle:radius)
\def\centerarc[#1](#2)(#3:#4:#5)
  { \draw[#1] ($(#2)+({#5*cos(#3)},{#5*sin(#3)})$) arc (#3:#4:#5); }

\newcommand{\leftstreaming}[4]{
  \draw[-latex] (#1,#2) ellipse (0.9*\W/4 and 0.4*#4);

  \draw[ultra thick, #3, -latex] (#1,#2) [partial ellipse=205:285:0.9*\W/4 and 
  0.4*#4];
  \draw[ultra thick, #3, -latex] (#1,#2) [partial ellipse=305:425:0.9*\W/4 and 
  0.4*#4];
  \draw[ultra thick, #3, -latex] (#1,#2) [partial ellipse=435:515:0.9*\W/4 and 
  0.4*#4];
}

\newcommand{\rightstreaming}[4]{
  \draw[-latex] (#1,#2) ellipse (0.9*\W/4 and 0.4*#4);
  \draw[ultra thick, #3, -latex] (#1,#2) [partial ellipse=105:25:0.9*\W/4 and 
  0.4*#4];
  \draw[ultra thick, #3, -latex] (#1,#2) [partial ellipse=225:125:0.9*\W/4 and 
  0.4*#4];
  \draw[ultra thick, #3, -latex] (#1,#2) [partial ellipse=345:255:0.9*\W/4 and 
  0.4*#4];
}

\newcommand{\streamingOne}[4]{
  \leftstreaming{#1}{#2}{#3}{#4}
  \rightstreaming{#1+\W/2}{#2}{#3}{#4}
}

\newcommand{\streamingTwo}[4]{
  \rightstreaming{#1}{#2}{#3}{#4}
  \leftstreaming{#1+\W/2}{#2}{#3}{#4}
}


\begin{tikzpicture}[]

  \coordinate (O) at (0,0);
  \coordinate (BL) at (-\W/2,-\H/2);
  \coordinate (TR) at (\W/2,\H/2);

  % walls
  \draw[pattern=north west lines, pattern color=black!50] ($(BL) - (\B, 
  \B)$) rectangle ++(2*\B+\W,2*\B+\H);

  % fluid
  \draw[fill=blue!10] (BL) rectangle (TR);
  % boundary layer
  \path[fill=blue!20] (BL) rectangle ($(BL) + (\W,\d)$);
  \path[fill=blue!20] (TR) rectangle ($(TR) - (\W,\d)$);


  % coordinate system
  \draw[<->] ($(-\W/2*0.9,-\H/4)$) -- node[left,pos=0] {$\ez$} ++(0,-\H/4*0.9) 
  -- node[above,pos=1] {$\ey$} ++(\H/7,0);

  % pressure field
  \draw[blue] plot[domain=-\W/2:\W/2,samples=100] 
  (\x,{+\amplitude*sin(deg(2*pi/(\W/\fac)*\x))});
  \draw[blue] plot[domain=-\W/2:\W/2,samples=100] 
  (\x,{-\amplitude*sin(deg(2*pi/(\W/\fac)*\x))});


  % annotations width
  \draw[|<->|] (-\W/2, \H/2+1.5*\B) -- node[above,midway] {$W$} ++(\W,0);
  \draw[|<->|] (\W/2+1.5*\B, -\H/2) -- node[right,midway] {$H$} ++(0,\H);

  \draw[|<->|] (-\W/2-1.5*\B, -\H/2) -- node[left,midway] {$\delta$} ++(0,\d);
  \draw[|<->|] (-\W/2-1.5*\B, +\H/2) -- node[left,midway] {$\delta$} ++(0,-\d);

  % schlichting streaming
  \streamingOne{-\W/4}{\H/2-\d/2}{red}{\d}
  \streamingTwo{-\W/4}{-\H/2+\d/2}{red}{\d}

  % rayleigh streaming
  \streamingTwo{-\W/4}{\H/4-\d/2}{olive}{\H/2-\d/3}
  \streamingOne{-\W/4}{-\H/4+\d/2}{olive}{\H/2-\d/3}
\end{tikzpicture}
}

  \includegraphics[]{External/acoustic_streaming.pdf}
  \caption{Schlichting and Rayleigh acoustic streaming for an one-dimensional 
      standing pressure wave (blue line) in a rectangular fluid cavity of width 
      $W$ and heigth $H$. Near the viscous boundary layer $\delta$ (darker 
  regions at the boundary) Schlichting streaming (\textcolor{red}{red arrows}) 
  is present and in the bulk of the fluid Rayleigh acoustic streaming 
  (\textcolor{olive}{olive arrows}).}
  \label{fig:TA-acoustic_streaming}
\end{figure}

The scattered field is -- as before -- decomposed into the gradient of a scalar 
velocity potential and the curl of a vector velocity potential
\begin{equation}
  \timeaverage{\ndOrder{\vel_{i,\MR{sc}}}} =
  \pdv{\ndOrder{\phi}}{x_{i}} + 
  \epsilon_{kli}\,\pdv{\ndOrder{\psi}_{l}}{x_{k}}.
  \label{eq:TA-nd-sc-decomposition}
\end{equation}

The boundary conditions for the time-averaged second order velocity field is 
defined at the particle-fluid interface and enforces the equality of the second 
order Eulerian velocity to the negative Stokes drift
\begin{equation}
  \timeaverage{\ndOrder{\vel_{i}}}
  =
  -
  \underbrace{
    \timeaverage{\frac{1}{-\iu\omega}\,\vel_{j}\,\pdv{\vel_{i}}{x_{j}}}
  }_{\text{Stokes drift}}
  \quad\text{at~}r=R.
  \label{eq:TA-nd-BCs}
\end{equation}


At this point we will not repeat the calculation to solve for the second order 
velocity field. Interested readers are pointed to \cname{Doinikov1994Rigid} 
Equation (4.1) through (4.34).

\begin{figure}
  \centering
  \begin{subfigure}[b]{\textwidth}
    \centering
    \caption{Modes through $n=5$}
    % \tikzsetnextfilename{SC_all}

\renewcommand{\tikzHelper}{\relPath/10_Figures/TikZ/scattering_data}

\begin{tikzpicture}
\begin{groupplot}[
  group style={
    vertical sep=6mm,
    horizontal sep=3mm,
    group size= 2 by 3,
  },
  title style={yshift=-2.5mm,},
  xmin=-5,
  xmax=5,
  ymin=0,
  ymax=5,
  point meta max=75,
  point meta min=4,
  view={0}{90},
  % colorbar,
  colormap = {whiteblack}{color(0cm)  = (black);color(1cm) = (white)},
  % colormap/bluered,                     % Colormap preset
  height=4cm,
  width=75mm,
  ]

  %%%%%%%%%%%%%%%%
  \nextgroupplot[
    title={$\omega t =0\,\pi$},
    ylabel={$\sfrac{x}{R}$},
    xticklabels={,,},
]
    \addplot3[
        contour filled={number=25,labels={false}},
        mesh/rows=110,
        mesh/check=false
    ] table[x=z, y=x, z=c] {\tikzHelper/phase_0.dat};

    \coordinate (top) at (rel axis cs:0,1);
  %%%%%%%%%%%%%%%%
  \nextgroupplot[
    title={$\omega t =0.2\,\pi$},
    xticklabels={,,},
    yticklabels={,,},
]
    \addplot3[
        contour filled={number=25,labels={false}},
        mesh/rows=110,
        mesh/check=false
    ] table[x=z, y=x, z=c] {\tikzHelper/phase_1.dat};

  %%%%%%%%%%%%%%%%
  \nextgroupplot[
    title={$\omega t =0.4\,\pi$},
    ylabel={$\sfrac{x}{R}$},
    xticklabels={,,},
]
    \addplot3[
        contour filled={number=25,labels={false}},
        mesh/rows=110,
        mesh/check=false
    ] table[x=z, y=x, z=c] {\tikzHelper/phase_2.dat};

  %%%%%%%%%%%%%%%%
  \nextgroupplot[
    title={$\omega t =0.6\,\pi$},
    xticklabels={,,},
    yticklabels={,,},
]
    \addplot3[
        contour filled={number=25,labels={false}},
        mesh/rows=110,
        mesh/check=false
    ] table[x=z, y=x, z=c] {\tikzHelper/phase_3.dat};

  %%%%%%%%%%%%%%%%
  \nextgroupplot[
    title={$\omega t =0.8\,\pi$},
    ylabel={$\sfrac{x}{R}$},
    xlabel={$\sfrac{z}{R}$},
]
    \addplot3[
        contour filled={number=25,labels={false}},
        mesh/rows=110,
        mesh/check=false
    ] table[x=z, y=x, z=c] {\tikzHelper/phase_4.dat};

  %%%%%%%%%%%%%%%%
  \nextgroupplot[
    title={$\omega t =1.0\,\pi$},
    yticklabels={,,},
    xlabel={$\sfrac{z}{R}$},
]
    \addplot3[
        contour filled={number=25,labels={false}},
        mesh/rows=110,
        mesh/check=false
    ] table[x=z, y=x, z=c] {\tikzHelper/phase_5.dat};

  \coordinate (bot) at (rel axis cs:1,0);

\end{groupplot}
  \path (top-|current bounding box.east)--
                    coordinate(legendposright)
                    (bot-|current bounding box.east);


\begin{axis}[%
  hide axis,
  scale only axis,
  %height=0.975\linewidth,
  %width=0.01\linewidth,
  at={(legendposright.east)},
  anchor=east,
  xshift=60mm,
  point meta min=4.0,
  point meta max=75.0,
  % colormap/bluered,                     % Colormap preset
  colormap = {whiteblack}{color(0cm)  = (black);color(1cm) = (white)},
  colorbar right,                  % Active colorbar
  colorbar sampled,                     % Steps in colorbar
  colorbar style={
    separate axis lines,
    samples=256,                        % Number of steps
    ylabel={Acoustic Velocity $\abs{\vb{v}}$ [\si{\mm\per\second}]},
  },
  every colorbar/.append style={
    height=\pgfkeysvalueof{/pgfplots/parent axis height}%+
               %\pgfkeysvalueof{/pgfplots/group/vertical sep}
  }
]
  \addplot [draw=none] coordinates {(0,0)};
\end{axis}

\end{tikzpicture}

    % \includegraphics[]{External/SC_all.pdf}
    \includegraphics[width=100mm]{example-image-a}
    % \label{fig:TA-SC_all}
  \end{subfigure}\\%
  \begin{subfigure}[b]{\textwidth}
    \centering
    \caption{Single mode one $n=1$}
    % \input{\relPath/10_Figures/TikZ/SC_mode1.tikz}
    % \includegraphics[]{External/SC_mode1.pdf}
    \includegraphics[width=100mm]{example-image-b}
    % \label{fig:TA-SC_mode1}
  \end{subfigure}
  \caption{First order fluid velocity field at different steps during one 
period. For both plots the same scaling applies. a) all modes through $n=5$; 
b) single mode one for $n=1$. Material properties for the particle and the 
fluid are seen in \cref{tab:TA-parameters,tab:TC-parameters}. The plotted data 
is generated with the Python module \emph{osaft}~\cite{FankhauserPython2022}.}
  \label{fig:TA-AS}
\end{figure}

\textcolor{red}{in \cref{fig:TA-AS} yo can asee bloah balh the time averaged 
streaming field around the particle for}

This time-averaged streaming field causes a force onto the particle in the 
direction of the flow which is know as Stokes' drag force
\begin{equation}
  \force^{\MR{drag}}_{i} = \gamma\,\timeaverage{\ndOrder{\vel_{i}}} = 
  6\pi\,R\,\muef\,
  \left( \timeaverage{\ndOrder{\vel_{i}}} - \vel_{i,\MR{p}} \right)
  \label{eq:TA-drag-force}
\end{equation}
where $\gamma$ is also called Stokes' drag coefficient and $\vel_{i,\MR{p}}$ 
the particle velocity.


\section{Acoustic Radiation Force\label{sec:TA-ARF}}

With the solution for the first and second order velocity field and hence for 
the stress tensors of the respective orders one can calculate the ARF with 
\cref{eq:TA-def-approx-ARF}. However, this formula and also the velocity fields 
are cumbersome to compute. It is cumbersome because \cname{Doinikov1994Rigid} 
formula as well as the solutions to the velocity fields are valid for any 
axisymmetric incident acoustic wave field and are not restricted to the ratios 
of the particle radius to the VBL thickness $\sfrac{R}{\delta}$, the ratio of 
the particle radius to the acoustic wavelength $\sfrac{R}{\wavelength_{\MR{a}}} 
= \sfrac{\R\,f}{\cfl}$, and the ratio of the VBL thickness to the acoustic 
wavelength $\sfrac{\delta}{\wavelength_{\MR{a}}}$.

\begin{figure}[tbp]
  \centering
  % \tikzsetnextfilename{lambda_mode}
% Parameters for vector
\pgfmathsetmacro{\W}{8.0}
\pgfmathsetmacro{\Wrect}{1.0}
\pgfmathsetmacro{\Hrect}{3.0}

\pgfmathsetmacro{\freq}{pi/4}

\pgfmathsetmacro{\xpOne}{\freq/pi*\W}
\pgfmathsetmacro{\xpTwo}{\freq/pi*\W+\W/2}

\begin{tikzpicture}[]

  \coordinate (O) at (0,0);
  \coordinate (TL) at (\W,\Hrect);

  % coordinate system

  \draw[<->] ($(-\Wrect*1.5,\Hrect/2)$) -- node[left,pos=0] {$\ez$} 
  ++(0,-\Hrect/1.7) -- node[right,pos=1] {$\ey$} ++(\Hrect/2,0);

  % fluid
  \fill[blue!10] (O) rectangle (TL);

  % middle line
  \draw[black, thick, dotted] ($(O) + (0,\Hrect/2)$) -- ++(\W,0);

  % nodal planes
  \draw[black, dotted] ($(\xpOne,\Hrect/2)$) -- ++(0,\Hrect/2);
  \draw[black, dotted] ($(\xpTwo,\Hrect/2)$) -- ++(0,\Hrect/2);


  % walls
  \draw[pattern=north west lines, pattern color=black!50] (0,0) rectangle 
  ++(-\Wrect,\Hrect);
  \draw[pattern=north west lines, pattern color=black!50] (\W,0) rectangle 
  ++(\Wrect,\Hrect);

  % pressure
  \draw[blue] plot[domain=0:\W,samples=100] (\x,{cos(deg(\freq*\x))+\Hrect/2});
  \draw[blue] plot[domain=0:\W,samples=100] 
  (\x,{-cos(deg(\freq*\x))+\Hrect/2});

  % force field
  % \draw[red, dashdotted] plot[domain=0:\W,samples=100] 
  % (\x,{sin(deg(2*\freq*\x))+\Hrect/2});

  % particles
  \node[shade,shading=ball,circle,ball color=black!50!white,minimum size=3mm] 
  (positive) at  (\xpOne,\Hrect/3) {};
  \node[shade,shading=ball,circle,ball color=black!50!white,minimum size=3mm] 
  (positivetwo) at  (\xpTwo,\Hrect/4) {};

  \node[shade,shading=ball,circle,ball color=red!50!white,minimum size=2mm] 
  (negative) at  (\W/2,\Hrect*2/3) {};

  % annotations width
  \draw[|<->|] ($(O) - (0,0.8)$) -- node[above,midway] {$W$} ++(\W,0);
  % annotations pressure field
  \draw[|<->|] ($(\xpOne,\Hrect)+ (0,0.5)$) -- node[above,midway] 
  {$\sfrac{\lambda_{\mathrm{a}}}{2}$} ++($(\W/2,0)$);
  % annotations positive
  \draw[thin,-](positive) to[out=-130, in=180] ++(-0.5,-0.7)
      node[right] {$\Phi > 0$};
  \draw[thin,-](positivetwo) to[out=300, in=180] ++(0.2,-0.4)
      node[right] {$\Phi > 0$};
  % annotations positive
  \draw[-, thin] (negative) to[out=120, in=180] ++(-0.5,0.8)
  node[right] {$\Phi < 0$};



\end{tikzpicture}

  \includegraphics[]{External/lambda_mode.pdf}
  \caption{Sketch of one-dimensional standing pressure wave (blue line) in a 
      fluid cavity with width $W$, an acoustic wavelength 
      $\wavelength_{\MR{a}}$, and two types of particle; red (negative contrast 
      factor $\Phi$), gray (positive contrast factor $\Phi$). The depicted mode 
  is the \emph{lambda}-mode.}
  \label{fig:TA-lambda_mode}
\end{figure}

But for most applications, firstly, the particle radius is much larger than the 
VBL thickness $R \gg \delta$ and, secondly, the acoustic wavelength is much 
larger than the particle radius $\wavelength_{\MR{a}}\gg R \gg \delta$. The 
most common formula for those restrictions and an one-dimensional standing 
pressure wave is the derivation by \cname{Gorkov1962}. He derives the ARF on a 
particle which is much smaller than the acoustic wavelength 
($\R\ll\lambda_{\MR{a}}$). In his derivations, he models the fluid as well as 
the particle as an inviscid compressible fluid. Hence, no VBL can be formed.  
The ARF is defined as the negative gradient of a potential
\begin{equation}
  \force^{\MR{rad}}_{i} = -\pdv{x_{i}}U^{\MR{rad}}
  \label{eq:TA-gorkov}
\end{equation}
where $U^{\MR{rad}}$ is the so-called \emph{Gor'kov potential}. It is defined 
as
\begin{equation}
  U^{\MR{rad}} = \frac{4}{3}\,\pi\,R^{3}\left\{ 
  \frac{f_{1}}{2}\,\kappa_{\MR{f}}\timeaverage{p^{2}_{\MR{a}}} - 
f_{2}\,\rhof\timeaverage{v^{2}_{i,\MR{in}}} \right\}
    \label{eq:TA-gorkov-potential}
\end{equation}
where $\kappa_{\MR{f}}$ is the compressibility of the fluid and
\begin{equation}
  f_{1} = 1 - \frac{\kappa_{\MR{s}}}{\kappa_{\MR{f}}} = 1 - \tilde{\kappa}
  \label{eq:TA-fone}
\end{equation}
the so-called monopole factor with $\kappa_{\MR{s}}$ being the compressibility 
of the particle and where
\begin{equation}
f_{2} = \frac{2(\tilde{\rho}-1)}{2\tilde{\rho}+1}
\label{eq:TA-ftwo}
\end{equation}
is the so called dipole factor with $\tilde{\rho} = \sfrac{\rhop}{\rhof}$ 
\cite{Gorkov1962,Bruus2012}.

\begin{figure}
    % \tikzsetnextfilename{SC_all}

\renewcommand{\tikzHelper}{\relPath/10_Figures/TikZ/scattering_data}

\begin{tikzpicture}
\begin{groupplot}[
  group style={
    vertical sep=6mm,
    horizontal sep=3mm,
    group size= 2 by 3,
  },
  title style={yshift=-2.5mm,},
  xmin=-5,
  xmax=5,
  ymin=0,
  ymax=5,
  point meta max=75,
  point meta min=4,
  view={0}{90},
  % colorbar,
  colormap = {whiteblack}{color(0cm)  = (black);color(1cm) = (white)},
  % colormap/bluered,                     % Colormap preset
  height=4cm,
  width=75mm,
  ]

  %%%%%%%%%%%%%%%%
  \nextgroupplot[
    title={$\omega t =0\,\pi$},
    ylabel={$\sfrac{x}{R}$},
    xticklabels={,,},
]
    \addplot3[
        contour filled={number=25,labels={false}},
        mesh/rows=110,
        mesh/check=false
    ] table[x=z, y=x, z=c] {\tikzHelper/phase_0.dat};

    \coordinate (top) at (rel axis cs:0,1);
  %%%%%%%%%%%%%%%%
  \nextgroupplot[
    title={$\omega t =0.2\,\pi$},
    xticklabels={,,},
    yticklabels={,,},
]
    \addplot3[
        contour filled={number=25,labels={false}},
        mesh/rows=110,
        mesh/check=false
    ] table[x=z, y=x, z=c] {\tikzHelper/phase_1.dat};

  %%%%%%%%%%%%%%%%
  \nextgroupplot[
    title={$\omega t =0.4\,\pi$},
    ylabel={$\sfrac{x}{R}$},
    xticklabels={,,},
]
    \addplot3[
        contour filled={number=25,labels={false}},
        mesh/rows=110,
        mesh/check=false
    ] table[x=z, y=x, z=c] {\tikzHelper/phase_2.dat};

  %%%%%%%%%%%%%%%%
  \nextgroupplot[
    title={$\omega t =0.6\,\pi$},
    xticklabels={,,},
    yticklabels={,,},
]
    \addplot3[
        contour filled={number=25,labels={false}},
        mesh/rows=110,
        mesh/check=false
    ] table[x=z, y=x, z=c] {\tikzHelper/phase_3.dat};

  %%%%%%%%%%%%%%%%
  \nextgroupplot[
    title={$\omega t =0.8\,\pi$},
    ylabel={$\sfrac{x}{R}$},
    xlabel={$\sfrac{z}{R}$},
]
    \addplot3[
        contour filled={number=25,labels={false}},
        mesh/rows=110,
        mesh/check=false
    ] table[x=z, y=x, z=c] {\tikzHelper/phase_4.dat};

  %%%%%%%%%%%%%%%%
  \nextgroupplot[
    title={$\omega t =1.0\,\pi$},
    yticklabels={,,},
    xlabel={$\sfrac{z}{R}$},
]
    \addplot3[
        contour filled={number=25,labels={false}},
        mesh/rows=110,
        mesh/check=false
    ] table[x=z, y=x, z=c] {\tikzHelper/phase_5.dat};

  \coordinate (bot) at (rel axis cs:1,0);

\end{groupplot}
  \path (top-|current bounding box.east)--
                    coordinate(legendposright)
                    (bot-|current bounding box.east);


\begin{axis}[%
  hide axis,
  scale only axis,
  %height=0.975\linewidth,
  %width=0.01\linewidth,
  at={(legendposright.east)},
  anchor=east,
  xshift=60mm,
  point meta min=4.0,
  point meta max=75.0,
  % colormap/bluered,                     % Colormap preset
  colormap = {whiteblack}{color(0cm)  = (black);color(1cm) = (white)},
  colorbar right,                  % Active colorbar
  colorbar sampled,                     % Steps in colorbar
  colorbar style={
    separate axis lines,
    samples=256,                        % Number of steps
    ylabel={Acoustic Velocity $\abs{\vb{v}}$ [\si{\mm\per\second}]},
  },
  every colorbar/.append style={
    height=\pgfkeysvalueof{/pgfplots/parent axis height}%+
               %\pgfkeysvalueof{/pgfplots/group/vertical sep}
  }
]
  \addplot [draw=none] coordinates {(0,0)};
\end{axis}

\end{tikzpicture}

    % \includegraphics[]{External/SC_all.pdf}
    \includegraphics[width=160mm]{example-image-a}
    \caption{ARF solution generated with \emph{gorkov}}
  \label{fig:TA-comparision-ARF}
\end{figure}

For a plane standing wave, where the wave vector is aligned with $\ez$, (see 
\cref{fig:TA-plane_wave}) the radiation force along the $\ez$ axis can be 
computed as~\cite{Bruus2012}
\begin{equation}
  \force_{z}^{\MR{rad}} = 
  4\pi\,\Phi(\tilde{\rho},\tilde{\kappa})\,k\,R^{3}\,E_{\MR{ac}}\,\sin(2kz)
  \label{eq:TA-ARFy}
\end{equation}
where $z$ denotes the position along $ez$ and with the so called acoustic 
energy density
\begin{equation}
  E_{\MR{ac}} = \frac{p_{\MR{a}}^{2}}{4\rhof\cfl^{2}}
  \label{eq:TA-Eac}
\end{equation}
and with the so-called acoustic contrast factor
\begin{equation}
  \Phi(\tilde{\rho},\tilde{\kappa}) = \frac{1}{3}\,f_{1} + \frac{1}{2}\,f_{2}.
  \label{eq:TA-Phi}
\end{equation}

From \cref{eq:TA-ARFy} one can deduce the following interesting properties of 
the ARF for a plane standing wave: a) the ARF scales with the volume of the 
spherical particle; b) the ARF is sinusoidal with double the period of the 
incident acoustic wave; c) depending on the sign of $\Phi(\tilde{\kappa}, 
\tilde{\rho})$ the ARF changes the sign; d) particles with 
$\Phi(\tilde{\kappa}, \tilde{\rho}) = 0$ are acoustically 
$\wavelength_{\MR{a}}$
\emph{invisible} and not displaced by the ARF.

Particles with a positive acoustic contrast factor $\Phi(\tilde{\kappa}, 
\tilde{\rho}) > 0$ are displaced by the ARF towards the pressure nodes (gray 
particles in \cref{fig:TA-plane_wave}) and particles with a negative acoustic 
contrast factor $\Phi(\tilde{\kappa}, \tilde{\rho}) < 0$ are displaced towards 
the pressure antinode (red particle in \cref{fig:TA-plane_wave}).

\textcolor{red}{general solyution converges to Gorkov 
\cref{fig:TA-comparision-ARF}}


\section{Acoustic Radiation Torque\label{sec:TA-VT}}

\Cref{eq:TA-ARFy} is valid for an one-dimensional pressure wave. However, for 
the configuration of two one-dimensional pressure waves which are orthogonal to 
each other a two dimensional pressure field is formed (see 
\cref{fig:TA-viscous_torque}). With the assumption that the object dimension is 
much smaller than the acoustic wavelength of both pressure fields separately, 
the particles with positive acoustic contrast factor ( $\Phi>0$ ) will be 
displaced into the pressure nodes of the two superposing pressure waves and the 
particles with negative acoustic contrast factor in the superposed pressure 
antinodes.

\begin{figure}[tbp]
  \centering
  % \tikzsetnextfilename{viscous_torque}
{\small
% Parameters for vector
\pgfmathsetmacro{\W}{12.0}
\pgfmathsetmacro{\L}{6.0}
\pgfmathsetmacro{\B}{0.5}

\pgfmathsetmacro{\fac}{2}
\pgfmathsetmacro{\amplitude}{0.8}
\pgfmathsetmacro{\doublefac}{2*\fac}

% Syntax: [draw options] (center) (initial angle:final angle:radius)
\def\centerarc[#1](#2)(#3:#4:#5)
  { \draw[#1] ($(#2)+({#5*cos(#3)},{#5*sin(#3)})$) arc (#3:#4:#5); }

\begin{tikzpicture}[]

  \coordinate (O) at (0,0);
  \coordinate (BL) at (-\W/2,-\L/2);
  \coordinate (TR) at (\W/2,\L/2);

  % walls
  \draw[pattern=north west lines, pattern color=black!50] ($(BL) - (\B, 
  \B)$) rectangle ++(2*\B+\W,2*\B+\L);

  % fluid
  \draw[fill=blue!10] (BL) rectangle (TR);

  % coordinate system
  \draw[<->] ($(-\W/2*0.9,-\L/4)$) -- node[left,pos=0] {$\ex$} ++(0,-\L/4*0.9) 
  -- node[right,pos=1] {$\ey$} ++(\L/4,0);

  % pressure field
  \draw[blue] plot[domain=-\W/2:\W/2,samples=100] 
  (\x,{-\amplitude*cos(deg(2*pi/(\W/\fac)*\x))});
  \draw[blue] plot[domain=-\W/2:\W/2,samples=100] 
  (\x,{\amplitude*cos(deg(2*pi/(\W/\fac)*\x))});

  \draw[purple] plot[domain=-\L/2:\L/2,samples=100] 
  ({-\amplitude*cos(deg(2*pi/(\L/\fac)*\x))},\x);
  \draw[purple] plot[domain=-\L/2:\L/2,samples=100] 
  ({\amplitude*cos(deg(2*pi/(\L/\fac)*\x))},\x);

  \foreach \x in {1,2,...,\doublefac}{
    \draw[dotted] ($(-\W/2, -\L*5/8 + \x*\L/2/\fac)$) -- ++(\W,0);
    \draw[dotted] ($( -\W*5/8 + \x*\W/2/\fac,-\L/2)$) -- ++(0,\L);
  }

  % non-spherical particles
  \foreach \x in {1,2,...,\doublefac}{
  \foreach \y in {1,2,...,\doublefac}{
    \pgfmathsetmacro{\X}{-\W*5/8 + \x*\W/2/\fac}
    \pgfmathsetmacro{\Y}{-\L*5/8 + \y*\L/2/\fac}
    \pgfmathparse{360 * random()}
    \filldraw[rotate around={\pgfmathresult:(\X,\Y)}, gray, shade, 
    shading=ball, ball color=black!50!white] (\X,\Y) ellipse (2mm and 3mm);
  }
  }

  \coordinate (TL) at (-\W*3/8,\L*3/8);
  \centerarc[thick,->](TL)(30:270:6mm)
  \node at ($(TL)+(-\W/12,\L/12)$) {$\Omega\left( \zeta \right)$};


  % annotations width
  \draw[|<->|] (-\W/2, \L/2+1.5*\B) -- node[above,midway] {$W$} ++(\W,0);
  \draw[|<->|] (\W/2+1.5*\B, -\L/2) -- node[right,midway] {$L$} ++(0,\L);
  \path (-\W/2-\B,0) -- node[midway,fill=white,rounded corners] 
  {$\propto\sin(\omega t)$} ++(\B,0);

  \path (-\W/8,\L/2+\B/2) -- node[midway,fill=white,rounded corners] 
  {$\propto\sin(\omega t + \zeta)$} ++(\W/4,0);
\end{tikzpicture}
}

  \includegraphics[]{External/viscous_torque.pdf}
  \caption{Sketch of a two dimensional orthogonal pressure field in a fluid 
    cavity of width $W$ and length $L$ with phase difference $\zeta$ and 
    acoustic excitation frequencies $\omega_{1}$ and $\omega_{2}$ along $W$ and 
  $L$, respectively. Particles gather in pressure nodes ($\Phi > 0$) along 
respective direction and non-spherical particles rotate with angular velocity 
$\Omega\left( \zeta \right)$.}
  \label{fig:TA-viscous_torque}
\end{figure}

Besides the translation of the objects to the equilibrium pressure position 
some of them will also rotate due to their non-spherical shape (see 
\cref{fig:TA-viscous_torque}). There does not exist an analytical closed form 
solution to the radiation torque exerted onto the objects~\cite{Lamprecht2017}.

 \begin{figure}
  \centering
  \begin{subfigure}[b]{0.35\textwidth}
    \centering
    \caption{$\zeta = \SI{0}{\radian}$}
    % \input{\relPath/10_Figures/TikZ/viscous_torque_0.tikz}
    \includegraphics[]{External/viscous_torque_0.pdf}
    % \label{fig:TO-QPDx}
  \end{subfigure}
  \hfill
  \begin{subfigure}[b]{0.3\textwidth}
    \centering
    \caption{$\zeta = \sfrac{\pi}{4}\,\si{\radian}$}
    % \input{\relPath/10_Figures/TikZ/viscous_torque_1.tikz}
    \includegraphics[]{External/viscous_torque_1.pdf}
    % \label{fig:TO-QPDkky}
  \end{subfigure}
  \hfill
  \begin{subfigure}[b]{0.3\textwidth}
    \centering
    \caption{$\zeta = \sfrac{\pi}{2}\,\si{\radian}$}
    % \tikzsetnextfilename{viscous_torque_2}

\pgfplotsset{%
    colormap={bwr}{
      color=(blue);
      color=(white);
      color=(red);
    }%
}

\begin{tikzpicture}
  \begin{axis}[view={0}{90},
      xlabel={ $\varphi$ [rad] },
      xtick={0,3.1415,6.28},
      xticklabels={0,$\pi$,$2\,\pi$},
      height=50mm,
      width=55mm,
      % colorbar,
      % colorbar horizontal,
      % colorbar style={
        % title={\footnotesize Acoustic pressure $p_{\MR{a}}$},
        % at={(0,1.4)},
        % anchor=north west,
        % xticklabels={,,},
        % % xtick={0,0.5,1},
      % }
    ]
      \addplot3[surf,mesh/rows=50,shader=interp] table[x=phi,y=t,z=zeta2] 
      {\relPath/10_Figures/TikZ/Viscous_Torque.dat};

  \end{axis}
\end{tikzpicture}

    \includegraphics[]{External/viscous_torque_2.pdf}
    % \label{fig:TO-QPDt}
  \end{subfigure}
  \caption{Acoustic pressure $p_{\MR{a}}$ around spherical particle 
      circumference ($\varphi$) in a two dimensional pressure wave field with 
      three different phase shifts $\zeta$ plotted over one excitation period. 
      The pressure amplitude is depicted qualitatively; red equals positive 
  pressure and blue negative.}
  \label{fig:TA-VT}
 \end{figure}

Not only non-spherical objects can rotate due to the acoustics, but also 
spherical particles can rotate in an orthogonal two dimensional pressure wave 
field if two conditions are met: 1) the viscous losses inside the VBL $\delta$ 
around the particle are high enough; 2) the excitation of the two pressure 
waves is phase shifted.

If the two conditions are met, then the by $\zeta$ phase shifted excitation 
will lead to a continuous phase change of the pressure amplitude along the 
surface of the particle (see \cref{fig:TA-VT}). This phase change will lead to 
a local time-averaged streaming in the VBL of the particle that initiates the 
rotation. The final rotational velocity is when the particle accelerating 
torque form the local streaming flow equals the opposite directed Stokes' 
rotational drag torque~\cite{Lamprecht2017}. This rotational velocity is -- 
amongst other parameters -- proportional to the phase shift $\zeta$. Therefore, 
$\zeta + \sfrac{\pi}{2}$ will cause rotations opposite to a phase shift of 
$\zeta$. For neighboring superimposed pressure nodes along $\ex$ or $\ey$ the 
rotational direction is also opposite. We will discuss this kind of particle 
rotation more extensively in \cref{ch:viscoustorque}.

% \section{Python module for Acoustofluidic\label{sec:TA-python}}
