\chapter*{Preface}
I would like to express my deepest gratitude for the opportunity to conduct my 
doctoral studies in the group of Mechanics and Experimental Dynamics, Institute 
for Mechanical Systems, Department of Mechanical and Process Engineering, ETH 
Zurich in the years 2016-2021. During these five years, I had the opportunity 
to acquire a diverse skill set and pursue interdisciplinary research projects, 
which would not have been successful without the tremendous support from my 
colleagues, project partners and supervisor. I would like to thank everyone who 
helped me in these five fantastic, diverse and intense years. Following, I 
would like to highlight some persons who deserve to be mentioned by name:

\begin{itemize}

\item First of all, I would like to express my deepest gratitude to Jürg Dual. 
  Despite his hands-off supervision style, which enabled me to develop my full 
  potential, I could always count on his support and corrections in my 
  ambitions to save me from gross mistakes. Jürg is one of the best supervisors 
  I had the pleasure to get to know and one of my idols with regards to 
  interpersonal competences and my future academic career.

\item I would like to thank Thomas Laurell for accepting to co-examine my 
  dissertation and for various meetings that increased the quality of my 
  publications and gave me indispensable ideas for my future academic career.

\item Special thanks go to Thierry Baasch. Thierry not only always had an 
  answer to all work related questions but also an open ear for things that did 
  not necessarily relate to my doctoral studies. He is a great mentor and 
  friend.

\item I would also like to especially thank Nino Läubli who was the driving 
  force behind our two collaborations that we managed to render successful 
  within a short amount of time. Despite his tight schedule, Nino always took 
  the time to help me on various occasions, especially regarding paper 
  optimisation and revision.

\item Without my first project partner, Dominik Haidas, who supported me in 
  droplet microfluidics and project management related questions, I would not 
  have been able to develop skills in this area, which are indispensable for my 
  future career.

\item My special gratitude also goes to Peter Ruppen. Without Peter, one of my 
  best device ideas would have ended unnoticed inside a cupboard. With his 
  expertise in bacteria handling, we were able to transform a previously 
  unnoticed device design into a patent and publication.

\item I would like to thank my collaborators from the "Powder Focusing" 
  project: Patrik Rohner, Shahab Eghbali, Choi Kwanghoon, and S\'ebastien 
  Vaucher for the refreshing and fruitful meetings.

\item I would like to thank everyone that I had the pleasure of getting to know 
  during various conferences and inspired me on various occasions, particularly 
  the ones from our acoustofluidics community. Especially, I would like to 
  thank Rune Barnkob, who is one of my role models regarding sophisticated 
  research and being a great mentor.

\item I would like to thank the operation team of the BRNC cleanroom for the 
  help with various chrome masks and device production. Especially, I would 
  like to thank Ute Drechsler. Ute is one of the most proficient persons 
  regarding clean room production I have ever met. With her limitless effort, 
  she manages to support all cleanroom users within their projects, keeps the 
  machines in BRNC up and running, and pursues her own projects for IBM, 
  simultaneously. Ute is one of my role models regarding dedication and passion 
  for her work.

\item The help regarding administrative tasks from Beate Fonf\'e, clean room 
  fabrication from Donat Scheiwiler and Stafan Blunier, small adjustments with 
  high impact at my experimental setups form Jean-Claude Thomasina, and 
  electronic equipment form Bernhard Zybach layed the basis for the success of 
  my experimental work. Thank you very much Beate, Donat, Stefan, Jean-Claude, 
  and Bernhard!

\item I would like to thank Anja Huss from Thorlabs GmbH. Together with here 
  competent support, we have been able to design three highly personalised 
  experimental setups, essential for the success of all presented projects in 
  this thesis.

\item I would like to thank my colleagues in the group of Mechanics and 
  Experimental Dynamic, Ivo Leibacher, Peter Reichert, Tobias Brack, Valentin 
  van Gemmeren, Dhananjay Deshmukh, Christoph Goering, Jonas Fankhauser, Cooper 
  Harshbarger and Alen Pavlic for their continuous support.

\item I would like to thank all students that I had the pleasure to supervise. 
  Without your work, I would not have been able to pursue so many ideas 
  simultaneously. Especially, I would like to thank Aurelia Bucciarelli, Marco 
  Hoseneder, Moritz Leuthner, Michel Manser, Alexandre Ratschat and Philipp 
  Suter who contributed significantly to my publications.

\item Last but not least, I would like to thank my family and friends, who 
  rendered the time during my doctoral studies even more enjoyable.

\end{itemize}

Even though doctoral theses tend to be forgotten after some years, I hope that 
I could make a significant contribution in the research area of acosutofluidics 
and show that there is still plenty of room for further improvements. My 
deepest desire is to use my knowledge for the development of novel applications 
that can tackle the great challenges that our generation has to face.

\begin{flushleft}
\begin{figure}[h]
\begin{flushleft}
 % \hspace{1 cm} \includegraphics[width=0.25\textwidth]{Unterschrift.png}
\end{flushleft}
\end{figure}
\vspace{-0.1 cm}
\hspace{1 cm} Michael Gerlt\\
\hspace{1 cm} Zürich, June 2021
\end{flushleft}
