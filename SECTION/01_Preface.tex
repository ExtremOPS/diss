\chapter*{Preface}

Although only my (long) name is written at the beginning of this thesis and 
although I really (!) did write and produce the content all by myself, there 
are a several persons that had a substantial contribution such that I could 
successfully make it to the defense of my work -- and if you are reading it, it 
probably also means that I passed the doctoral examination -- and successfully 
complete my doctoral studies at the ETH in Zurich with Prof. Dr. Jürg Dual.

Amongst many persons that helped in one way or the other, I will mention some 
of them. If your name is missing and you think that it should also belong in 
this list, sincere apologies to you, take the credit, and add an item to the 
list by hand.

% \begin{itemize}[label=$\multimap$]
% \begin{itemize}[label=$\bowtie$]
\begin{itemize}[label=$\succ$]

  \item First and foremost, I thank Prof. Dr. Jürg Dual very much for giving me 
    the opportunity to pursue my doctoral studies in Zurich at the ETH even 
    though my Bachelor and Master was neither at ETH, nor in one of his main 
    research areas. I am not sure, if I would have taken this uncertainty. I 
    enjoyed the freedom he has given me during my research and valued every 
    feedback from him. Besides his exceptional knowledge in various -- 
    sometimes unexpected -- areas of science, I valued his engagement in the 
    extra-curricular activities within his group or even with all groups of the 
    whole institute.

  \item Secondly, I thank Prof. Dr. Daniel Ahmed for saying \emph{yes} to my 
    request of being the second supervisor; \emph{onek dhonnobad}. As it is 
    with Jürg, I enjoyed our talks because they were not solely about the 
    research questions at hand, but also about more personal matters. I wish 
    you all the best for your future academic career.

  \item Special thanks is also well placed towards my co-workers which not only 
  helped me with the challenges I faced during my experiments but also created 
  a working environment where laughter was welcomed and created at anytime. 
  Without you I would not have had such a great time during my doctorate. 
  Unfortunately, due to Corona we had really only one conference outing 
  together.

  \item Jürg's Forschungsgruppe besteht nicht nur aus den DoktorandInnen, 
    sondern auch aus Personen, welche dafür sorgen, dass wir unbesorgt unserer 
    Arbeit nachgehen können. Im besonderen bedanke ich mich bei Beate 
    Fonf$\acute{e}$, Martina Koch, Bernhard Zybach, Jean-Claude Tomasina und 
    Donat Scheiwiller für ihre Arbeit, die vor allem im Hintergrund statt 
    gefunden hat. Ihr habt sichergestellt, dass ich mich wirklich 
    ausschliesslich auf meine Experimente konzentrieren konnte, weil ich 
    wusste, dass ihr die anderen Komplikationen abdeckt.

  \item Dr. Andreas Lamprecht ist mein Vorgänger an der Optical Trap. Obwohl 
    wir -6 Monate Überlappung hatten, haben wir es geschafft, möglichst viel 
    Wissen von dir auf mich zu übertragen. Ich bin dir sehr dankbar dafür, dass 
    du dir hin und wieder die Zeit genommen hattest, um mir einerseits Dinge 
    vor Ort an der ETH zu erklären und andererseits Ideen für mögliche 
    Experimente zu geben, die du -- zum Glück oder auch leider -- nicht mehr 
    geschafft hattest. Es ist keine Übertreibung zu sagen, dass ohne dich 
    Kapitel 4 bis 6 nicht in dieser Art zu Stande gekommen wären.

  \item Ich bedanke mich bei Dr.-Ing. Thomas Zehelein für die Zeit, die er sich 
    während seiner Ferien genommen hat, um einen sehr frühen Entwurf meiner 
    Arbeit zu lesen. Das Feedback, dass ich von dir bekommen habe, war sehr 
    hilfreich, weil es einige \emph{blinde Flecken} offen gelegt hat. Ich bin 
    davon überzeugt, dass durch dein konstruktives Lesen, die Arbeit für ein 
    breiteres Publikum zugänglicher wurde.

  \item Unter den StudentInnen, die ich betreut habe möchte ich Giulia Zobrist 
    hervorheben. Ich bedanke mich bei dir nicht nur für die beiden 
    erfolgreichen Arbeiten, die du bei und mit uns gemacht hast, sondern vor 
    allem auch für dein gutes und kritisches Feedback zu unterschiedlichen 
    Entwürfen von meinen Manuskripten.

  \item Nicht zu vergessen ist meine liebe Mutter; sie ist die beste Mutter, 
    die ich habe. Ich bin überglücklich, dass sie die meine ist und ich 
    bewundere sie für das meiste, was sie macht und tut. Wäre nur jeder halb so 
    gütig, offenherzig, und nachsichtig, würden sich viele Probleme von selbst 
    lösen. Du bist und warst immer genug!

  \item Nicht zu letzt, bedanke ich mich bei meiner Exfreundin, Anna-Maria. Sie 
  ist nicht nur der Grund, warum wir überhaupt in die Schweiz gekommen sind, 
  sondern auch einer der Gründe, warum ich mich wirklich hier heimisch fühle. 
  Du hast mir nicht nur geholfen, als es Mal nicht so mit der Forschung lief, 
  sondern du hast für jegliche Themen ein offenes Ohr für mich; du nimmst mich 
  (meistens) ernst und hinterfragst meine Ideen und Vorschläge kritisch und 
  konstruktiv. Ich würde dich jederzeit wieder heiraten.

  \item

\end{itemize}

\vspace*{\fill}

\begin{flushleft}
\begin{figure}[h]
\begin{flushleft}
 \hspace{1 cm}
 \includegraphics[height=3cm]{Unterschrift.png}
\end{flushleft}
\end{figure}
\vspace{-0.1 cm}
\hspace{1 cm} Christoph $(\ldots)$ Goering\\
\hspace{1 cm} Baar, May 2022
\end{flushleft}
