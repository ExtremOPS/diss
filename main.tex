\documentclass[
  fontsize=13pt,
  a4paper,
  chapterprefix,
    ]{scrbook}

\usepackage{emptypage}

\usepackage{geometry}
\geometry{a4paper,top=30mm,left=22mm,right=22mm,bottom=30mm,headsep=10mm,footskip=12mm}
\renewcommand{\baselinestretch}{1.05}\normalsize %Zeilenabstand
%vor Boston: 1.1 Zeilenabstand, Grösse 13 pt top=30mm,left=18mm,right=18mm,bottom=30mm,headsep=10mm,footskip=12mm

%distance between floats on the top or the bottom and the text, standard: 20.0pt plus 2.0pt minus 4.0pt
\setlength{\textfloatsep}{7pt plus 1.0pt minus 2.0pt}
% distance between two floats, standard: 12.0pt plus 2.0pt minus 2.0pt:
\setlength{\floatsep}{7pt plus 1.0pt minus 2.0pt}
%distance between floats inserted inside the page text (using h) and the text, standard: 12.0pt plus 2.0pt minus 2.0pt:
\setlength{\intextsep}{7pt plus 1.0pt minus 2.0pt}

\usepackage{indentfirst} %Einzug bei jedem Abschnitt

\usepackage{amsmath}
\usepackage{amssymb}
\usepackage{xfrac}

\usepackage[colorlinks = true,
            linkcolor = blue,
            urlcolor  = blue,
            citecolor = blue,
            anchorcolor = blue]{hyperref}
% must be after hyperref and amsmath
\usepackage[nameinlink]{cleveref}
  \crefname{figure}{FIG.}{FIGs.}
  \crefname{table}{TAB.}{TABs.}
  \crefname{equation}{EQ.}{EQs.}
  \crefname{section}{Section}{Section}

\usepackage{floatflt}
\usepackage{moresize}
\usepackage[]{ethimes}               % New styles and commands
\setcounter{tocdepth}{3} \setcounter{secnumdepth}{3}
\usepackage{fancyhdr}                % Headings
\usepackage{graphicx}                % EPS figures
  \graphicspath{
      {SECTION/30_Timeconstant/10_Figures/LaTeX/}
      {SECTION/30_Timeconstant/10_Figures/PGF/}
      {SECTION/30_Timeconstant/10_Figures/}
    }
\usepackage[dvips]{epsfig}           % EPS figures
\usepackage{epstopdf}
\usepackage{float}                   % Placement of floating objects
\usepackage{here}                   % Placement of floating objects
\usepackage{lipsum}                   %um eine Fussnote zu platzieren ohne Nr
\usepackage[ngerman, spanish, english]{babel}           %für Gänsefüsschen etc
\usepackage[utf8]{inputenc}\DeclareUnicodeCharacter{2212}{-}
\usepackage[T1]{fontenc}
\usepackage{siunitx} %SI units

\usepackage{rotating} % Rotating table
\usepackage{booktabs}
  \newcolumntype{x}[1]{>{\centering\arraybackslash\hspace{0pt}}p{#1}}
\usepackage{array}
\usepackage[table,xcdraw]{xcolor}
\usepackage{tabularx}

% \usepackage{cite}                    % bei Referenzen [15,12,11,10] sortiert es zu [10-12, 15]
\usepackage[
  backend=biber,
  url=false,
    ]{biblatex}
\addbibresource{All.bib}
% \usepackage[square,numbers]{biblatex}
\usepackage{bm}

\newcommand\blfootnote[1]{%
  \begingroup
  \renewcommand\thefootnote{}\footnote{#1}%
  \addtocounter{footnote}{-1}%
  \endgroup
}

\usepackage[
  textformat=simple,
  labelfont=bf,
  format=hang]{caption}  %Caption options
\usepackage{afterpage}
\usepackage{subcaption}


\usepackage{fix-cm}
\usepackage{titlesec}
% \titleformat{\chapter}[display]
% {\bfseries\LARGE}
% {\vspace{-4ex}\selectfont\color{lightgray}\hfill
%   \Large\textrm{\chaptertitlename}
%   \fontsize{85}{70}\textbf{\thechapter}}
% {4ex}
% {}
% [\vspace{1ex}\titlerule]

\colorlet{chaptercolor}{blue!80!black}

\setkomafont{chapter}{\normalfont\color{chaptercolor}\huge}
\setkomafont{chapterprefix}{\Large}
\renewcommand*{\raggedchapter}{\raggedleft}
\renewcommand*{\chapterformat}{%
  \MakeUppercase{\chapappifchapterprefix{}}%
  \rlap{\enskip\resizebox{!}{1.2cm}{\thechapter} \rule{15cm}{1.2cm} }%
}
\RedeclareSectionCommand[beforeskip=30pt,afterskip=20pt]{chapter}
\renewcommand*\chapterheadmidvskip{\par\nobreak\vspace{10pt}}

\titleformat{\section}
{\normalfont\bfseries\Large}
{\thesection.}{0.5em}{}

\titleformat{\subsection}
{\normalfont\bfseries\normalsize}
{\thesubsection.}{0.5em}{}

\usepackage{fancyhdr}

\fancypagestyle{plain}{%
  \fancyhf{}%
  \fancyfoot[OR]{\textbf{\thepage}}%
  \renewcommand{\headrulewidth}{0pt}% Line at the header invisible
  \renewcommand{\footrulewidth}{0pt}% Line at the footer visible
}

\pagestyle{fancy}
\fancyfoot{}
\fancyhead{}
\renewcommand{\chaptermark}[1]{
  \markboth{\chaptername\ \thechapter.\ #1}{}}
\renewcommand{\sectionmark}[1]{
  \markright{\thesection.\ #1}}

\fancyhead[LE]{\bfseries\leftmark}
\fancyhead[RO]{\bfseries\rightmark}

\fancyfoot[LE,RO]{\textbf{\thepage}}
\renewcommand{\headrulewidth}{0.3pt}

\usepackage{tikz}
  \usetikzlibrary{external}
\tikzset{external/force remake}
\tikzset{external/system call={
  pdflatex \tikzexternalcheckshellescape -halt-on-error
  -interaction=batchmode -jobname "\image" "\texsource" && % or ;
  pdftops -eps "\image".pdf}}
  \tikzexternalize[prefix=Plots/cache/]
  % \tikzsetexternalprefix{Plots/Cache/}

\usepackage{pgfplots}
  % \usepgfplotslibrary{external}
  % \tikzexternalize
  \usepgfplotslibrary{groupplots}

\pgfplotsset{
  compat = 1.18,
  cycle list = {
    {black}, {lightgray},
    {black, densely dashed}, {lightgray, densely dashed},
    {black, dotted}, {lightgray, dotted},
    {black, loosely dotted}, {lightgray, loosely dotted}
  },
  width = 12cm,
  height = 9cm,
  every axis/.append style = {
    line width = 0.2mm,
    tick style = {line width = 0.2mm}
  },
  every axis plot/.append style = {
    line width = 0.4mm
  },
  enlarge x limits = false,
  enlarge y limits = {value = 0.05, auto},
  every axis legend/.append style = {
    nodes = right
  },
  xticklabel style={/pgf/number format/fixed},
  yticklabel style={/pgf/number format/fixed},
}

\makeatletter
\pgfplotsset{
/pgfplots/row step/.style={
/pgfplots/x filter/.append code={
        \ifnum\coordindex=0
                \def\c@pgfplots@eachnthpoint@xfilter{0}
                \edef\c@pgfplots@eachnthpoint@xfilter@cmp{#1}
        \else
                \pgfplotsutil@advancestringcounter\c@pgfplots@eachnthpoint@xfilter
                \ifx\c@pgfplots@eachnthpoint@xfilter@cmp\c@pgfplots@eachnthpoint@xfilter
                        \def\c@pgfplots@eachnthpoint@xfilter{0}
                \else
                        \let\pgfmathresult\pgfutil@empty
                \fi
        \fi
}
},
}
\makeatother
\pgfplotsset{filter discard warning=false}

 % Kapitel-Header-Stil
\newcommand{\relPath}{}
\newcommand{\cname}[1]{\citeauthor{#1}~\cite{#1}}
\newcommand{\cyear}[1]{In \citeyear{#1}, \cname{#1}}

\def\svginput{\input}

\newcommand{\figWidth}{80mm}
\newcommand{\figWidthDouble}{160mm}
\newcommand{\subfigWidth}{75mm}


\newcommand{\tikzHelper}{a}
\newcommand{\tmp}{a}

\newcommand{\SiO}{SiO$_{2}$}

\newcommand{\comsol}[1]{({\ttfamily #1})}

\newcommand{\conj}[1]{#1^{\ast}}

\newcommand{\pertubationorder}[2]{
  #1^{(#2)}
}

\newcommand{\zeOrder}[1]{
  \pertubationorder{#1}{0}
}

\newcommand{\stOrder}[1]{
  \pertubationorder{#1}{1}
}

\newcommand{\ndOrder}[1]{
  \pertubationorder{#1}{2}
}

\newcommand{\MR}[1]{\mathrm{#1}}

\newcommand{\bessel}[2]{j_{#1}\left(#2\right)}
\newcommand{\dbessel}[2]{j^{\prime}_{#1}\left(#2\right)}
\newcommand{\hankel}[2]{h_{#1}\left(#2\right)}
\newcommand{\dhankel}[2]{h^{\prime}_{#1}\left(#2\right)}
\newcommand{\legendre}[2]{P_{#1}\left(#2\right)}

\newcommand{\xv}{x_{\MR{v}}}

%%%%%%%%%%%%%%%%%%%%
%%%% MATH Commands
%%%%%%%%%%%%%%%%%%%%
\newcommand{\stress}{\sigma_{ij}}
\newcommand{\stresstensor}{\bm{\underline{\underline{\sigma}}}}

\newcommand{\intensity}{I}

\newcommand{\wavenumber}{k}
\newcommand{\wavevector}{\vb{\wavenumber}}

\newcommand{\fresnelR}{\mathcal{R}}
\newcommand{\fresnelT}{\mathcal{T}}
\newcommand{\fresnelr}{r}
\newcommand{\fresnelt}{t}

\newcommand{\cNA}{C_{\MR{NA}}}

\newcommand{\incident}{\theta_{\MR{i}}}
\newcommand{\reflected}{\theta_{\MR{r}}}
\newcommand{\transmitted}{\theta_{\MR{t}}}

\newcommand{\lbeta}{\beta}

\newcommand{\vbl}{\delta}

\newcommand{\normalized}[1]{\tilde{#1}}

\newcommand{\refraction}{n}
\newcommand{\nreal}{\refraction^{\prime}}
\newcommand{\nimag}{\refraction^{\prime\prime}}

\newcommand{\nf}{\refraction_{\MR{f}}}
\newcommand{\ns}{\refraction_{\MR{s}}}


\newcommand{\normal}{n_{i}}
\newcommand{\normalvector}{\vb{n}}

\newcommand{\speed}{c}

\newcommand{\pre}{p}
\newcommand{\velpotential}{\phi}

\newcommand{\vel}{v}
\newcommand{\velvector}{\vb{\vel}}

\newcommand{\power}[2]{P_{\MR{#1}}^{(#2)}}

\newcommand{\timeaverage}[1]{\left\langle #1 \right\rangle}

\newcommand{\force}{F}
\newcommand{\forcevector}{\vb{\force}}

\newcommand{\FARF}{\forcevector^{\MR{rad}}}
\newcommand{\FAS}{\forcevector^{\MR{drag}}}

\newcommand{\DV}{\Delta V}
\newcommand{\Dx}{\Delta x}
\newcommand{\Dy}{\Delta y}
\newcommand{\Dz}{\Delta z}

\newcommand{\fs}{f_{\MR{s}}}

\newcommand{\Dfour}{\SI{4.39}{\micro\meter}}
\newcommand{\Dtwo}{\SI{2.06}{\micro\meter}}
\newcommand{\R}{R}
\newcommand{\Rprime}{R^{\prime}}
\newcommand{\Rtwo}{R_{2}}
\newcommand{\Rfour}{R_{4}}

\newcommand{\RA}{R_{\MR{QPD}}}

\newcommand{\Dt}{\Delta t}

\newcommand{\ex}{\vb{e}_{x}}
\newcommand{\ey}{\vb{e}_{y}}
\newcommand{\ez}{\vb{e}_{z}}

\newcommand{\fex}{f_{\MR{ex}}}

\newcommand{\wavelength}{\lambda}

\newcommand{\lap}{\wavelength_{\MR{p}}}
\newcommand{\laF}{\wavelength_{\MR{F}}}
\newcommand{\lal}{\wavelength_{\MR{l}}}

\newcommand{\viscosity}{\mu}
\newcommand{\density}{\rho}
\newcommand{\kinematicviscosity}{\nu}

% fluid properties
\newcommand{\cfl}{c_{\MR{f}}}
\newcommand{\muef}{\viscosity_{\MR{f}}}
\newcommand{\rhof}{\density_{\MR{f}}}

\newcommand{\rhop}{\density_{\MR{p}}}

\newcommand{\timeconstant}[1]{\tau_{\MR{#1}}}

\newcommand{\tdrag}{\timeconstant{drag}}
\newcommand{\tOT}{\timeconstant{OT}}
\newcommand{\tas}{\timeconstant{AS}}
\newcommand{\tarf}{\timeconstant{ARF}}
\newcommand{\tqpd}{\timeconstant{QPD}}
\newcommand{\tiner}{\timeconstant{iner}}

% nomarlized measures
\newcommand{\tkappa}{\tilde{\kappa}}
\newcommand{\trho}{\tilde{\rho}}
\newcommand{\tdelta}{\tilde{\delta}}
% imaginary unit
\newcommand{\iu}{\MR{i}\mkern1mu}

\newcommand{\um}{\si{\micro\meter}}
\newcommand{\rpm}{rpm}

\newcommand{\normBdLayer}{\tilde{\delta}}
\newcommand{\finalOmega}{\Omega_{\MR{equ}}}

\newcommand{\Rcrit}{\R_{\mathrm{crit}}}
\newcommand{\fp}{f_{\mathrm{p}}}
\newcommand{\cf}{\cfl}
% nomarlized measures
\newcommand{\llaser}{\lal}
 % Kapitel-Header-Stil

\renewcommand{\\}{\par}
%damit wird jeder Abschnitt zum Paragraph, wodurch man unten mittels
%\setlength{\parindent}{5mm} einen Einzug definieren kann nach jedem "\\"

\begin{document}

\hyphenation{ acou-sto-pho-re-sis re-so-na-tors wave-length quar-ter}
\pagenumbering{roman}
\pagestyle{fancy}                   % Special headings

\setlength{\parindent}{4mm}       % Paragraph Einzug am Anfang jedes Paragraphen

\begin{titlepage}
{Diss. ETH No. tbd. \vspace{2.5cm}}
\begin{center}
%\LARGE{\textbf{Acoustophoresis for lab-on-a-chip systems}}
\LARGE{\textbf{Pushing the Limits of Acoustofluidics for Single Cell Analysis, 
Bacteria Transformation and Metal 3D Printing}}
\end{center}
\vspace{2.0cm}
\begin{center}
{A thesis submitted to attain the degree of}
\end{center}
\begin{center}
{\textsc{Doctor of Sciences of ETH Zurich}}
%{SWISS FEDERAL INSTITUTE OF TECHNOLOGY ZURICH}
\end{center}
\begin{center}
{(Dr. sc. ETH Zurich)} \end{center}
\vspace{2.0cm}
\begin{center}
{presented by}
\end{center}
\begin{center}
  {\textsc{\underline{Christoph} Ludwig Georg Ananda Goering}}
\end{center}
\begin{center}
%{MSc\ ETH \ ME \\
{M.Sc. (TUM) in mechanicla engineering,\\
Technische Universit\"at M\"unchen\\
born on 17.08.1991 \\
citizen of Germany}
\end{center}
\vspace{1cm}
\begin{center}
{accepted on the recommendation of \\ \vspace{0.3cm}
Prof.\ Dr. J\"urg Dual, examiner \\
Prof.\ Dr. Daniel Ahmed, co-examiner}
\end{center}
\vspace{1cm}
\begin{center}
2022
\end{center}

\cleardoublepage
\thispagestyle{empty}
\vspace*{5.0cm}

%\begin{flushright}
\begin{center}
\vspace*{0.5cm}
\textit{"Mehr als die Vergangenheit interessiert mich die Zukunft,\\ denn in 
ihr gedenke ich zu leben." \\}
\vspace*{0.5cm}
Albert Einstein\\
\vspace*{2cm}

Ich widme diese Dissertation \\
meinen Eltern Heike und Klaus sowie meiner Verlobten Rebecca,\\
die immer für mich da sind und an meiner Seite stehen.


\end{center}
%\end{flushright}


\cleardoublepage
\end{titlepage}

\chapter*{Preface}

Although only my (long) name is written at the beginning of this thesis and 
although I really (!) did write and produce the content all by myself, there 
are a several persons that had a substantial contribution such that I could 
successfully make it to the defense of my work -- and if you are reading it, it 
probably also means that I passed the doctoral examination -- and successfully 
complete my doctoral studies at the ETH in Zurich with Prof. Dr. Jürg Dual.

Amongst many persons that helped in one way or the other, I will mention some 
of them. If your name is missing and you think that it should also belong in 
this list, sincere apologies to you, take the credit, and add an item to the 
list by hand.

% \begin{itemize}[label=$\multimap$]
% \begin{itemize}[label=$\bowtie$]
\begin{itemize}[label=$\succ$]

  \item First and foremost, I thank Prof. Dr. Jürg Dual very much for giving me 
    the opportunity to pursue my doctoral studies in Zurich at the ETH even 
    though my Bachelor and Master was neither at ETH, nor in one of his main 
    research areas. I am not sure, if I would have taken this uncertainty. I 
    enjoyed the freedom he has given me during my research and valued every 
    feedback from him. Besides his exceptional knowledge in various -- 
    sometimes unexpected -- areas of science, I valued his engagement in the 
    extra-curricular activities within his group or even with all groups of the 
    whole institute.

  \item Secondly, I thank Prof. Dr. Daniel Ahmed for saying \emph{yes} to my 
    request of being the second supervisor; \emph{onek dhonnobad}. As it is 
    with Jürg, I enjoyed our talks because they were not solely about the 
    research questions at hand, but also about more personal matters. I wish 
    you all the best for your future academic career.

  \item Special thanks is also well placed towards my co-workers which not only 
  helped me with the challenges I faced during my experiments but also created 
  a working environment where laughter was welcomed and created at anytime. 
  Without you I would not have had such a great time during my doctorate. 
  Unfortunately, due to Corona we had really only one conference outing 
  together.

  \item Jürg's Forschungsgruppe besteht nicht nur aus den DoktorandInnen, 
    sondern auch aus Personen, welche dafür sorgen, dass wir unbesorgt unserer 
    Arbeit nachgehen können. Im besonderen bedanke ich mich bei Beate 
    Fonf$\acute{e}$, Martina Koch, Bernhard Zybach, Jean-Claude Tomasina und 
    Donat Scheiwiller für ihre Arbeit, die vor allem im Hintergrund statt 
    gefunden hat. Ihr habt sichergestellt, dass ich mich wirklich 
    ausschliesslich auf meine Experimente konzentrieren konnte, weil ich 
    wusste, dass ihr die anderen Komplikationen abdeckt.

  \item Dr. Andreas Lamprecht ist mein Vorgänger an der Optical Trap. Obwohl 
    wir -6 Monate Überlappung hatten, haben wir es geschafft, möglichst viel 
    Wissen von dir auf mich zu übertragen. Ich bin dir sehr dankbar dafür, dass 
    du dir hin und wieder die Zeit genommen hattest, um mir einerseits Dinge 
    vor Ort an der ETH zu erklären und andererseits Ideen für mögliche 
    Experimente zu geben, die du -- zum Glück oder auch leider -- nicht mehr 
    geschafft hattest. Es ist keine Übertreibung zu sagen, dass ohne dich 
    Kapitel 4 bis 6 nicht in dieser Art zu Stande gekommen wären.

  \item Ich bedanke mich bei Dr.-Ing. Thomas Zehelein für die Zeit, die er sich 
    während seiner Ferien genommen hat, um einen sehr frühen Entwurf meiner 
    Arbeit zu lesen. Das Feedback, dass ich von dir bekommen habe, war sehr 
    hilfreich, weil es einige \emph{blinde Flecken} offen gelegt hat. Ich bin 
    davon überzeugt, dass durch dein konstruktives Lesen, die Arbeit für ein 
    breiteres Publikum zugänglicher wurde.

  \item Unter den StudentInnen, die ich betreut habe möchte ich Giulia Zobrist 
    hervorheben. Ich bedanke mich bei dir nicht nur für die beiden 
    erfolgreichen Arbeiten, die du bei und mit uns gemacht hast, sondern vor 
    allem auch für dein gutes und kritisches Feedback zu unterschiedlichen 
    Entwürfen von meinen Manuskripten.

  \item Nicht zu vergessen ist meine liebe Mutter; sie ist die beste Mutter, 
    die ich habe. Ich bin überglücklich, dass sie die meine ist und ich 
    bewundere sie für das meiste, was sie macht und tut. Wäre nur jeder halb so 
    gütig, offenherzig, und nachsichtig, würden sich viele Probleme von selbst 
    lösen. Du bist und warst immer genug!

  \item Nicht zu letzt, bedanke ich mich bei meiner Exfreundin, Anna-Maria. Sie 
  ist nicht nur der Grund, warum wir überhaupt in die Schweiz gekommen sind, 
  sondern auch einer der Gründe, warum ich mich wirklich hier heimisch fühle. 
  Du hast mir nicht nur geholfen, als es Mal nicht so mit der Forschung lief, 
  sondern du hast für jegliche Themen ein offenes Ohr für mich; du nimmst mich 
  (meistens) ernst und hinterfragst meine Ideen und Vorschläge kritisch und 
  konstruktiv. Ich würde dich jederzeit wieder heiraten.

  \item

\end{itemize}

\vspace*{\fill}

\begin{flushleft}
\begin{figure}[h]
\begin{flushleft}
 \hspace{1 cm}
 \includegraphics[height=3cm]{Unterschrift.png}
\end{flushleft}
\end{figure}
\vspace{-0.1 cm}
\hspace{1 cm} Christoph $(\ldots)$ Goering\\
\hspace{1 cm} Baar, May 2022
\end{flushleft}


\tableofcontents

\cleardoublepage
\pagenumbering{arabic}
\chapter*{Abstract} \markboth{Abstract}{Abstract}
\addcontentsline{toc}{chapter}{Abstract}
%Abstract structure: background – activity/purpose – methods – results – 
%conclusion (BAMRC)

A typical channel within a micro-scale acoustofluidic (MSAF) device has a small 
cross-section where the height is usually less than \SI{200}{\um} and the width 
less than \SI{5}{\mm}; the length can be up to several \si{\cm}, however, often 
is not of big interest. Direct measurement of, e.g., the pressure produced by 
the acoustic excitation is impossible due to the smallness of the region of 
interest. Additionally, the acoustic driving frequencies are generally above 
\SI{100}{\kilo\hertz} such that the time resolution of measurements must be 
even at least twice as high if one is also interested in the transient 
behaviour and build up of, e.g., the acoustic pressure field and not only the 
steady-state.

At the moment, the most common and straightforward way to approximate the 
acoustic pressure within the channel is to optically measure the velocity of 
several objects of known size and material properties and then calculate back 
which pressure would have led to this velocity. The validity and correctness of 
the pressure approximation depends on several uncertainties. Besides the object 
dimensions and the material parameters of the object and the fluid, the biggest 
uncertainty is the validity of the underlying theory of the acoustic radiation 
force (ARF) that is used for the calculation. There exist many MSAF models for the 
calculation of the acoustic forces which differ mainly in the assumptions 
regarding the physical model for the fluid and the immersed object. Each theory 
has its parameter space where it is superior to the others because it includes, 
e.g., the effect of visco-elasticity of the fluid.

Here, an optical trapping (OT) apparatus is utilized to investigate two 
phenomena where controversies exist in the MSAF community: 1) the transient 
build up of the ARF and the drag force from acoustic streaming (AS) for a 
continuous and pulsed acoustic excitation; 2) the quantification of the 
steady-state rotational velocity of a spherical particle driven by the acoustic 
viscous torque where the viscous boundary layer (VBL) thickness is comparable 
to the particle radius itself.

So far, OTs have mainly been used as force sensors on single particles within 
MSAF devices. For the measurements of both phenomena we take advantage of the 
fine spatial and temporal resolution that the OT offers, as well as the OT 
property that single particle measurements are possible.

In order to measure the build up of the ARF and AS, an acoustic excitation 
frequency and measurement location within the standing pressure wave was used 
where the two forces were orthogonal to each other. The orthogonality as well 
as the division into ARF and drag force from AS was measured and validated by 
force measurements with the OT throughout the fluid channel with differently 
sized particles.

The results of a continuous excitation showed that the ARF starts to build up 
almost instantaneously after the acoustic excitation was switched on, whereas 
the AS takes significantly longer. Interestingly, the fast ARF build up was 
expected from theoretical considerations, but the slow AS build up was 
underestimated by a factor of about 4. The pulsed excitation experiments 
revealed that depending on the specific pulse parameters the build up of AS can 
be suppressed substantially while the ARF is not affected as much as AS. 
Therefore, smaller particles can still be mainly manipulated by the ARF because 
the relative importance of AS decreases for a pulsed excitation faster than for 
the ARF. Our measurements strengthen experimental findings for a pulsed 
excitation that could not yet be explained theoretically.

For the steady-state rotational speed measurement, a high viscosity mixture of 
water with glycerol (7 to 3) was created such that the formed VBL around the 
particle was about the same as the particle radius. The phase difference 
between two acoustic excitation sources spatially orthogonal to each other led 
to a time-averaged acoustic streaming field in the VBL of the particle along 
its circumference. This streaming field creates a driving viscous torque that 
causes a rotation with the rotational velocity at which the driving viscous 
torque equals the counteracting viscous drag torque.

A theoretical formula overestimates the steady-state rotational speed for the 
experimental parameters by more than one order of magnitude. This was expected 
because, up to now, there are no theories that are valid for the regime where 
the radius is the same size or smaller than the VBL. However, a numerical study 
investigated exactly this regime and proposed a calculation for the final 
rotational velocity including the effects of the VBL. The rotational velocities 
measured with the OT confirmed two points: 1) the expected invalidity of the 
simplified theory in the regime of high viscosity (VBL in the same order of 
magnitude as the particle dimension) and, hence, the necessity of its inclusion 
in the calculations; 2) the correctness of the numerical results.


\chapter*{Zusammenfassung}
\markboth{Zusammenfassung}{Zusammenfassung}
\addcontentsline{toc}{chapter}{Zusammenfassung}



Die typischen Dimensionen eines Flüssigkeitskanals für eine mikroskalierte 
akustofluidische (engl: \emph{micro-scale acoustofluidic} -- MSAF) Anwendung 
sind eine Höhe von weniger als \SI{200}{\um}, eine Breite von weniger als 
\SI{5}{\mm} und eine Länge von mehreren \si{\cm}, wobei die Länge nicht von 
grosser Wichtigkeit ist.  Die direkte Messung von physikalischen Grössen, wie 
zum Beispiel, dem Druck generiert durch die akustische Anregung ist unmöglich, 
da der Bereich der Messung klein ist.  Ausserdem sind die akustischen 
Frequenzen generell grösser als \SI{100}{\kilo\hertz}, so dass die Messfrequenz 
mindestens doppelt so hoch sein muss, wenn unter anderem auch das transiente 
Verhalten und der Aufbau des akustischen Drucks von Interesse ist und nicht nur 
dessen eingeschwungener Zustand.

Bis jetzt ist die häufigste und einfachste Methode den akustischen Druck 
innerhalb des Fluidkanals abzuschätzen, die Geschwindigkeiten von mehreren 
Objekten mit bekannten Dimensionen und Materialparametern optisch zu messen und 
dann anhand der Geschwindigkeiten auf den herrschenden akustischen Druck 
zurückzurechnen. Die Validität und Richtigkeit der Druckabschätzung hängt von 
mehreren Unsicherheiten ab. Neben den Dimensionen und den Materialparametern 
des Objekts und den Materialparametern des Fluides ist die grösste Unsicherheit 
die Validität der angewendeten Theorie für die Berechnung der akustischen 
Strahlungskraft (engl: \emph{acoustic radiation force} -- ARF). Es gibt viele 
Modelle für die Berechnung der ARF in der MSAF, welche sich grösstenteils bei 
der Annahme des physikalischen Modells für das Objekt und das Fluid 
unterscheiden. Jede Theorie für sich hat einen Parameterbereich, wo sie genauer 
ist als die anderen, weil sie zum Beispiel die Viskoelastizität des Fluides 
berücksichtigt.

In dieser Arbeit wird eine optische Falle (engl. \emph{optical trap} -- OT) 
verwendet, um zwei MSAF Phänomene zu untersuchen, bei welchen es ungeklärte 
Kontroversen gibt: 1) der transiente Aufbau der ARF und der Widerstandskraft 
aufgrund einer akustischen Strömung (engl: \emph{drag force from acoustic 
  streaming} -- AS) für eine ununterbrochene und eine gepulste akustische 
Anregung; 2) die Quantifizierung der stationären Rotationsgeschwindigkeit eines 
sphärischen Partikels, welches durch das akustisch-viskose Drehmoment (engl: 
\emph{acoustic viscous torque}) angetrieben wird und welches eine viskose 
Grenzschichtdicke (engl: \emph{viscous boundary layer thickness} -- VBL) hat, 
die so gross wie der Partikelradius selbst ist.

Bisher sind OT vor allem als Kraftsensoren für einzelne Partikel innerhalb von 
MSAF Geräten eingesetzt worden. Für die Messung der genannten Phänomene wird 
die genaue örtliche und zeitliche Auflösung der OT verwendet, wie auch die 
Möglichkeit Messungen an einzelnen Partikeln durchzuführen.

Für die Messung des Aufbaus der ARF und des AS wurde eine akustische 
Anregungsfrequenz und Messpunkte innerhalb der stehenden Druckwelle verwendet, 
bei deren Kombination die ARF und die Kräfte von AS senkrecht zueinander waren.  
Die Orthogonalität als auch die Aufteilung in ARF und AS wurde durch mehrere 
Kraftmessungen innerhalb des ganzen Fluidkanals mit unterschiedlichen 
Partikelgrössen validiert.

Die Ergebnisse der ununterbrochenen Anregung zeigten, dass der Aufbau der ARF 
sofort nach Beginn der akustischen Anregung beginnt, wohingegen AS deutlich 
langsamer ist. Der zügige Aufbau der ARF wird von der Theorie so vorhergesagt, 
während der Aufbau des AS um den Faktor 4 unterschätzt wird. Die Experimente 
mit einer gepulsten Anregung zeigten, dass abhängig von den Pulsparametern der 
Aufbau von AS weitestgehend verhindert werden kann, aber der Aufbau der ARF 
nicht im gleichen Masse beeinträchtigt wird. Daher werden kleinere Partikel zum 
grössten Teil wegen der ARF manipuliert, weil die relative Wichtigkeit von AS 
für eine gepulste Anregung schrumpft. Unsere Experimente bestärken andere 
experimentelle Ergebnisse mit einer gepulsten Anregung, welche bisher durch 
keine Theorie erklärt werden können.

Für die Messung der stationären Rotationsgeschwindigkeit wurde eine Mischung 
von Wasser und Glycerol (7 zu 3) hergestellt, dessen Viskosität so gross war, 
dass die VBL um das Partikel herum in etwa so gross wie der Partikelradius 
selbst war. Die Phasenverschiebung zwischen den beiden akustischen Feldern, die 
räumlich senkrecht zueinander standen, führte zu einem zeitgemittelten AS Feld 
innerhalb der VBL in Richtung des Umfangs des Partikels. Dieses AS Feld ist der 
Ursprung für ein antreibendes akustisch viskoses Drehmoment, welches das 
Partikel bis zu der Rotationsgeschwindigkeit beschleunigt, wo das 
Antriebsmoment gleich dem entgegengerichteten viskosen Widerstandsmoment ist. 

Eine theoretische Formel überschätzt die stationäre Rotationsgeschwindigkeit 
für diese experimentellen Parameter mit mehr als einer Grössenordnung. Dieser 
Fehler war erwartet, weil es bis jetzt noch keine Theorie gibt, welche für den 
Fall gilt, wo der Partikelradius so gross oder grösser ist wie die VBL selbst.  
Eine numerische Studie untersuchte unter anderem diesen Parameterbereich für 
sphärische Partikel und gab eine Formel für die stationäre 
Rotationsgeschwindigkeit an, welche die viskosen Effekte der VBL 
berücksichtigt. Die mit der OT gemessenen Rotationsgeschwindigkeiten 
bestätigten zwei Punkte: 1) die erwartete Ungültigkeit der vereinfachten 
Theorie für grosse Fluidviskositäten (VBL in der gleichen Grössenordnung wie 
das Partikel) und daher die notwendige Berücksichtigung der Viskosität für 
diesen Bereich; 2) die Richtigkeit der numerischen Ergebnisse.



\cleardoublepage
\renewcommand{\relPath}{SECTION/30_Timeconstant}
 
\chapter[Dynamic Timeconstant Measurement]{Dynamic Measurement of the Acoustic 
Streaming Time Constant utilizing an Optical Tweezer}\label{ch:timeconstant}
\textit{This chapter is original work by Christoph Goering:
\footnote{: DOI: 10.1103/PhysRevE.104.025104, reproduced under the terms of the 
Creative Commons Attribution 4.0 license.}}

\vspace{5mm} \noindent
C. Goering and J. Dual, "Dynamic measurement of the acoustic streaming time 
constant utilizing an optical tweezer", Physical Review E, 2021, \textbf{104}, 
025104.


\section{Abstract}
The combination of a bulk acoustic wave device and an optical trap allows 
studying the build up time of the respective acoustic forces. In particular, we 
are interested in the time it takes to build up the acoustic radiation force 
and acoustic streaming. For that, we measure the trajectory of a spherical 
particle in an acoustic field over time. The shape of the trajectory is 
determined by the acoustic radiation force and by acoustic streaming; both 
acting on different time scales. For that, we utilize the high temporal 
resolution ($\Dt = \SI{0.8}{\us}$) of an optical trapping setup. With our 
experimental parameters the acoustic radiation force on the particle and the 
acoustic streaming field theoretically have characteristic build up times of 
\SI{1.4}{\us} and \SI{1.44}{\ms}, respectively. By choosing a resonance mode 
and a measurement position where the acoustic radiation force and acoustic 
streaming induced viscous drag force act in orthogonal directions, we can 
measure the evolution of these effects separately. Our results show, that the 
particle is accelerated nearly instantaneously by the acoustic radiation force 
to a constant velocity, whereas the acceleration phase to a constant velocity 
by the acoustic streaming field takes significantly longer. We find that the 
acceleration to a constant velocity induced by streaming takes in average about 
17'500 excitation periods ($\approx \SI{4.4}{\ms}$) longer to develop than the 
one induced by the acoustic radiation force. This duration is about 4 times 
larger than the so-called momentum diffusion time which is used to estimate the 
streaming build up. In addition, this rather large difference in time can 
explain why a pulsed acoustic excitation can indeed prevent acoustic streaming 
as it has been shown in some previous experiments.

\section{Introduction\label{sec:introduction}}

In recent years, acoustofluidics has provided many powerful tools. Due to being 
contact-less, label-free, and biocompatible 
\cite{Antfolk2015,Abdulla2020,Zielke2020,Binkley2020,Cai2020}, acoustofluidic 
manipulation can be used in medical applications for cancer research
\cite{Antfolk2015,Abdulla2020,Zielke2020,Binkley2020}, Alzheimer research 
\cite{Cai2020}, targeted drug delivery \cite{Bose2015}, and for pumping medical 
fluids \cite{Wu2019}. In addition, there are biological 
\cite{Gerlt2020,Xie2019} and engineering applications (e.g., micro-pumping 
\cite{Wu2019,Huang2014,Lin2019,Ozcelik2021}).

Most of these applications utilize the acoustic radiation force (ARF) to 
manipulate objects on the micro-scale. The ARF is a second-order time-averaged 
effect that arises from the interaction of an acoustic field scattered at an 
object surface and a background acoustic field 
\cite{Doinikov1994,Hasegawa1969,Yosioka1955,Gorkov1962,Bruus2012}.
These objects can be solid particles, air bubbles, fluid droplets, biological 
samples, as long as their material properties (density $\rho$ and speed of 
sound $c$) are different from the surrounding medium. However, there coexists 
a fluid motion called acoustic streaming (AS) 
\cite{Nyborg1965,Kolb1956,Nyborg1953}. This motion can arise either from
viscous losses in the fluid (Eckhart type streaming \cite{Eckart1948}) or it 
can arise in the viscous boundary layer at a fluid to wall interface 
(Schlichting and Rayleigh streaming \cite{Riley1998,Schlichting1932}).


The theoretical derivations usually describe the steady-state of the AS field. 
A theoretical numerical study \cite{Muller2015} investigated the temporal build 
up of the ARF and AS field. In contrast to the ARF, the viscous drag force 
arising from AS is independent of the object material properties because it is 
a motion of the fluid. The AS direction coincides with the direction of the 
relative motion between fluid and particle.

For a spherical object of radius $R$, the drag force in laminar flow scales 
linearly with the object radius $\FAS \propto R$. In contrast to the $\FAS$, 
the ARF scales with the volume $\FARF \propto \R^{3}$ \cite{Bruus2012-10}.  
Based on the fluid and the object material properties, the $\FARF$ will 
dominate over the $\FAS$ if the radius $\R$ is greater than the critical radius 
$\R_{\text{crit}}$, where $\FAS = \FARF$ holds. The direction of $\FAS$ can be 
different from the $\FARF$. Therefore, the $\FAS$ is usually undesired.

The ARF and the AS occur not only in the bulk of the fluid, but also on sharp 
edges of a device \cite{Doinikov2020a,Doinikov2020b,Leibacher2015,Nama2016}. 
So-called micro-streaming around the surface of a spherical particle can even 
cause a sign inversion of the ARF if the viscous boundary layer $\delta$ is 
sufficiently large \cite{Baasch2019}. However, there are applications that take 
advantage of the AS \cite{Antfolk2014,Mao2017,Hao2020}: a complete overview of 
AS applications can be found in \cite{Wiklund2012a}.

In literature, it is well understood how long it takes until the acoustic 
field, and hence the ARF, needs to build up \cite{Muller2015} and how long the 
particle focusing takes \cite{Bruus2012-10}. However, it is still not fully 
clear how long it takes for the AS to build up, and what the definition for the 
analytical AS time constant is. In the acoustofluidics community, it is 
generally accepted that the build up for the AS field takes longer than the 
build up of the ARF. By using a pulsed actuation of the acoustic field and 
therefore exploiting this time offset, \citeauthor{Hoyos2013} prevented the 
build up of AS \cite{Hoyos2013,Castro2016}. They varied the number of periods 
for which the acoustic actuation is switched on and off, respectively. They 
experimentally showed that for a ratio of about 1 to 1 between 500 on- and 500 
off-periods the streaming velocity is less than 50\% of its steady-state 
magnitude while the ARF is not affected by that much.

\citeauthor{Muller2015} studied the build up of the acoustic energy density and 
streaming velocity with a numerical model \cite{Muller2015}. Their model 
consisted of a fluid cavity without any surrounding structure such as the 
cavity walls. They found numerically that indeed the ARF builds up 
significantly faster than the AS. However, the simulations with a pulsed 
actuation of different ratios of on- to off-periods did not prevent the build 
up of AS because its decay -- as the build up -- is slow compared to the ARF. 
The streaming builds up significantly slower during the on-periods, however, it 
does not decay to its initial value during the off-periods. Over time the 
influence of AS increases because the ARF alternates between some magnitude in 
the on-periods and zero in the off-periods. This implies, that the simulation 
of \citeauthor{Muller2015} could not explain the experimental results by 
\citeauthor{Hoyos2013}.

In this work, we experimentally measure the time until a \Dtwo~spherical 
silicon-dioxide (\SiO) particle moves with constant velocity when accelerated 
by the ARF and AS. Instead of using a camera, we utilize a data acquisition 
board (DAQ) with a sampling frequency of $f_{\text{s}} = \SI{1.25}{\MHz}$ to 
measure the relative particle trajectory as soon as the ultra-sound (US) is 
switched on. This high sampling frequency $f_{\text{s}}$ yields a high 
temporal resolution of $ \Dt = \SI{0.8}{\us}$. Considering the acoustic 
excitation frequency $\fex = \SI{4.015}{\MHz}$, we sample at least every fourth 
excitation period.

The optical tweezer (OT) for this study has already been successfully applied 
in the fields of acoustofluidics for stationary force measurements within a 
microfluidic chip \cite{Lamprecht2016,Lakaemper2015} as well as acoustic 
viscous torque investigations \cite{Lamprecht2021}. Here, we characterize in a 
first step the stationary force field in the bulk of the device to ensure, that 
we measure in a second step the time resolved build up of AS and the ARF 
separately and not their superposition. The separation is done by choosing a 
particle position within the acoustic field, where the $\FAS$ and $\FARF$ are 
orthogonal to each other. In order to measure in the second step solely the 
effects of the acoustic field on the particle and not the characteristics of 
the OT, we alter the usual trapping setup. The modification is that the 
particle is released from the OT before the acoustic excitation starts and 
retrapped after it.  Hence, during the measurement just gravity and the forces 
of the acoustic field act upon the particle. With our modified trapping setup, 
we are able to measure precisely the ARF and AS induced movement of a single 
particle in the bulk of the fluid.

Our manuscript is structured as follows: in \cref{sec:theory} we derive and 
list all time constants in our system and we compute the traveled distances of 
a free floating particle in an acoustic field. Those influences need to be 
considered for our measurement protocol. In addition, we perform numerical AS 
simulations of our device to further understand the AS field; in 
\cref{sec:experimental-setup} we explain our experimental setup and its 
modifications; in \cref{sec:experimental-procedure} we show the results of the 
stationary force measurement, before explaining our time evolution measurement 
protocol and the data post-processing; and in \cref{sec:results} we show and 
discuss the results of this study.




\section{Theory \label{sec:VT-theory}}

Two orthogonal standing waves excited at the same frequency $f$ and with a 
relative phase shift $\zeta$ exert a torque on spherical particles. Under these 
conditions, an acoustic streaming field is formed inside the viscous boundary 
layer

\begin{equation}
    \delta = \sqrt{\frac{\mu_{f}}{\rho_{f}\,\pi\,f}}
    \label{eq:VT-delta}
\end{equation}

at the fluid-particle interface, where $f$ is the frequency (of excitation), 
$\mu_{f}$ the dynamic fluid viscosity, and $\rho_{f}$ the density of the fluid. 
The resulting viscous surface stress on the particle results in a non-zero 
torque. This torque is called the acoustic VT and is qualitatively shown in 
\cref{fig:VT-Fig1} for two orthogonal standing waves with a phase shift of 
$\zeta = \sfrac{\pi}{2}$.

%%%%%%%%%%%
\begin{figure}
    \centering
    \includegraphics[width=100mm]{\relPath/10_Figures/Fig1.png}
    \caption{Schematic of the time-averaged acoustic VT acting on a sphere. At a 
    constant rotational rate $\Omega$ of the particle the propulsive acoustic 
  VT is in balance with its counteracting viscous drag 
  torque.\label{fig:VT-Fig1}}
\end{figure}%
%%%%%%%%%%%

The analytical solution for the total time-averaged VT 
$\Gamma_{\text{tot}}(\Omega)$ on a small rotating spherical particle 
($R\ll\lambda=\sfrac{c_f}{f}$) within two orthogonal plane standing waves is 

%%%%%%%%%%%%%%%%%%%
\begin{equation}
  \label{eq:VT-Eq1}
  \begin{split}
      &\Gamma_{\text{tot}}(\Omega) = \\
      &= \frac{3}{4} \frac{\delta S_s A_{X} A_{Y}}{\rho_{f} c_{f}^{2}} \sin\zeta \cos(kX) \cos(kY) - 8 \pi \mu_f R^3\,\Omega \\
     &= \Gamma_{\text{IN}} - \tilde{D}\,\Omega
   \end{split}
 \end{equation}
%%%%%%%%%%%%%%%%%%%
where $S_s$ is the sphere surface area, and $c_f$ the speed of sound of the 
fluid, $k=\sfrac{2\pi}{\lambda}$ the wavenumber in the fluid, and ($A_{X}$; 
$A_{Y}$) the pressure amplitudes of the two orthogonal standing waves 
\cite{Wang1989, Lamprecht2013}.  The phase shift $\zeta$ and the sphere position 
($X$;$Y$) determine the rotation direction.  The result of 
$\Gamma_{\text{tot}}(\Omega)$ can be split up into a torque driven by the 
acoustic excitation $\Gamma_{\text{IN}} \propto R^{2}$ and a viscous drag torque 
$\tilde{D}\,\Omega \propto R^{3}$ related to Stokes drag \cite{Lamprecht2013}. In 
the theory of \citeauthor{Nyborg1958} \cite{Nyborg1958} and \citeauthor{Wang1989} 
\cite{Wang1989}, the particle rotation was not considered in their analysis of the 
acoustic VT $\Gamma_{\text{IN}}$.  However, \citeauthor{Lamprecht2013} 
\cite{Lamprecht2013} introduced a moving boundary of the particle so that the 
driving $\Gamma_{\text{IN}}$ appears independently of the rotational rate 
$\Omega$. The rotational axis of the particle is always perpendicular to both 
directions of the incident waves and its rotational rate is limited by Stokes 
drag coefficient $\tilde{D} = 8 \pi \mu_f R^3$. The steady-state rotational rate 
$\finalOmega$ is defined as
%%%%%%%%%%%%%%%%%%%
\begin{equation}
  \label{eq:VT-AcGovEqConti}
  \finalOmega=\frac{\Gamma_{\text{IN}}}{\tilde{D}}.
\end{equation}
%%%%%%%%%%%%%%%%%%%
Since $\finalOmega \propto \sfrac{1}{R}$ \cite{Lamprecht2013}, bigger particles will 
reach a lower steady-state rotational rate $\finalOmega$. This occurs for the 
equilibrium state $\Gamma_{\text{IN}}(t=t^\star)= \tilde{D} (t = t^\star )$.  
After $ t^\star = \SI{0.5}{\milli\second}$ a \SI{100}{\micro\meter} large 
particle rotates with the steady-state rate of $\SI{11.33}{\hertz}$
(\SI{680}{\rpm}) at an acoustic pressure amplitude of \SI{171}{\kilo\pascal} 
\cite{Lamprecht2015}. Please note that the time constant $\tau \approx 
t^{\star}$ is proportional to $R^2$ \cite{Lamprecht2015}, so a particle with 
$2\,R=\SI{2.06}{\um}$ reaches the equilibrium in less than 
\SI{1}{\micro\second}. \citeauthor{Hahn2016} \cite{Hahn2016} showed numerically 
that the analytical and the numerical results can differ by orders of magnitude 
for $\normBdLayer > 1$ and water as fluid. For acoustic particle manipulation in 
water, the differences between the analytical solution and the numerical results 
become negligible for particles with a normalized viscous boundary layer 
$\normBdLayer < \sfrac{1}{15}$. In addition, the analytical predictions neglect 
the density ratio $\tilde{\rho} = \sfrac{\rho_{\text{s}}}{\rho_{\text{f}}}$ 
between the particle and the surrounding fluid.  \citeauthor{Hahn2016} 
\cite{Hahn2016} concluded that the density ratio $\tilde{\rho}$ has a 
non-negligible effect on the magnitude of the acoustic VT and that depending on 
this ratio even the direction of the torque can change.

\section{Experimental Setup\label{sec:TC-experimental-setup}}

\subsection{Optical Trap Setup}

Our OT has already been applied in several other publications 
\cite{Lamprecht2021,Lamprecht2017,Lamprecht2016,Lakaemper2015} to the field of 
ARF and AS measurements in bulk acoustic wave (BAW) devices. All components are 
described there extensively. We highlight here the position detection system 
and the modifications from the force measurement setup that were necessary for 
this study. These modifications are needed because we use one \SI{785}{\nm} 
near-infrared diode laser (LuxX 785-200, Omicron Laser, Rodgau-Dudenhofen, 
Germany) for the optical trapping and also for the optical position detection.  
We detect the position of the trapped particle relative to the trap center by 
monitoring the voltages of two quadrant photo diodes (QPD) placed in the back 
focal plane (see \cref{fig:TC-laserpath}). It is possible to resolve the movement 
of the particle in all three dimensions. However, the in-plane ($xy$) and the 
axial ($z$) position detection differ.

\begin{figure}[H]
  \centering
  \def\svgwidth{85mm}
  \svginput{\relPath/10_Figures/LaTeX/Microscope.pdf_tex}
  \caption{Shutter location in laser path and schematic of laser path through 
  microscope setup; full details on the setup in 
  \cite{Lamprecht2016,Lamprecht2017}.}\label{fig:TC-laserpath}
\end{figure}

For the $xy$ position the laser beam is focused onto the QPDxy (see also 
\cref{fig:TC-laserpath}) such that the spot diameter is about five times smaller 
than the opening aperture of the QPDxy \cite{Lamprecht2017}. An in-plane 
movement of the trapped particle changes the spot location on the QPDxy which 
results in a voltage change on the four quadrants. As long as the spot is 
within the QPD opening the spatial movement is linearly related to the QPD 
voltage. By summing and subtracting these four voltages from each other, one 
can get the values corresponding to a movement along $\ex$ and $\ey$ separately 
\cite{Lamprecht2017}.

For the axial $z$ position a second QPD is needed and this QPD is overfilled 
with the laser spot (see also \cref{fig:TC-laserpath}). When the particle moves 
axially the spot diameter changes its size. If the diameter decreases more 
intensity is measured by QPDz and leads to a higher voltage and vice versa. For 
small movements ($\Dz < \R$) the relation is linear \cite{Dreyer2004}.

When converting the measured voltage changes from QPDxy and QPDz to the 
particle displacement with unit of meters, the $xy$ voltage and the $z$ voltage 
have different scaling. The three voltage-meter conversion factors are found by 
calibrating the OT via the power spectrum analysis of the trapped particle 
Brownian motion \cite{Lamprecht2021,Lamprecht2016,Lakaemper2015}.

As discussed before, the time constant $\tOT$ of the OT is larger than the 
time constant for the ARF and AS (see \cref{tab:TC-time-constants}). Therefore, we 
need to switch the laser off and then monitor the particle trajectory without 
the trapping forces, in order to measure the time evolution of the particle 
movement and not the time constant of the optical trap. This means, the particle 
is not stably trapped while measuring. However, we need the laser light for 
the position detection on the QPDs. Therefore, we reduce the laser power to a 
minimum such that the resulting trapping forces are negligibly small. As a 
result of the low power, the voltage magnitude decreases significantly on the 
QPDs, such that it is not measurable anymore. Thus, we exchange the neutral 
density (ND) filters from the force measurement setup 
\cite{Lamprecht2016,Lamprecht2021} with the fast optical shutter FOS-NIR(1100) 
(LC TEC, Borlänge, Sweden). This filter is specified to open from 0\% to 90\% 
transmittance in less than \SI{15}{\ms} and close from 100\% transmittance to 
10\% in less than \SI{5}{\ms}. The ND filters and the shutter are needed to 
reduce the intensity on the QPDs and prevent overexposure and hence damage. 
The transmittance of the shutter can be controlled with the applied driving 
voltage. Before and after the measurements the shutter is almost completely 
closed to mimic the ND filters and opened for the actual measurement with 
reduced laser power.

Lastly, we operate the laser in the so-called \emph{analogue modulation} mode 
such that the output laser power is proportional to an externally applied DC 
voltage which is sampled with more than \SI{1.5}{\MHz} by the laser controller 
unit. The low power mode for the position detection is operated with less than 
\SI{0.5}{\mW}. The low voltage DC signal for the laser averaged \SI{84.13}{\mV} 
with a standard deviation of \SI{0.13}{\mV} providing a very consistent voltage 
and hence laser power. With this power the trapping potential is too weak to 
keep the particle inside the focus of the laser beam in any of the three 
spatial directions. The usual laser power for the stationary force measurements 
is between \SI{100}{\mW} and \SI{175}{\mW}.

\begin{figure}[H]
  \centering
  % \tikzsetnextfilename{daq-sync}
\begin{tikzpicture}
  \tikzset{nodestyle/.style={pos=0.0, above, anchor=south west}}
  % \draw[white, fill=black!10!white] (2.5,0) rectangle ++(1.5,3.5);
  % \draw[|<->|] (2.5,3.5) -- ++(1.5,0) node[midway,above=2mm,align=center] 
  % {\small Shutter open\\US on\\laser power low};

  % \draw[|<->|] (1.0, 0.5) -- ++(6.0,0) node[midway, above] {particle is free to 
  % float and move};
  % axis system
  \draw[thick,->] (-0.2,0) -- ++(6.5,0) node[anchor=north west] {$t$ 
  [\si{\milli\second}]};

  % time ticks
  \draw (0, 2pt) -- ++(0, -4pt) node[anchor=north] {$-25$};
  \draw (1.0, 2pt) -- ++(0, -4pt) node[anchor=north] {$-24$};
  \draw (2.5, 2pt) -- ++(0, -4pt) node[anchor=north] {$0$};
  \draw (4.0, 2pt) -- ++(0, -4pt) node[anchor=north] {$30$};
  \draw (5.0, 2pt) -- ++(0, -4pt) node[anchor=north] {$55$};
  \draw (6.0, 2pt) -- ++(0, -4pt) node[anchor=north] {$75$};

  % laser power
  \draw[dashed] (-0.5,2.7) -- node[nodestyle] {$P_{\text{trapping}}$} ++(1.5,0) 
  -- ++(0,-1.0) -- node[nodestyle] {$P_{\text{measure}}$} ++(5.0,0) -- 
  ++(0,1.0) -- ++(1,0) node[nodestyle] {laser};

  % shutter
  \draw[dotted] (-0.5,4) -- node[nodestyle] {closed} ++(0.5,0) -- ++(1.5,-1.5) 
  -- node[nodestyle] {open} ++(2.5,0) -- ++(1.5, 1.5) -- ++(1.5,0) 
  node[nodestyle] {shutter};

  % US
  \draw[] (-0.5, 0.2) -- node[nodestyle] {off} ++(3.0,0) -- ++(0,1) -- 
  node[nodestyle] {on} ++(2.5,0) -- ++(0,-1) -- node[nodestyle] {US} ++(1.5,0);


\end{tikzpicture}

  \includegraphics[]{Plots/cache/daq-sync.eps}
  \caption{Schematic of controller timings for the shutter, the laser, and the 
      US. During the $P_{\text{measure}}$ state the particle is not trapped by 
  the OT. In the time interval $[\SI{0}{\ms}, \SI{30}{\ms}]$ (1) the shutter is 
  fully opened, (2) the US is switched on, and (3) the particle is free to 
  move. During this interval the measurement is performed.}\label{fig:TC-daq-sync}
\end{figure}

\subsection{Controller Timing and Data Acquisition}

The data acquisition (DAQ) board NI-USB 6356 (National Instruments, Austin, TX, 
USA), the laser power, the piezo excitation voltage, and the shutter 
transmittance are actuated in a defined sequence. We use an Arduino Board with 
two 12-Bit DAC units (MCP4725, Adafruit, New York, NY, USA) for controlling the 
timing and the DC voltage for the laser. The timings are depicted in 
\cref{fig:TC-daq-sync}. For $t<-\SI{24}{\ms}$ the laser is in its high power state 
and keeps the particle fixed in position against external forces. At 
$t=\SI{-25}{\ms}$ the shutter starts opening. The opening time is specified 
with less than \SI{15}{\ms} from 0\% transmittance to 90\%. At 
$t=\SI{-24}{\ms}$ the laser power changes to its low power state. Hence, the 
particle is free to move and starts its sedimentation. At $t=\SI{0}{\ms}$ the 
US is switched on. For \SI{30}{\ms} the shutter is fully opened, the particle 
is free to move, and the US is on. Then the shutter starts to close again. In 
these \SI{30}{\ms} we measure the time evolution of the particle. At 
$t=\SI{55}{\ms}$ the US is switched off and at $t=\SI{75}{\ms}$ the laser power 
is increased to its high power state. The time between two consecutive 
measurements is greater than \SI{2}{\s}, such that the fluid within the cavity 
is fully at rest again.

\subsection{Device, Particles, and Fluid}

Our device is a glass-silicon-glass device manufactured by Gesim GmbH 
(Radeberg, Germany). The material of the two glasses is B33 from Schott (Mainz, 
Germany). A sketch is shown in \cref{fig:TC-device} and its dimensions are listed 
in \cref{tab:TC-device-dimensions}. The top glass and the fluid cavity are limited 
in the $\ez$ direction because our microscope setup cannot focus deeper than 
\SI{250}{\um} \cite{Lamprecht2016,Lamprecht2017}. We define the origin of our 
coordinate system so that $z = 0$ is in the middle of the fluid cavity and $y = 
0$ is in the middle between the silicon cavity walls. We use as a reference 
point $x = 0$ such that it is approximately in the middle of the PZT length 
$l$. For all reported measurements we use the same position $x_{\text{ref}}$ as 
reference for $x=0$.

The fluid cavity is in the middle between the two silicon layers and the 
PZT is a PZ 26 element from Meggit A/S (Kvistgaard, 
Denmark). It is glued with Epo-Tek (Billerica, MA, USA) H20S two component 
epoxy onto the device. It is located at the edge of the device in $\ey$ 
direction and centered along the long side. The small height of 
the PZT is necessary to prevent physical contact with the microscope lens.

Our particles are silicon-dioxide (\SiO) particles from (microParticles GmbH, 
Berlin, Germany) with a diameter of $D_{2}=\SI{2.06}{\um}$. For the device 
characterization we also use particles from the same manufacturer with the same 
material properties, but with a diameter of $D_{4} = \SI{4.39}{\um}$. The 
particles are immersed in filtered (\SI{0.2}{\um}) and distilled water. To 
avoid particle-particle interactions during the experiment, we keep the 
particle concentration low.

We use the \Dtwo~particles because they are the smallest particles that work 
well in our OT. In addition, the critical radius where the ARF equals the drag 
force from AS can be found via \cite{Barnkob2012}
\begin{equation}
  \R_{\text{crit}} = \sqrt{\frac{3}{\Phi}}\,\delta
\end{equation}
where $\Phi$ is the acoustic contrast factor with thermoviscous correction 
\cite{Settnes2012}




\begin{subequations}
\begin{eqnarray}
  \Phi\left( \tkappa, \trho, \tdelta \right) &=& \frac{1}{3} f_{1}\left( 
  \tkappa \right) + \frac{1}{2}\,\text{Re}\left[ f_{2}\left( \trho, 
  \tdelta\right) \right],\\
  %%%%%%
  f_{1}\left( \tkappa \right) &=& 1 - \tkappa, \quad 
  \tkappa=\frac{\kappa_{\text{p}}}{\kappa_{\text{f}}},\\
  %%%%%%
  f_{2}\left( \trho, \tdelta \right) &=& \frac{2\left[ 1-\Gamma\left( \tdelta 
  \right) \right]\left( \trho-1 \right)}{2\,\trho + 1 - 3\,\Gamma\left( \tdelta 
  \right)}, \quad \trho=\frac{\rhop}{\rhof}\\
  %%%%%%
  \Gamma\left( \tdelta \right) &=& -\frac{3}{2}\left[ 1 + \iu \left( 1 + 
  \tdelta \right) \right]\tdelta, \quad \tdelta = \frac{\delta}{R}, \quad 
  \delta = \sqrt{\frac{\muef}{\rhof\,\pi f}}.
%
\end{eqnarray}
\end{subequations}
Here $\kappa_{\text{p}}$ is the particle and $\kappa_{\text{f}}$ the fluid 
compressibility, $\delta$ the viscous boundary layer thickness, and $\iu$ the 
imaginary unit. For our parameters (see \cref{tab:TC-parameters}) 
$\R_{\text{crit}} $ is equal to \SI{0.63}{\um} and \SI{0.65}{\um}, with and 
without ($\tdelta = 0$) thermoviscous correction, respectively.

With increasing particle size, two effects take place: 1) the ratio between ARF 
($\propto \R^{3}$) and AS ($\FAS\propto \R$) magnitude increases, because of 
their respective scaling, and 2) the measurement time decreases, because a 
greater ARF leads to more displacement, which in turn makes re-trapping more 
difficult.

% \begin{equation}
%   \delta = \sqrt{\frac{\muef}{\pi\,\rhof\,\fex}}
% \end{equation}

% \begin{equation}
%   \Phi = \frac{1}{3}\left[ \frac{5\,\tilde{\rho}-2}{2\,\tilde{\rho}+1} - 
%   \tilde{\kappa} \right]
% \end{equation}



\section{Experimental Procedure\label{sec:TC-experimental-procedure}}
\afterpage{

\begin{figure}[H]
  \centering
  \includegraphics[width=\figWidthDouble]{\relPath/10_Figures/4um.pdf}
  % \input{10_Figures/PGF/4um_map.pgf}
  \caption{Measured steady-state acoustic forces for a \Dfour~particle with 
    $\fex=\SI{4.015}{\MHz}$ and $V_{\text{pp}} = \SI{10.7}{\volt}$. The top row 
    depicts the forces along $\ey$ and the bottom along $\ez$. The two columns 
    correspond to two different measurement $yz$-planes at $x=\SI{-0.1}{\mm}$ 
  and $x=\SI{0.1}{\mm}$, respectively.}\label{fig:TC-4um-map}
\end{figure}

\begin{figure}[H]
  \centering
  % \input{10_Figures/PGF/2um_map.pgf}
  \includegraphics[width=\figWidthDouble]{\relPath/10_Figures/2um.pdf}
  \caption{Measured steady-state acoustic forces for a \Dtwo~particle with 
    $\fex=\SI{4.015}{\MHz}$ and $V_{\text{pp}} = \SI{10.7}{\volt}$. The top row 
    depicts the forces along $\ey$ and the bottom along $\ez$. The two columns 
  correspond to two different measurement $yz$-planes at $x=\SI{-0.1}{\mm}$ and 
$x=\SI{0.1}{\mm}$, respectively.}\label{fig:TC-2um-map}
\end{figure}
\clearpage
}

\subsection{Stationary Force Measurement}
In preparation for the time evolution measurement, where a spatial position of 
orthogonal AS forces and ARFs is beneficial, we characterized our device with 
two sets of stationary force measurements at a constant excitation frequency. 
For those measurements the optical trapping force is greater than the acoustic 
forces. One measurement was with a \Dtwo, and the other with a \Dfour~diameter 
particle. For changing the particle size we needed to empty and refill the 
device.  We kept the ambient conditions and experiment settings between the two 
measurements as constant as possible. For the measurements with the 
\Dtwo~particle the ambient temperature was \SI{24.49}{\celsius} in average with 
a standard deviation of \SI{0.10}{\celsius} and for the measurement with the 
\Dfour~particle the average temperature was \SI{24.78}{\celsius} with a 
standard deviation of \SI{0.25}{\celsius} ensuring the same experimental 
conditions for both measurements. More details regarding the protocol of those 
measurements can be found in \cite{Lamprecht2016} by 
\citeauthor{Lamprecht2016}.

We defined two $yz$ measurement planes, with $x_{1} = \SI{-0.1}{\mm}$ and 
$x_{2} = \SI{0.1}{\mm}$, respectively. In each plane we defined a grid in 
$y_{i}\in\{-0.20,0.19,\dots,0.20\}\,\si{\mm}$ and 
$z_{j}\in\{-30,-20,\dots,30\}\,\si{\um}$. At each point $(y_{i}, z_{j})$ we 
measured the forces in all three dimensions 5 times for \SI{3}{\second} each.  
Our excitation frequency was set to $\fex = \SI{4.015}{\MHz}$ and the applied 
voltage was $U_{\text{pp}} = \SI{10.7}{\volt}$. We choose $\fex$ based on a 
frequency sweep and the corresponding maximal forces in this sweep. With the 
chosen $\fex$ and the fluid speed of sound $\cfl \approx 
\SI{1500}{\meter\per\second}$, we obtain the theoretical acoustic wavelength of 
$\lap = \sfrac{\cfl}{\fex} \approx \SI{375}{\um}$. Hence, with the frequency 
$\fex$ and a channel width of $W = \SI{3}{\mm}$, 16 pressure nodal lines are 
present. For each spatial position we averaged the forces over the 
\SI{3}{\second} timespan and also over the 5 repetitions.

\Cref{fig:TC-4um-map,fig:TC-2um-map} visualize stationary force measurement 
results as contour plots for the two particle sizes. In addition, 
\Cref{fig:TC-averaged_forces_vs_dy} depicts the measured forces in $\ey$ and 
$\ez$ directions, when the data is additionally averaged over the 7 different 
heights $\Dz$. For \Cref{subfig:TC-F_y,subfig:TC-F_z}, the left vertical axis 
is the scale for the \Dfour~particle and the right vertical axis for 
\Dtwo~particles.

In \Cref{subfig:TC-F_y} the force wavelength $\laF$ is estimated to be 
\SI{180}{\um} which is in line with the theoretical wavelength $\laF = 
\sfrac{\lap}{2}$. One can also note that the shape of two force measurements is 
consistent. The ratio of the mean maximal force amplitudes $\frac{1.25}{0.17} = 
7.13$ is about the same as the ratio of the cubed diameter
\begin{equation}
  {\left( \frac{\Dfour}{\Dtwo} \right)}^{3} \approx 2.13^{3} \approx 9.68
 \label{eq:TC-ARF-AS-scaling}
\end{equation}
Based on the theoretical scaling laws we conclude that the forces in the $\ey$ 
direction are ARF dominant.


In \Cref{subfig:TC-F_z} one can see the measured forces in $\ez$ for both particle 
sizes and both measurement $yz$ planes. As for the forces in $\ey$ direction, 
in \Cref{subfig:TC-F_y}, the forces in $\ez$ direction are averaged over all 
$\Dz$. The force magnitude for both sizes is smaller than in $\ey$ direction 
for both particle sizes. The shapes, however, are similar but not as consistent 
as in \Cref{subfig:TC-F_y}. The ratio of the mean maximal force amplitudes 
$\frac{0.25}{0.08} \approx 3.1$ is about the same as the ratio of the two 
diameters, which suggests that in the $\ez$ direction the forces on the 
particle are AS dominated (see \Cref{eq:TC-ARF-AS-scaling}).

\begin{figure}[H]
  \centering
  \begin{subfigure}{\figWidth}
    \centering
    \caption{$F_{y}$ [\si{\pico\newton}]}\label{subfig:TC-F_y}
    % \tikzsetnextfilename{avgF_y_vs_dy}
\begin{tikzpicture}
  \begin{axis}[%
      scale only axis,
      width = 60mm,
      height = 5cm,
      axis y line*=left,
      legend style={
        fill=blue!10!white,
        font=\tiny,
        at={(0.03,0.05)},
        anchor=south west},
      xlabel = {$\Dy$ [\si{\mm}]}]

    \fill[fill=black!15!white] ({axis cs:-0.06,-2}|-{rel axis cs:0,0}) 
    rectangle ({axis cs:-0.03,2}|-{rel axis cs:0,1});

    \addlegendimage{empty legend}
    \addlegendentry{\hspace{-.6cm}\textbf{$\Rfour$}}

    \addplot[thick, blue] table[x=dy, y=F4_y1] 
    {\relPath/10_Figures/TikZ/averaged_yz_Forces.dat};
    \addlegendentry{$x_{1}$};

    \addplot[thick, blue, dashed] table[x=dy, y=F4_y2] 
    {\relPath/10_Figures/TikZ/averaged_yz_Forces.dat};
    \addlegendentry{$x_{2}$};


    \draw[|<->|] ({axis cs:-0.135,0}|-{rel axis cs:0,0.95}) -- ({axis 
    cs:0.05,0}|-{rel axis cs:0,0.95}) node[midway,below] 
    {$\sfrac{\lap}{2}=\laF$};


  \end{axis}
  \pgfplotsset{every axis y label/.append style={rotate=180,yshift=86mm}}
  \begin{axis}[%
      scale only axis,
      width = 60mm,
      height = 5cm,
      legend style={
        fill=lightgray,
        font=\tiny,
        at={(0.97,0.95)},
        anchor=north east},
      axis x line=none,
    axis y line*=right]

    \addlegendimage{empty legend}
    \addlegendentry{\hspace{-.6cm}\textbf{$\Rtwo$}}

    \addplot[thick, dotted] table[x=dy, y=F2_y1] 
    {\relPath/10_Figures/TikZ/averaged_yz_Forces.dat};
    \addlegendentry{$x_{1}$};

    \addplot[thick,loosely dashed] table[x=dy, y=F2_y2] 
    {\relPath/10_Figures/TikZ/averaged_yz_Forces.dat};
    \addlegendentry{$x_{2}$};

  \end{axis}
\end{tikzpicture}

    % \includegraphics[width=\subfigWidth]{Plots/cache/avgF_y_vs_dy.eps}
    \includegraphics[]{Plots/cache/avgF_y_vs_dy.pdf}
  \end{subfigure}%
  \begin{subfigure}{\figWidth}
    \centering
    % \tikzsetnextfilename{avgF_z_vs_dy}
\begin{tikzpicture}
  \begin{axis}[%
      scale only axis,
      width = 60mm,
      height = 5cm,
      axis y line*=left,
      legend style={
        fill=blue!10!white,
        font=\tiny,
        at={(0.03,0.05)},
        anchor=south west},
      xlabel = {$\Dy$ [\si{\mm}]}]

    \fill[fill=black!15!white] ({axis cs:-0.06,-2}|-{rel axis cs:0,0}) 
    rectangle ({axis cs:-0.03,2}|-{rel axis cs:0,1});

    \addlegendimage{empty legend}
    \addlegendentry{\hspace{-.6cm}\textbf{$\Rfour$}}

    \addplot[thick,blue] table[x=dy, y=F4_z1] 
    {\relPath/10_Figures/TikZ/averaged_yz_Forces.dat};
    \addlegendentry{$x_{1}$};

    \addplot[thick, blue, dashed] table[x=dy, y=F4_z2] 
    {\relPath/10_Figures/TikZ/averaged_yz_Forces.dat};
    \addlegendentry{$x_{2}$};

  \end{axis}
  \pgfplotsset{every axis y label/.append style={rotate=180,yshift=86mm}}
  \begin{axis}[%
      scale only axis,
      width = 60mm,
      height = 5cm,
    axis y line*=right,
      yticklabel style={
        /pgf/number format/fixed,
        /pgf/number format/precision=2
      },
      legend style={
        fill=lightgray,
        font=\tiny,
        at={(0.97,0.95)},
        anchor=north east},
      axis x line=none]

    \addlegendimage{empty legend}
    \addlegendentry{\hspace{-.6cm}\textbf{$\Rtwo$}}

    \addplot[thick, dotted] table[x=dy, y=F2_z1] 
    {\relPath/10_Figures/TikZ/averaged_yz_Forces.dat};
    \addlegendentry{$x_{1}$};

    \addplot[thick,loosely dashed] table[x=dy, y=F2_z2] 
    {\relPath/10_Figures/TikZ/averaged_yz_Forces.dat};
    \addlegendentry{$x_{2}$};

  \end{axis}
\end{tikzpicture}

    % \includegraphics[width=\subfigWidth]{Plots/cache/avgF_z_vs_dy.eps}
    \caption{$F_{z}$ [\si{\pico\newton}]}\label{subfig:TC-F_z}
    \includegraphics[]{Plots/cache/avgF_z_vs_dy.pdf}
  \end{subfigure}%
  \caption{Measured steady-state acoustic forces when averaged over the cavity 
    height. All values are in \si{\pico\newton}. For each plot the left 
    $y$-axis is the measured force on the \Dfour~($D_{4}$) particle and the 
    right one for the \Dtwo~($D_{2}$) particle, respectively.
    The gray shaded area corresponds to the positions where the time evolution 
  is measured.}\label{fig:TC-averaged_forces_vs_dy}
\end{figure}

\subsection{Measurement Protocol for Time Evolution}

Based on a set of proof-of-concept experiments (data not shown here) and the 
information from numerical simulations that the AS field in a \emph{real} 
device can substantially differ from the AS field of fluid cavity-only 
structure, we selected $x = 0$, $y_{i} \in 
\{-0.15,-0.14,\dots,0.10\}\,\si{\mm}$, and $z_{j} \in \{-10,0,10\}\,\si{\um}$. 
This choice means, that we measure at the same $y_{i}$ and $z_{j}$ as for the 
stationary force measurement. We have the same excitation frequency ($\fex = 
\SI{4.015}{\MHz}$) as in the stationary force measurements from before. However 
we set the excitation amplitude slightly higher to $U_{\text{pp}} = 
\SI{11.7}{\volt}$ in order to increase the signal to noise ratio (SNR).

We control the whole measuring routine with a self-written Python program. 
Before each measurement, the offset of the QPDs is checked and, if needed, 
adjusted. First we measure without US and then we measure with US on. We repeat 
this procedure 50 times before moving to the next location.

For the time evolution measurement, we acquire with a sampling rate of $\fs 
=\SI{1.25}{\MHz}$ ($\Dt = \SI{0.8}{\us}$) for \SI{125}{\ms} the three QPD 
signals, the signal for the shutter, and the DC signal for the laser as soon as 
the shutter starts opening ($t = \SI{-25}{\ms}$ in \Cref{fig:TC-daq-sync}). 
Between $t =\SI{0}{\ms}$ and $t = \SI{30}{\ms}$ the shutter is completely open 
and the US is switched on. Extending the measurement time further has no 
benefit because the particle will be outside the linear regimes of the QPDs and 
might move too far from the OT trapping region such that it cannot be 
recaptured after the laser changes to its high power state again.

We repeat 50 times per position because the particle starts sedimenting 
as soon as the laser power drops to the lower value. During this movement the 
particle still undergoes Brownian motion. Hence, the trajectory is not straight 
along the $\ez$ direction. With 50 datasets, we can average this random 
movement out.

Taking the approximation of \cref{eq:TC-mod-free-fall} into account, a \Dtwo~large 
\SiO~sphere sedimenting in water reaches its terminal velocity 
almost instantaneously, because the inertia term is small; additionally, the 
sphere travels about $0.12\,\Rtwo$ in \SI{55}{\ms}. Therefore, after 
\SI{25}{\ms} the particle is still in the linear regime of the QPDz. The static 
gravitational force ($\tilde{m}g$) with the added buoyancy of water is less 
than \SI{40}{\femto\newton} for the \Dtwo~particle. This is more than 6 times 
smaller than the maximal measured force in $\ez$ direction. Therefore, we 
assume in areas of maximal forces along $\ez$ that the driving force of this 
movement is either the acoustic field or $\FAS$. With an ideal sedimentation in 
the first \SI{25}{\ms} along $\ez$, the laser spot on QPDxy does not change at 
all during the sedimentation.

\subsection{Data Processing}

The acquired data is postprocessed with Python. We look at discrete points 
every $t_{k} = k\cdot\SI{0.1}{\ms}$ with $k\in \mathbb{N}$. In addition, we use 
a moving average for the data at $t_{k}$ with a centered window size of 101 
data points, corresponding to a timespan of \SI{80}{\us}. Next, we subtract the 
data series without US from the series with US to obtain the delta voltage 
$\DV_{m}$, with $m$ being $y$ or $z$. This quantity allows us to further reduce 
unwanted noise. This step serves also as data quality check because all 
measurements have the same protocol until $t=\SI{0}{\ms}$. Hence, the delta 
voltage $\DV_{m}$ must be \emph{zero} for $t\leq\SI{0}{\ms}$. Then, we average 
$\DV_{m}$ over the 50 repetitions per spatial position $y_{i}, z_{j}$. As last 
step for the time evolution plots, we normalize the data by the $\max\left( 
\left\vert \DV_{m}(t)\right\vert \right)$ for $\SI{10}{\ms} < t < 
\SI{30}{\ms}$.

\section{Results and Discussion\label{sec:TC-results}}

\Cref{subfig:TC-res_DV_y} shows that the maximal averaged voltage difference 
$\DV_{y}$ for the \Dtwo~particle while having the leaser in the 
$P_{\text{measure}}$ mode. It has the same shape as the stationary force 
measurement in \cref{subfig:TC-F_y}.  However, the smoothness of $\DV_{y}$ is 
worse. We attribute this to the nature of the experiment, as the recorded 
motion of the particle is caused by two effects; one is the acoustic field and 
the other is the always present Brownian motion.  For the stationary force 
measurements the particle is fixed in place by the optical potential and the 
Brownian motion is negligible.

By measuring the same shape with the two experiments, we could validate our 
measurement protocol. As for the stationary measurements, the SNR of the 
evolution measurement and also shape are better for the in-plane $\ey$ than the 
axial $\ez$ (see \cref{subfig:TC-F_y} and \cref{subfig:TC-F_z}). Nevertheless, 
\cref{subfig:TC-F_z} and \cref{subfig:TC-res_DV_z} also show similar shapes. We 
want to stress again, that the amplitudes of \cref{fig:TC-DV_vs_dy} are not 
comparable to each other for $\ey$ and $\ez$ (see 
\cref{sec:TC-experimental-setup}).
% This step enables data comparability, because the raw magnitudes are 
% inherently different. As stated before, the in-plane position detection along 
% $\ex$ and $\ey$ functions differently than the axial along $\ez$.

The numerical streaming simulations of a fluid cavity with and without the
surrounding structure showed that the streaming field is a local effect in a
model with surrounding structure. In our experiments we saw similar tendencies.  
However, not all measured spatial locations had enough actual signal strength 
to further investigate. In \Cref{fig:TC-evolutioin-V} we plot the time evolution 
of the signal for four different $\Dy$ where it is clear that the signal is due 
to the acoustic field and not to noise or Brownian motion.

\begin{figure}[ht]
  \centering
  \begin{subfigure}{\figWidth}
    \centering
    \caption{Data for $y$-component ($m = y$)}\label{subfig:TC-res_DV_y}
    % \tikzsetnextfilename{avgV_y_vs_dy}
\begin{tikzpicture}
  \begin{axis}[%
      scale only axis,
      width = 60mm,
      height = 45mm,
      xticklabel style={
        /pgf/number format/fixed,
        /pgf/number format/precision=2
      },
      legend style={
        fill=lightgray,
        font=\tiny,
        at={(0.97,0.05)},
        anchor=south east
      },
      legend cell align={left},
      ylabel={$\max\left( \DV_{m}\left( t \right) \right)$ [\si{\mV}]},
      xlabel = {$\Dy$ [\si{\mm}]}]

      \fill[fill=black!15!white] ({axis cs:-0.065,-0.0002}|-{rel axis cs:0,0}) 
      rectangle ({axis cs:-0.025,0.0002}|-{rel axis cs:0,1});

    \addplot[thick,mark=*,mark size=1pt] table[x=dy, y=DV_y_m10] 
    {\relPath/10_Figures/TikZ/averaged_yz_mVoltages.dat};
    \addlegendentry{$\Dz = -10$};

    \addplot[thick, dotted,mark=*,mark size=1pt] table[x=dy, y=DV_y_m00] 
    {\relPath/10_Figures/TikZ/averaged_yz_mVoltages.dat};
    \addlegendentry{$\Dz = 0$};

    \addplot[thick, dashed,mark=*,mark size=1pt] table[x=dy, y=DV_y_p10] 
    {\relPath/10_Figures/TikZ/averaged_yz_mVoltages.dat};
    \addlegendentry{$\Dz = +10$};

    % wavelength
    \draw[|<->|] ({axis cs:-0.135,0}|-{rel axis cs:0,0.55}) -- ({axis 
    cs:0.05,0}|-{rel axis cs:0,0.55}) node[midway,above] 
    {$\sfrac{\lap}{2}=\laF$};

  \end{axis}
\end{tikzpicture}

    \includegraphics[]{/avgV_y_vs_dy.pdf}
  \end{subfigure}%
  \begin{subfigure}{\figWidth}
    \centering
    \caption{Data for $z$-component ($m = z$)}\label{subfig:TC-res_DV_z}
    % \tikzsetnextfilename{avgV_z_vs_dy}
\begin{tikzpicture}
  \begin{axis}[%
      scale only axis,
      width = 60mm,
      height = 45mm,
      xticklabel style={
        /pgf/number format/fixed,
        /pgf/number format/precision=2
      },
      legend style={
        fill=lightgray,
        font=\tiny,
        at={(0.97,0.05)},
        anchor=south east
      },
      legend cell align={left},
      xlabel = {$\Dy$ [\si{\mm}]}]

      \fill[fill=black!15!white] ({axis cs:-0.065,-0.0002}|-{rel axis cs:0,0}) 
      rectangle ({axis cs:-0.025,0.0002}|-{rel axis cs:0,1});

    \addplot[thick,mark=*,mark size=1pt] table[x=dy, y=DV_z_m10] 
    {\relPath/10_Figures/TikZ/averaged_yz_mVoltages.dat};
    \addlegendentry{$\Dz = -10$};

    \addplot[thick, dotted,mark=*,mark size=1pt] table[x=dy, y=DV_z_m00] 
    {\relPath/10_Figures/TikZ/averaged_yz_mVoltages.dat};
    \addlegendentry{$\Dz = 0$};

    \addplot[thick, dashed,mark=*,mark size=1pt] table[x=dy, y=DV_z_p10] 
    {\relPath/10_Figures/TikZ/averaged_yz_mVoltages.dat};
    \addlegendentry{$\Dz = +10$};

  \end{axis}
\end{tikzpicture}

    \includegraphics[]{/avgV_z_vs_dy.pdf}
  \end{subfigure}%
  \caption{Maximal $\DV_{y}$ and $\DV_{z}$ averaged over all repetitions in the 
    timespan between \SI{35}{\ms} and \SI{55}{\ms} for the three different 
    measurement heights $\Dz = \SIlist[list-units=single, list-final-separator 
    = {, }, list-pair-separator= {, }] {-10;0;10}{\um}$. The gray shaded area 
    represents the $\Dy_{i}$ of best signal strength for $\max\left( 
    \DV_{z}\left( t \right) \right)$. The data points of best strength are 
    taken for the time evolution results. The wavelength marker represents the 
  same length as in \cref{subfig:TC-F_y}.}\label{fig:TC-DV_vs_dy}
\end{figure}%

Since we show $\DV_{m}$ rather than the absolute voltage amplitudes, we can 
further validate our protocol. For $\sfrac{t}{t_{0}} < 0$, where $t_{0} = 
\sfrac{1}{\fex}$ and $\sfrac{t}{t_{0}}=0$ represents the time when the US is 
switched on (in \cref{fig:TC-daq-sync} $t = \SI{0}{\ms}$), all data series in 
\cref{fig:TC-evolutioin-V} are zero. All data series for $\ez$ are more noisy than 
for $\ey$. However, we also have the same amplitude of noise in $\ey$ 
direction. But, the normalization value for the data series for $\ey$ is 
inherently larger than for $\ez$ (see \cref{fig:TC-DV_vs_dy}).

\afterpage{
\begin{figure}[ht]
  \centering
  % \tikzsetnextfilename{evolution_V}
%%%%%%%
% READ TABLE
%%%%%%%
\pgfplotstableread{\relPath/10_Figures/TikZ/evolution_yz_Voltages.dat}{\data}
%%%%%%%
% LINES FOR ALL GROUPPLOTS
%%%%%%%
\renewcommand{\tikzHelper}{
  \fill[fill=black!10!white] (axis cs:-80,0) rectangle (axis cs:0,1);

  \draw[dotted] (axis cs:0,0) -- (axis cs:0,1);
  \draw[dotted] (axis cs:50,0) -- (axis cs:50,1);
  \draw[dotted] (axis cs:100,0) -- (axis cs:100,1);
  \draw[dotted] (axis cs:-80,0.5) -- (axis cs:120,0.5);
  \draw[dotted] (axis cs:-80,0.5) -- (axis cs:120,0.5);
}



\begin{tikzpicture}
   \begin{groupplot}[%
       scale only axis,
       group style={
         group size= 2 by 4,
         group name=plots,
         vertical sep=4pt,%
         horizontal sep=8pt},%
       height=40mm,%
       width=64mm,%
        xticklabel style={
          /pgf/number format/fixed,
          /pgf/number format/precision=2
        }]

%%%%%%
%%% PLOT (1,1)
%%%%%%

   \nextgroupplot[%
      legend style={
        fill=lightgray,
        font=\tiny,
        at={(0.03,0.95)},
        anchor=north west
      },
      legend cell align={left},
     xticklabels={,,},
     % title={$\DV_{y}\,|\,\Dy = \SI{-0.06}{\milli\meter}$},%
     title={Data for $y$-component ($m = y$)},%
     ylabel={$\sfrac{\DV_{m}}{\DV_{m,\text{max}}}$}]

      \tikzHelper
      \draw[thick,|<->|] (axis cs:-80,0.25) -- (axis cs:0,0.25) node[midway, 
      above] {US off};

      \addplot[thick] table[x=dt, y=DV_y_m06_m10] {\data};

      \addplot[thick, dotted] table[x=dt, y=DV_y_m06_m00] {\data};

      \addplot[thick, dashed] table[x=dt, y=DV_y_m06_p10] {\data};

      \addlegendentry{$\Dz = \SI{-10}{\um}$};
      \addlegendentry{$\Dz = \SI{0}{\um}$};
      \addlegendentry{$\Dz = \SI{+10}{\um}$};



%%%%%%
%%% PLOT (1,2)
%%%%%%

   \nextgroupplot[%
     xticklabels={,,},
     yticklabels={,,},
     title={Data for $z$-component ($m = z$)}]%
   % title={$\DV_{z}\,|\,\Dy = \SI{-0.06}{\milli\meter}$}]

      \tikzHelper

      \addplot[thick] table[x=dt, y=DV_z_m06_m10] {\data};

      \addplot[thick, dotted] table[x=dt, y=DV_z_m06_m00] {\data};

      \addplot[thick, dashed] table[x=dt, y=DV_z_m06_p10] {\data};

%%%%%%
%%% PLOT (2,1)
%%%%%%

   \nextgroupplot[%
     xticklabels={,,},
     ylabel={$\sfrac{\DV_{m}}{\DV_{m,\text{max}}}$ },
     % title={$\Dy = \SI{-0.05}{\milli\meter}$}
   ]

      \tikzHelper

      \addplot[thick] table[x=dt, y=DV_y_m05_m10] {\data};

      \addplot[thick, dotted] table[x=dt, y=DV_y_m05_m00] {\data};

      \addplot[thick, dashed] table[x=dt, y=DV_y_m05_p10] {\data};

%%%%%%
%%% PLOT (2,2)
%%%%%%

   \nextgroupplot[%
     xticklabels={,,},
     yticklabels={,,},
     % title={$\Dy = \SI{-0.05}{\milli\meter}$}
   ]

      \tikzHelper

      \addplot[thick] table[x=dt, y=DV_z_m05_m10] {\data};

      \addplot[thick, dotted] table[x=dt, y=DV_z_m05_m00] {\data};

      \addplot[thick, dashed] table[x=dt, y=DV_z_m05_p10] {\data};

%%%%%%
%%% PLOT (3,1)
%%%%%%

   \nextgroupplot[%
     xticklabels={,,},
     ylabel={$\sfrac{\DV_{m}}{\DV_{m,\text{max}}}$ },
     % title={$\Dy = \SI{-0.04}{\milli\meter}$}
   ]

      \tikzHelper

      \addplot[thick] table[x=dt, y=DV_y_m04_m10] {\data};

      \addplot[thick, dotted] table[x=dt, y=DV_y_m04_m00] {\data};

      \addplot[thick, dashed] table[x=dt, y=DV_y_m04_p10] {\data};

%%%%%%
%%% PLOT (3,2)
%%%%%%

   \nextgroupplot[%
     xticklabels={,,},
     yticklabels={,,},
     % title={$\Dy = \SI{-0.04}{\milli\meter}$}
   ]

      \tikzHelper

      \addplot[thick] table[x=dt, y=DV_z_m04_m10] {\data};

      \addplot[thick, dotted] table[x=dt, y=DV_z_m04_m00] {\data};

      \addplot[thick, dashed] table[x=dt, y=DV_z_m04_p10] {\data};


%%%%%%
%%% PLOT (4,1)
%%%%%%

   \nextgroupplot[%
     ylabel={$\sfrac{\DV_{m}}{\DV_{m,\text{max}}}$},
     xlabel={$10^{3}\,\sfrac{t}{t_{0}}$ },
     % title={$\Dy = \SI{-0.03}{\milli\meter}$}
   ]

      \tikzHelper

      \addplot[thick] table[x=dt, y=DV_y_m03_m10] {\data};

      \addplot[thick, dotted] table[x=dt, y=DV_y_m03_m00] {\data};

      \addplot[thick, dashed] table[x=dt, y=DV_y_m03_p10] {\data};

%%%%%%
%%% PLOT (4,2)
%%%%%%

   \nextgroupplot[%
     yticklabels={,,},
     xlabel={$10^{3}\,\sfrac{t}{t_{0}}$},
     % title={$\Dy = \SI{-0.03}{\milli\meter}$}
   ]

      \tikzHelper

      \addplot[thick] table[x=dt, y=DV_z_m03_m10] {\data};

      \addplot[thick, dotted] table[x=dt, y=DV_z_m03_m00] {\data};

  \end{groupplot}

%%%%%%
%%% TEXT NEXT TO PLOTS
%%%%%%
  \node[rotate=90] at (plots c2r1.east) [yshift=-5mm] {$\Dy = 
  \SI{-0.06}{\milli\meter}$};
  \node[rotate=90] at (plots c2r2.east) [yshift=-5mm] {$\Dy = 
  \SI{-0.05}{\milli\meter}$};
  \node[rotate=90] at (plots c2r3.east) [yshift=-5mm] {$\Dy = 
  \SI{-0.04}{\milli\meter}$};
  \node[rotate=90] at (plots c2r4.east) [yshift=-5mm] {$\Dy = 
  \SI{-0.03}{\milli\meter}$};

\end{tikzpicture}

  \includegraphics[]{/evolution_V.pdf}
  \caption{Time evolution of the normalized $\DV_{y}$ (left column) and 
    $\DV_{z}$ (right column) for the three measurement heights $\Dz = 
    \SIlist[list-units=single, list-final-separator = {, }, 
    list-pair-separator= {, }] {-10;0;10}{\um}$ and the positions for $\Dy = 
    \SIlist[list-units=single, list-final-separator = {, }, 
    list-pair-separator= {, }] {-0.06;-0.05;-0.04;-0.03}{\mm}$. The gray shaded 
    area of each plot marks the time when the US is off; 
  $t_{0}=\sfrac{1}{\fex}$.}\label{fig:TC-evolutioin-V}
\end{figure}
}

For all 12 positions $(y_{i}, z_{j})$ in \cref{fig:TC-evolutioin-V} the signal 
along $\ey$ starts changing as soon as the US is switched on. This 
is in line with the estimation of \cref{eq:TC-tau-arf} for $\tarf$. For all data 
series $m = z$ it takes significantly more time until the movement with 
constant velocity starts. To further compare the results, we take as criteria 
the period $p^{\ast} = \sfrac{t^{\ast}}{t_{0}}$, when the normalized $\DV_{m} 
\ge 0.5$ is reached. In \cref{tab:TC-results}, the absolute periods for this 
criteria and the offset between the movement along $\ey$ ($m=z$) and the 
movement along $\ez$ ($m=y$) are shown. Taking a different criteria value 
(e.g., the normalized $\DV_{m} \ge 0.3$) changes the absolute magnitude of the 
values $p^{\ast}$, however the offset does not change significantly. The 
average for all $\Delta p^{\ast}$ is about 17'500 which equates to $\approx 
\SI{4.35}{\ms}$ for the excitation frequency $\fex = \SI{4.015}{\MHz}$.

\begin{table*}
  \centering
  \footnotesize
  \begin{tabular}{ll *{4}{x{27mm}}}
    \toprule
    \toprule
  {\bfseries $\Dy$} & [\si{\mm}] & -0.06 & -0.05 & -0.04 & -0.03 \\

    \midrule
    % {\small
  {\bfseries $p^{\ast}_{y}$ } & ($\times 1000$) [-] & 64.2, 69.5, 70.7 & 65.8, 
  70.3, 70.3 & 65.8, 60.6, 76.7 & 73.5, 57.4, 74.3 \\[2mm]

  {\bfseries $p^{\ast}_{z}$} & ($\times 1000$) [-] & 80.7, 87.9 88.3 & 86.3, 
  85.5, 86.7 & 83.1, 89.5, 87.5 & 87.9, 86.7,  \\

    \midrule
    
  {\bfseries $\Delta p^{\ast}$} & ($\times 1000$) [-] & 16.5, 18.4, 17.6 & 
  20.5, 15.2, 16.4 & 17.3, 18.9, 10.8 & 14.4, 29.3, \\
    % }
    \bottomrule
    \bottomrule
    
  \end{tabular}
  \caption{Absolute periods $p^{\ast}_{m}$ when the normalized $\DV_{m} > 0.5$.  
    The three values per column correspond to the three heights $\Dz = 
    \SIlist[list-units=single, list-final-separator = {, }, 
    list-pair-separator= {, }] {-10;0;10}{\um}$ per $\Dy$ respectively. For 
  $\Dy = \SI{-0.03}{\mm}$ and $\Dz = \SI{10}{\um}$ no data is available for 
$p_{z}^{\ast}$. The last row states the offset $\Delta p^{\ast} = p^{\ast}_{z} 
- p^{\ast}_{y}$}\label{tab:TC-results}
\end{table*}

In addition, all slopes for the $y$ movement ($m=y$) are linear almost 
immediately after the US is
switched on. This suggests, that the ARF is constant and accelerates the 
particle fast to its terminal velocity. The measured voltages and also their 
differences are linearly related to the traveled distances. Hence, a constant 
increase in voltage, which means a constant voltage increase per time 
$\sfrac{\text{d}\,\DV_{m}}{\text{d}t} = const.$, implies a constant particle 
speed along the $\ey$ direction. The particle trajectory in $\ez$ direction is 
predominantly affected by the streaming field. This fluid motion takes more 
time until it is established. With the same reasoning as before, a linear slope 
for the $z$ movement ($m=z$) in \cref{fig:TC-evolutioin-V} implies a constant 
force and constant particle speed. A constant speed means a non-changing 
streaming field and therefore a constant streaming velocity.


\section{Conclusion\label{sec:TC-conclusion}}

In this work we presented the measurement of the temporal evolution of the AS 
field and the ARF in a BAW device utilizing an OT. We slightly modified our 
validated optical trapping setup \cite{Lamprecht2016,Lamprecht2021} to 
accommodate the requirements of this experiment. With a temporal resolution of 
$\Dt=\SI{0.8}{\us}$ we could measure at least every fourth time period of 
excitation. We validated our measurement protocol against the stationary force 
field.

We monitored the trajectory of a \Dtwo~\SiO~particle as soon as the US 
excitation of the device started. We selected measurement positions in a 
standing pressure wave mode where ARF dominates in one direction and AS 
orthogonal to it. In addition, we chose the spatial location within the mode 
to maximize the amplitude of both effects. Our measurements show, that the ARF 
is established almost immediately after the US is switched on; whereas the AS 
takes in average 17'500 excitation periods (\SI{4.4}{\ms}) longer to evolve. 
This time is about four times larger than the theoretical approximation with 
the momentum diffusion time.

These results show that the build up of AS takes significantly longer than the 
build up of the ARF. This temporal difference can explain why a pulsed acoustic 
excitation can prevent streaming as it has been experimentally shown by 
\citeauthor{Hoyos2013} \cite{Hoyos2013,Castro2016}. In addition, the results of 
the streaming simulations of a cavity-only model and a whole-device model show 
that simplified models are enough for simulations of the pressure fields, 
however they cannot reflect \emph{real} streaming patterns. This insight might 
also explain why \citeauthor{Muller2015} could not reproduce the suppression of 
AS with a pulsed excitation in their cavity-only model.



% \appendix
% \begin{appendix}
% %\iffalse %Beginn langer Kommentar
\chapter[Appendix]{Supplemental to Chapter 5\label{ch:app-pulsing}}

% \end{appendix}
% \cleardoublepage

% \renewcommand{\bibname}{References} %Rename chapter to References
\addcontentsline{toc}{chapter}{References} %add references to Outline
% \bibliographystyle{siam}
% \bibliography{All}
\printbibliography


\setlength{\parindent}{0mm} %Abschnitt-Einzug auf 0 setzen
\renewcommand{\\}{\newline} %damit wird wieder \\ \\ möglich für doppelte Zeilenabstände


% \chapter*{List of Publications}
\markboth{List of Publications}{List of Publications}                     % heading
\addcontentsline{toc}{chapter}{List of Publications}          % list in table 


\makeatletter
\DeclareCiteCommand{\fullcite}
  {\defcounter{maxnames}{\blx@maxbibnames}%
    \usebibmacro{prenote}}
  {\usedriver
     {\DeclareNameAlias{sortname}{default}}
     {\thefield{entrytype}}}
  {\multicitedelim}
  {\usebibmacro{postnote}}
\DeclareCiteCommand{\footfullcite}[\mkbibfootnote]
  {\defcounter{maxnames}{\blx@maxbibnames}%
    \usebibmacro{prenote}}
  {\usedriver
     {\DeclareNameAlias{sortname}{default}}
     {\thefield{entrytype}}}
  {\multicitedelim}
  {\usebibmacro{postnote}}
\makeatotherts

%as first author}
\addsec{Publications in Peer-Reviewed Scientific Journals}
\begin{itemize}
  \item \fullcite{Lamprecht2021}\footnote{A.L. and C.G. shared first author}
  \item \fullcite{Goering2021}
  \item \fullcite{Goering2022}
  \item \fullcite{Fankhauser2022}\footnote{J.F. and C.G. shared first author}
\end{itemize}


\addsec{Oral Presentations at International Conferences}
C. Goering, A. Lamprecht, I.A.T. Schaap, and J. Dual.  \emph{"Direct 
Measurement of Small Spherical Particle Rotation driven by the Acoustic Viscous 
Torque"} Acoustics Virtually Everywhere, 07-11 December 2020, Virtual 
Conference.\\
  \\
C. Goering and J. Dual.\emph{"Measuring the temporal difference in build up 
between the acoustic radiation force and acoustic streaming with an optical 
tweezer."} Acoustofluidics, 26-27 August 2021, Virtual Conference.\\

% \makeatletter
\newcommand\tabfill[1]{%
  \dimen@\linewidth
  \advance\dimen@\@totalleftmargin
  \advance\dimen@-\dimen\@curtab
  \parbox[t]\dimen@{#1\ifhmode\strut\fi}%
}
\makeatother

\begin{floatingfigure}{4.5cm} %Abstand vom rechten Rand (formerly 3.5)
\centerline{ \includegraphics[width=3.7cm]{SECTION/Portrait.png}} 
\label{portrait}
\end{floatingfigure}

\noindent\textbf{ \\ \underline{Christoph} Ludwig Georg Ananda Goering} \\
\begin{flushleft}
\noindent  Born on 17$^\mathrm{th}$ of August, 1991 in Augsburg, Germany \\
\end{flushleft}

\vspace{2.8cm} \noindent
\textbf{Education}                                        \\
\vspace{-0.5cm} \noindent


\begin{tabbing}
  \hspace{25mm} \=   \kill


2018--2022 \> \tabfill{PhD student in Prof. J. Dual's group, Institute of 
Mechanical Systems, Swiss Federal Institute of Technology (ETH) Zurich, 
Switzerland}\\
2015--2017 \> \tabfill{M.Sc. in mechanical engineering at the Technical 
University of Munich (TUM), Germany.}\\
2011--2014 \> \tabfill{B.Eng. in \emph{Maschinenbau -- Konstruktion und 
Entwicklung} at Dualen Hochschule Baden-Württemberg (DHBW), Ravensburg, 
Germany.}\\
2002--2011 \> \tabfill{Gymnasium bei St. Stephan, Augsburg; graduation with the 
German Abitur.}\\
\end{tabbing}

\vspace{-0.5cm} \noindent
\textbf{Professional experience}\\
\vspace{-0.8cm} \noindent
\begin{tabbing}
  \hspace{25mm} \=   \kill
  2015--2017 \> \tabfill{Werksstudent at BMW AG, Munich, Germany.}\\
  2011--2014 \> \tabfill{Dualer Student at Zeppelin Systems GmbH, 
  Friedrichshafen, Germany.}\\
  2011--2017 \> \tabfill{Auxiliary staff at MABRIS, Augsburg, Germany.}\\
\end{tabbing}

\vspace{-0.5cm} \noindent
\textbf{Extra curricular activities}\\
\vspace{-0.8cm} \noindent
\begin{tabbing}
  \hspace{25mm} \=   \kill
  2012--2015 \> \tabfill{Member of Global Formula Racing (GFR); winner of 
  multiple events overall in the Formula Student.}\\
  1991-- \> \tabfill{Being outdoors.}\\

\end{tabbing}


\end{document}
